
%\begin{tabular}{@{\ttfamily}l l}
\begin{tabular}{p{6cm}p{7.2cm}}
\texttt{\#} & Introduces a comment\\
% int i & Defines an integer variable called \texttt{i}\\
% int x, y & Defines integer variables called \texttt{x} and \texttt{y}\\
% boolean b & Defines a boolean variable called \texttt{b}\\
% String s & Defines a string variable called \texttt{s}\\
\texttt{num = 99} & Assigns the integer value 99 to a variable called \texttt{num}\\
\texttt{word = input()} & Takes a string from the input and puts it
into the variable \texttt{word} \\
\texttt{n = int(word)} & Parses the number in the variable
\texttt{word} and puts its integer value into the variable \texttt{n}
\\
\texttt{word = str(n)} & Converts the number in the variable
\texttt{n} to its string representation and puts it into the variable \texttt{word} \\
\texttt{print(word)} & Outputs the value of \texttt{word}\\
\texttt{print(word, end = "")} & The same, but without the return-of-line at the end\\
\texttt{print("The answer is " + x)} & Outputs ``The answer is '' followed by the string \texttt{x}\\
%x = 99 & Assigns the value 99 to \texttt{x}\\
\texttt{+}, \texttt{-}, \texttt{*}, \texttt{//}, \texttt{\%} & Arithmetic operators.  \texttt{*}, \texttt{//} and \texttt{\%} take precedence over \texttt{+} and \texttt{-}.\\
\texttt{len(s)} & Gives length of string \texttt{s}\\
\texttt{s[x:y]} & Gives substring of \texttt{s} starting with the \\
                  & character at position \texttt{x} and ending with \\
                  & the last character before position \texttt{y}.\\
                  & (The first character of the string is at \\ 
                  & position zero.)\\
\\
\verb!if A:!    & If \texttt{A} is True,\\
\verb!    B!    & \quad do \texttt{B},\\
\verb!elif C:!  & else if \texttt{C} is True,\\
\verb!    D!    & \quad do \texttt{D},\\
\verb!else:!    & else\\
\verb!    E!    & \quad do \texttt{E}.\\
\\
\texttt{x == y}          & Tests whether \texttt{x} is equal to
\texttt{y}.  Note the double ``\texttt{=}''.\\
\verb+!=+, \verb+>+, \verb+<+, \verb+>=+, \verb+<=+ 
                  & Not equal to, greater than, less than, etc.\\
% \{ & Used to bracket together separate \\ 
% \} & statements to make a block\\
\\
\verb!while A:! & While \texttt{A} is True, do \texttt{B}. \texttt{B} can be a block\\
\verb!    B!   &     with several statements.\\
\\
\verb!True!, \verb!False! & The boolean values \emph{True} and \emph{False}.\\
\verb+not+, \verb+and+, \verb+or+  & The boolean operators \emph{not}, \emph{and} and \emph{or}.
\end{tabular}
\vspace*{5pt}


%%% Local Variables:
%%% TeX-master: "cheatmain"
%%% End:

