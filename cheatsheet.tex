\section{Summary of the language features mentioned in this introduction}

\begin{tabular}{@{\ttfamily}l l}
// & Introduces a comment\\
int i & Defines an integer variable called \texttt{i}\\
int x, y & Defines integer variables called \texttt{x} and \texttt{y}\\
boolean b & Defines a boolean variable called \texttt{b}\\
String s & Defines a string variable called \texttt{s}\\
int num = 0 & Defines and initializes an integer variable called \texttt{num}\\
str = System.console().readLine() & Takes a string from the input
(ignoring leading spaces) \\
n = Integer.parseInt(str) & Parses the number in the string
\texttt{str} and puts its value in \texttt{n} \\
& and puts it into \texttt{str}\\
print num & Outputs the value of \texttt{num}\\
println num & The same, but adding a return-of-line at the end\\
print "The answer is " + x & Outputs "The answer is " followed by the value of \texttt{x}\\
x = 99 & Assigns the value 99 to \texttt{x}\\
+, -, *, /, \% & Arithmetic operators.  *, / and \% take precedence over + and -\\
s.length() & Gives length of string \texttt{s}\\
s.substring(x, y) & Gives substring of \texttt{s} starting at character position x \\
& and ending at character position y\\
 & (The first character of the string is character zero.)\\
if (A) B; else C; & If A is true, do B, else do C.\\
(x == y) & Tests whether \texttt{x} is equal to \texttt{y}.  Note double "="\\
!=, > <, >=, <= & Not equal to, greater than, less than, etc\\
\{ & Used to bracket together separate statements\\
\} & to make them into a block\\
while (A) B; & While A is true, do B.  B can be a block.\\
true, false & The boolean values \texttt{true} and\texttt{ false}.\\
!, \verb+&&+, or !!  & The boolean operators \emph{not}, \emph{and} and \emph{or}.\\
\end{tabular}
\vspace*{5pt}


%%% Local Variables:
%%% TeX-master: "primer"
%%% End:
