\section{Variables, identifiers and expressions}

In this section we will discuss integer numbers, arithmetic expressions, 
identifiers, comments, and strings.

%\subsection{Integer variables}
\subsection{Integer numbers}

As well as strings, we can also have integer numbers (often just
\emph{integers}) as values: whole numbers.
%Integers
So variables can contain not only strings, but also any
% \footnote{Actually, a computer's memory
%   is finite and therefore they put limits to the possible values for an
%   integer variable, but they are big enough for most programs.}
integer value like~0,~-1,~-200, or~1966.\footnote{In contrast to many
  other programming languages, Python does not have a limit to how
  large an integer value can become.}

% We would declare an \verb!int! variable called \verb!i! as follows:

% \begin{Verbatim}
%     int i
% \end{Verbatim}

We can assign an integer value, say, \verb!0!,
to a variable, say, \verb!i!:

\begin{Verbatim}
    i = 0
\end{Verbatim}

% We can change the value of an integer with an assignment:

% \begin{Verbatim}
%     i = 1
% \end{Verbatim}

or, if we had two variables \verb!i! and \verb!j!, we could assign:

\begin{Verbatim}
    i = j
\end{Verbatim}

We can even say that \verb.i. takes the result of adding two numbers
together:

\begin{Verbatim}
    i = 2 + 2
\end{Verbatim}

which results in \verb.i. having the value 4. 

\subsection*{Integers and strings}
\label{sec:intstr}

You have probably noticed that, when dealing with integer
values the symbol ``\texttt{+}'' represents addition of numbers, while -- as
we saw in the last section -- when dealing with strings the same
symbol represents concatenation. Therefore, the statement

\begin{Verbatim}
    word = 'My name is ' + 'Inigo Montoya'
\end{Verbatim}

results in \verb+word+ having the value `My name is Inigo Montoya',
while the statement

\begin{Verbatim}
    n = 10 + 7
\end{Verbatim}

results in \verb+n+ having the value 17 (not 107). What happens if we
mix integer values and string values when using ``\texttt{+}'', say in the
following statement?

\begin{Verbatim}
    answer = 'Answer: ' + 42
\end{Verbatim}

In that case, Python will give us an error message similar to the
following one:

\begin{Verbatim}
TypeError: cannot concatenate 'str' and 'int' objects
\end{Verbatim}

This means that Python does not know how to connect (``concatenate'')
the string `Answer: '\ (Python uses the name ``str'' for the
\emph{type} of strings) and the integer 42 (``int'').
To clarify that we want to use ``\texttt{+}'' to concatenate two strings (and
not to use ``\texttt{+}'' to add two values), we can convert the ``int'' value
42 to the ``str'' value `42' with the command \verb!str()!:

\begin{verbatim}
    answer = 'Answer: ' + str(42)
\end{verbatim}

%  converts the integers to strings and performs
% concatenation.
% For example,
Similarly, the small program

\begin{Verbatim}
    word = 'My name is ' + 'Inigo Montoya'
    n = 10 + 7
    text = word + ' and I am ' + str(n)
    print(text)
\end{Verbatim}

will print on the screen ``My name is Inigo Montoya and I am 17''. It
is important to know that the same symbol can be used for different
things, but we will come to this later again; now let's go back to
writing programs with a bit of maths in them using integers.

\subsection{Reading integers from the keyboard}
\label{sec:intkeyboard}

In the last section, we saw how we can read a string of characters
from the keyboard, 
using \verb+input()+. We can use the same command
to read a number\ldots but the computer will not know it is a number,
it will think it is a string of characters. If we want to tell the
computer that a sequence of characters \emph{is} a number,
we need to convert it.
This is similar to the way we needed to tell the computer that we
wanted to treat a number as a character string in the previous example.
We can do this easily by \emph{parsing} it using the command
\verb+int()+: 

\VerbatimInput[frame=single,label=Example]{src/s2Example1.py}

When we parse a string that contains an integer we obtain an integer
with the correct value. If we try to parse a string that is not an
integer (for example, the word `Tom') the program will terminate with an error
message on the screen. If we do not parse the string and use it as if
it was an integer, the results will be unpredictable. This
is a common source of errors in programs. You can check for yourself
what happens if you do not parse the string in the former example,
e.g., what happens with this program:

\VerbatimInput[frame=single,label=Example]{src/s2Example1_noParsing.py}

Now, assuming that you read your integers and always remember to parse
them, what maths can you do with them?

\subsection{Operator precedence}
\label{sec:prec}

Python uses the following arithmetic operators (amongst others):
\bigskip

\begin{tabular}{ll}
\verb!+! & addition\\
\verb!-! & subtraction\\
\verb!*! & multiplication\\
\verb!//! & division\\
\verb!%! & modulo\\
\end{tabular}
\bigskip

The last one is perhaps unfamiliar.  The result of \verb!x % y! (``x mod y'')
is the remainder that you get after dividing the integer x by the integer y.
For example, \verb!13 % 5! is \verb!3!; \verb!18 % 4! is \verb!2!; 
\verb!21 % 7! is \verb!0!, and \verb!4 % 6 is 4! (6 into 4 won't go, 
remainder 4).  \verb!num = 20 % 7! would assign the value 6
to \verb!num!.

How does the computer evaluate an expression containing more than one
operator? For example, given \verb!2 + 3 * 4!, does it do the
addition first, thus getting \verb!5 * 4!, which comes to 20, or does
it do the multiplication first, thus getting \verb!2 + 12!, which
comes to 14?  Python, in common with other programming languages and
with mathematical convention in general, gives precedence to \verb!*!, \verb!//! and
\verb!%! over \verb!+! and \verb!-!.  This means that, in the example, it does the
multiplication first and gets 14.

If the arithmetic operators are at the same level of precedence, it
takes them left to right.  \verb!10 - 5 - 2! comes to 3, not 7.  You
can always override the order of precedence by putting brackets into
the expression; \verb!(2 + 3) * 4! comes to 20, and \verb!10 - (5 - 2)! comes
to 7.

Some words of warning are needed about division. First, remember that
integer values are whole numbers.
So the result of a division with \verb-//- is
only the integer part without the decimal part. Try to
understand what the following program does:
\pagebreak

\VerbatimInput[frame=single,label=Example]{src/s2Example.py}

A computer would get into difficulty if it tried to divide by zero.
Consequently, the system makes sure that it never does.
If a program tries to get the computer to divide by zero, the program
is unceremoniously terminated, usually with an error message on the
screen.

% You may also have wondered why we write \verb!//! instead of \verb!/!
% for integer division in Python. The reason is that \verb!/! already
% stands for an operation called ``floating point division''. This
% operation does not cut off the decimal part of the result, but it will
% still need to ``round'' the result because not all decimal numbers
% (say, $1/3 = 0.333333...$) can be represented as a decimal fraction.

\subsection*{Exercise A}

Write down the output of the above program without executing it.

% \VerbatimInput[frame=single,label=Example]{src/s2Example.groovy}

Now execute it and check if you got the right values. Did you get them
all right? If you got some wrong, why was it?


\subsection{Identifiers and comments}

I said earlier that you could use more or less any names for your variables.
I now need to qualify that.

The names that the programmer invents are called \emph{identifiers}.  The
rules for forming identifiers are that the first character can be a letter
(upper or lower case, usually the latter) and subsequent characters
can be letters or digits 
or underscores.  (Actually the first character can be an underscore, but
identifiers beginning with an underscore are often used by system programs
and are best avoided.)  Other characters are not allowed.  Python is
case-sensitive, so \verb!Num!, for example, is a different identifier from
\verb!num!.

The only other restriction is that you cannot use any of the language's keywords
as an identifier.  You couldn't use \verb!if! as the name of a variable,
for example. There are many keywords but most of them are words that you
are unlikely to choose. 
Ones that you might accidentally hit upon are
\texttt{assert}, \texttt{break}, \texttt{class}, \texttt{continue},
\texttt{def}, \texttt{del}, \texttt{except}, \texttt{finally},
\texttt{global}, \texttt{import}, \texttt{pass}, \texttt{raise},
\texttt{return}, \texttt{try} and \texttt{yield}. You should also avoid 
using words which, though not technically keywords, have special significance
in the language, such as \verb!int! and \verb!print!.

Programmers often use very short names for variables, such as
\verb!i!, \verb!n!, or 
\verb!x! for integers.  There is no harm in this if the variable is used to
do an obvious job, such as counting the number of times the program goes round
a loop, and its purpose is immediately clear from the context.  If, however, its
function is not so obvious, \textbf{it should be given a name that
gives a clue as to the role it plays in the program}.  If a variable is
holding the total of a series of integers and another is holding the
largest of a series of integers, for example, then call them \verb!total!
and \verb!maxi! rather than \verb!x! and \verb!y!.

The aim in programming is to write programs that are ``self-documenting'',
meaning that a (human) reader can understand them without having to read
any supplementary documentation.  A good choice of identifiers helps to
make a program self-documenting.

Comments provide another way to help the human reader to understand a
program.  Anything on a line after ``\texttt{\#}'' is ignored by the interpreter,
so you can use this to annotate your program.  You might summarise
what a chunk of program is doing:

\begin{Verbatim}
    # sorts numbers into ascending order
\end{Verbatim}

or explain the purpose of an obscure bit:

\begin{Verbatim}
    x = x * 100 // y   # x as percent of y
\end{Verbatim}

Comments should be few and helpful.  Do not clutter your programs with
statements of the obvious such as:

\begin{Verbatim}
    num = num + 1   # add 1 to num
\end{Verbatim}

Judicious use of comments can add greatly to a program's readability, but they
are second-best to self-documentation. Note that comments are %all but
ignored by the computer: their weakness is that it is all too
easy for a programmer to modify the program but forget to make any
corresponding modification to the comments, so the comments no longer quite
go with the program. At worst, the comments can become so out-of-date as
to be positively misleading, as illustrated in this case: 

\begin{verbatim}
    num = num + 1  # decrease num
\end{verbatim}

What is wrong here? Is the comment out of date? Is there a bug in the
code and the plus should be a minus? Remember, use comments sparingly
and make them matter. A good rule of thumb is that comments should
explain ``why'' and not ``how'': the code already says \emph{how}
things are done! 

\subsection*{Exercise B}

Say for each of the following whether it is a valid identifier
in Python and, if not, why not:
\begin{Verbatim}
BBC, Python, y2k, Y2K, old, new, 3GL, a.out, 
first-choice, 2nd_choice, third_choice
\end{Verbatim}

\subsection{A bit more on strings}

We have done already many things with strings: we have
created them, printed them on the screen, even concatenated several of
them together to get a longer value. But there are many other useful
things we can do with strings.

For example, if you want to know how long a string is, you can find
out using the \emph{function} \verb!len()!.\footnote{A function
in Python is a named sequence of instructions. We 
\emph{call} the function using its name in order to execute its instructions.
We can call a function without knowing what instructions it consists of, so
long as we know what it does.}
% {\em method}\footnote{A method is a named sequence of instructions. We 
% {\em call} the method using its name in order to execute its instructions.
% We can call a method without knowing what instructions it consists of, so
% long as we know what it does.}. 
% If  \verb!str! is the string, then \verb!str.length()! gives its 
% length. Do not forget the dot or the brackets: methods are always 
% prefixed with a dot and followed by brackets\footnote{Now you can see that
%   console() and readLine() are also methods.} (sometimes empty,
% sometimes not).
You could say, for example:

\begin{Verbatim}
    print('Please enter some text: ', end = '')
    word = input()
    length = len(word)
    print('The string ' + word + ' has ' + str(length) + ' characters.')
\end{Verbatim}

You can obtain a substring (or a ``slice'')
of a string as in the %\verb!substring! method.
following example:
If the variable \verb!s! has the string value `\verb!Silverdale!', then
\verb!s[0:6]! will give you the first six letters, i.e., the string
`\verb!Silver!'.  The first number in square brackets
after the string
says where you want the substring to begin, and the second number says
where you want it to end (to be precise, the second number is the
position of the first character that we \emph{don't} want in the
substring). \emph{Note that the initial character of the string
is treated as being at position~0, not position~1}.
Similarly, \verb!s[2:6]! will give you the string \verb!lver!.

If you leave out the second number, you get the rest of the string.
For example, \verb!s[6:]! would give you `\verb!dale!', i.e., the
tail end of the string beginning at character 6 ('d' is character 6,
not 7, because the 'S' character is character 0).

You can output substrings with \verb!print! or assign them to other
strings or combine them with the ``\texttt{+}'' operator.  For example:

\VerbatimInput[frame=single,label=Example]{src/s2Example2.py}

will output `soft ices'.

\subsection*{Exercise C}

Say what the output of the following program fragment would be:

\VerbatimInput[frame=single,label=Example]{src/s2Example3.py}

Then run the program and see if you were right. 


%%% Local Variables:
%%% mode: latex
%%% TeX-master: "main"
%%% End:

