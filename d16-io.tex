
\section{Input/Output (I/O)}
\label{sec:inputoutput-io}

Computer programs are just processors of data. They take some data as
input, they return some data as output. Up to now, all the data that
our little programs have processed was either provided by a human user
or already included in the program. In the real world, however, the
most usual sources of data are computers; either the computer where
the program is running (local files), a computer in the vicinity (a
database server), or a remote computer (remote resources through the
net). 

We are going to learn now how to use the first of these three sources
of information: how to read from the local disk and how to write to
the local disk. 

\subsection{The basics}
\label{sec:basics}

\subsubsection*{File systems}
\label{sec:filesystems}

Computers have different levels of memory, but there is a clear
separation between primary and secondary memory. Primary memory is
faster but requires electricity to run; if the computer is switched
on, the contents of the memory are lost. Secondary memory is slower,
but its contents are persistent, i.e.~they are still there when the
computer is switched off and on again. Seconday memory is usually
implemented in the form of a hard disk or so-called flash
memories. It is usually referred to as ``disk'', regardless of the
actual technology used for it. 

Data is stored in secondary memory in files\footnote{This name, as
  many others, comes from an ancient era where files, archives,
  directories, and folders were physical objects that contained
  documents or data.}  
The file system is a subsystem of the operating
systems that keeps all your files in place, and provides a way of
accessing them. Usually this is done by means of hierarchies: there is
one folder/directory at the top level of the hierarchy (called the
\emph{root}), which contains some files and subdirectories, each of
these subdirectories may contain files and/or subdirectories, etc. In
order to access a file \verb+taxForm2012.odt+ inside a folder
\verb+taxes+, inside a folder \verb+MyStuff+, you may access it with a
route like: 

\begin{verbatim}
    /home/john/MyStuff/taxes/taxForm2012.odt                     (unix)
    c:\Documents and Settings\john\MyStuff\taxes\taxForm2012.odt (windows)
\end{verbatim}

Different operating systems use different conventions for the root of
the tree of subdirectories and the separation of the levels of
hieararchy. Unix systems (e.g.~Linux, MacOS/X) use a single root
(``/'') for the whole filesystem, and separate directories using a
slash (``/''). Windows use different roots, one per physycal device or
partition (C:, D:, etc) and separate directories with a 
backslash~(``\verb+\+''). 

\subsubsection*{Process}
\label{sec:process}

The process of reading from / writing to an external source always
follows the same sequence of steps:

\begin{enumerate}
\item Open the resource (e.g.~file). If it cannot be opened, finish.
\item Read from and write to the resource.
\item Close the resource. 
\end{enumerate}

This process is the basis of all interaction with external data
sources, including files, databases, and remote resources\footnote{As
  we will see, in the network resources this is usually hidden from
  the programmer.}.
It is \emph{very important to close the resource} (file, database
connection, remote connection / socket) at the
end. Otherwise, the resource may not be accessed by other programs (or
even by your same program in some situations). In a way, closing a
data source is like releasing memory that you no longer
used. Unfortunately, it is not possible to create something like a
garbage collector for data sources, so programmers have to close their
data sources manually.



Read/Write

% File.SEPARATOR
%
% Talk about finally
%
% and the new way of doing finally in Java 7

%%% Local Variables:
%%% mode: latex
%%% TeX-master: "d16"
%%% End:
