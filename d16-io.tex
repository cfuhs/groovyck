
\section{Input/Output (I/O)}
\label{sec:inputoutput-io}

Computer programs are just processors of data. They take some data as
input, they return some data as output. Up to now, all the data that
our little programs have processed was either provided by a human user
or already included in the program. In the real world, however, the
most usual sources of data are computers; either the computer where
the program is running (local files), a computer in the vicinity (a
database server), or a remote computer (remote resources through the
net). 

We are going to learn how to use the first of these three sources
of information: how to read from the local disk and how to write to
the local disk. 
%
Next week we will learn a bit about network programming, and next term
we will learn about database access. 

\subsection{The basics}
\label{sec:basics}

\subsubsection{File systems}
\label{sec:filesystems}

Computers have different levels of memory, but there is a clear
separation between primary and secondary memory. Primary memory
(usually referred to as simply ``memory'' or ``RAM'')
faster but requires electricity to run; if the computer is switched
off, the contents of the memory are lost. Secondary memory (usually
referred to as ``disk'') is slower,
but its contents are persistent, i.e.~they are still there when the
computer is switched off and on again. Secondary memory is usually
implemented in the form of a hard disk or so-called flash
memories. It is usually referred to as ``disk'', regardless of the
actual technology used for it. 

Data is stored in secondary memory in files\footnote{This name, as
  many others, comes from an ancient era where files, archives,
  directories, and folders were physical objects that contained
  documents or data.}  
The file system is a subsystem of the operating
systems that keeps all your files in place, and provides a way of
accessing them. Usually this is done by means of hierarchies: there is
one folder/directory at the top level of the hierarchy (called the
\emph{root}), which contains some files and subdirectories, each of
these subdirectories may contain files and/or subdirectories, etc. In
order to access a file \verb+taxForm2012.odt+ inside a folder
\verb+taxes+, inside a folder \verb+MyStuff+, you may access it with a
route like: 

\begin{verbatim}
    /home/john/MyStuff/taxes/taxForm2012.odt                     (unix)
    c:\Documents and Settings\john\MyStuff\taxes\taxForm2012.odt (windows)
    ./MyStuff/taxes/taxForm2012.odt
    MyStuff/taxes/taxForm2012.odt
\end{verbatim}

Different operating systems use different conventions for the root of
the tree of subdirectories and the separation of levels in the 
hierarchy. Unix systems (e.g.~Linux, MacOS/X) use a single root
(``/'') for the whole filesystem, and separate directories using a
slash (``/''). Windows use different roots, one per physical device or
partition (C:, D:, etc) and separate directories with a 
backslash~(``\verb+\+''). 

File routes that start at the root are called \emph{absolute} and
those that do not are called \emph{relative} (because they are
relative to the current directory). Absolute routes in
Unix always start with ``/'', but in Windows they can start in
different ways (e.g.~''\verb+C:\+'', ''\verb+D:\+'', ''\verb+\+'',
etc) because there are many roots. Relative routes may start with
a dot (meaning ``current directory'') but they usually start with the
name of a file or directory. 


\subsubsection{The I/O Process}
\label{sec:process}

The process of reading from / writing to an external source always
follows the same sequence of steps:

\begin{enumerate}
\item Open the resource (e.g.~file). If it cannot be opened, finish.
\item Read from and/or write to the resource.
\item Close the resource. 
\end{enumerate}

This process is the basis of all interaction with external data
sources, including files, databases, and remote resources\footnote{As
  we will see, in the case of network resources this process is
  usually hidden from the programmer.}.
It is \emph{very important to close the resource} (file, database
connection, remote connection / socket) at the
end. Otherwise, the resource may not be used by other programs (or
even by your same program in some situations). In a way, closing a
data source is like releasing memory that you no longer
used. Unfortunately, it is not possible to create something like a
garbage collector for data sources, so programmers have to close their
data sources manually.

\subsection{File names in Java and the File class}
\label{sec:files-java}

File names are represented in Java by the class \verb+File+ in package
\verb+java.io+. It is important to note that Java uses the Unix
tradition of considering (almost) everything a file: source code,
letters, and 
spreadsheets are files, but so are directories too. Therefore,
an object of class \verb+File+ can represent the name of an actual
file or the name of a directory.

This class implements a lot of methods that are useful when
interacting with the local disk. Some of the most commonly used
include: 
\verb+createNewFile()+, 
\verb+delete()+, 
\verb+exists()+, 
\verb+isDirectory()+, 
\verb+isFile()+, 
\verb+length()+, 
\verb+list()+ (lists all files in a directory), 
and
\verb+mkdir()+ (creates a directory). There are many other useful
methods, as you can see on the JavaDoc of class \verb+File+\footnote{As you
  have done many times in the past weeks, you can find the JavaDoc of
  this class by searching for ``java file''. Usually the first link
  will be the documentation of the class. }, and most of them have
quite self-descriptive name. 

Using the \verb+File+ class to get a pointer to a file (or directory)
on disk is really easy: 

\begin{verbatim}
    String filename = "filename.txt"; 
    File file = new File(filename);
\end{verbatim}

The name of the file (\verb+filename+) can be any valid route,
e.g. \verb+/home/john/file.odt+. However, there are two things that a
Java developer should keep in mind to make sure that Java programs can
run on any computer: 

\begin{description}
\item[Slash as separator] You should use slash (``/'') as a separator,
  as it works in both Unix and windows systems. For extra security,
  you can use the static final field \verb+File.separator+ that always
  has the right value for the operating system the Java Virtual
  Machine is running on. Do not use names like \verb+.\myFile.txt+;
  use either \verb+./myFile.txt+ or ---to make sure it works in all
  computers---:

\begin{verbatim}
    ''.'' + File.separator + ''myFile.txt''
\end{verbatim}

\item[Relative filenames: ] Your filenames should always be
  \emph{relative}, rather than absolute. Absolute routes will not work
  among different operating systems. For example, \verb+myFile.txt+ and
  \verb+./myStuff/myFile.txt+ are fine, but \verb+c:\myStuff\file.txt+
  is not. 
\end{description}

\subsection{Reading from and writing to files}
\label{sec:reading-from-writing}

Once we have a pointer to a file (through its name) we can try opening
it as we said in Section~\ref{sec:process}. Note that the file may or
may not exist. The \verb+File+ object is only a reference to the
name. If we want to know whether the file actually exists, we can call the
\verb+exists()+ method. 

\verb+File+ objects cannot be directly opened in Java. Instead, an
input source of data (a \verb+Reader+) can be created with a file and
it is this source that is opened (and closed at the end) to
read. Conversely, an output sink of data (a \verb+Writer+) can be
created with a file, opened, written to, and closed. 

% TODO: talk about streams? Probably not. The small note at the end
% about binary files should do, I think. 

\subsubsection{Readers}
\label{sec:reader}

\verb+Reader+ is the abstract class from which all other reader
classes descend, and it defines methods to read characters, either one
by one or many in one go. 
There are three main reader classes, each of them with
a different purpose: 

\begin{description}
\item[FileReader: ] The class for reading from text files\footnote{For
  binary files you use a FileInputStream.} (including
  common formats like CSV and XML (and XML-based formats like
  OpenDocument). 
\item[BufferedReader: ] A general-purpose reader that reads characters
  from a character input stream, buffering them for efficiency. This
  means that you do not need to read character by character. It
  provides useful methods like \verb+readLine()+\footnote{Although the
  name is the same, and the functionality is quite similar, this
  method has nothing to with the readLine() method of class Console,
  the one that we use when we execute System.console().readLine().}.
\item[StringReader: ] A string-based reader that uses a string as the
  input source of data. Can be useful if the data is already stored in
  a String instead of being read from a file or network connection. 
\end{description}

\paragraph{Opening a data source}
\label{sec:open-data-source}

A \verb+Reader+ is automatically open at creation (with
\verb+new+). There is no need to explicitly open it with a method
call. 

The usual idiom to create an input source
combines a \verb+FileReader+ (to read from a
\verb+File+) with a \verb+BufferedReader+ (to have access to buffered
data input and the \verb+readLine()+ method). If a \verb+FileReader+ is
directly used (without buffering), the performance of the application
may suffer. 

\begin{verbatim}
    File file = new File("here/be/the/route/to/the/file");
    BufferedReader in = new BufferedReader(new FileReader(file)); 
\end{verbatim}

Creation of a \verb+FileReader+ may throw an
\verb+FileNotFoundException+, and this is a checked exception. This
means that the code above must be placed inside a \verb+try+/\verb+catch+
construct or the method must declare that it \verb+throws+ this
exception. 

\paragraph{Reading from the source}
\label{sec:reading-from-source}

Once the data source is opened, we can read from it line by line using
\verb+readLine()+. When we reach the end of the file,
\verb+readLine()+ returns \verb+null+. 

Usually text files are read line by line, and something is done 
for every line. This looks like a situation to use a \verb+while+
loop, as in this example:

\begin{verbatim}
    String line = in.readLine());
    while (line != null) {
        // do things here with the line...
        line = in.readLine();
    }
\end{verbatim}

This repeats the \verb+readLine()+ call, so we may think of using a
\verb+do...while()+ loop: 

\begin{verbatim}
    do {
        String line = in.readLine());
        // do things here with the line...
    } while (line != null) {
\end{verbatim}

\ldots but this is wrong. Sooner or later the end of the file will be
reached and a \verb+NullPointerException+ will be thrown. So we need
to use a \verb+while+ loop, but we can write it a bit neater (although
it seems confusing at first): 

\begin{verbatim}
    while ((String line = readLine()) != null) {
        // do things here with the line...
        line = in.readLine();
    }
\end{verbatim}

This is the usual idiom for reading lines from text files in
Java. Although mixing assignment and conditional clauses in the same
line is usually discouraged because it can be confusing, this idiom is
so common that everybody uses it without any confusion. 

\paragraph{Closing the source}
\label{sec:closing-source}

When we have finished reading from the file, we must close the
data source. This is easily done with the \verb+close()+ method. 

\begin{verbatim}
     in.close();
\end{verbatim}

Alert readers may have noticed that this poses a small problem. Look
at the following code: 

\begin{verbatim}
01    File file = new File("file.csv");
02    try {
03        BufferedReader in = new BufferedReader(new FileReader(file); 
04        while ((String line = in.readLine()) != null) {
05            // ... do things with the data here
06        }
07        in.close();
08    } catch (FileNotFoundException ex) {
09        System.out.println("File " + file + " does not exist.");
10    } catch (IOException ex) {
11        ex.printStackTrace();
12    }    
\end{verbatim}

If a \verb+FileNotFoundException+ is thrown on line 03 nothing bad
happens; it is dealt with at the appropriate catch block\footnote{If
  you look at the Java Doc, you will see that FileNotFoundException is
  a subclass of IOException, so this catch block must come before the
  other one.}. But what
happens if an \verb+IOException+ is thrown at line 04 or inside the
loop? The execution flow will jump to the catch clause on line 10,
skipping line 07 and the data source will never be closed! 

\begin{itemize}
\item We cannot move line 07 out of the \verb+try+ block because
  \verb+in+ is defined inside that scope.
\item We cannot move line 07 to the \verb+catch+ block because that means
  \verb+in+ will not be closed if no exceptions occur, which is even
  worse. 
\item Another possibility would be to declare \verb+in+ out of the
  \verb+try+ block and initialise it inside, then closing it
  outside. However, this risks a \verb+NullPointerException+ if it is
  never initialised (e.g.~because the file does not exist or is not
  readable) before \verb+in.close()+ is called. 
\item Finally, a fourth possibility is to duplicate the \verb+close()+
  call, which is clearly suboptimal.
\end{itemize}

We just need a way to make sure that a data source is always closed. 

This is what \verb+finally+ is for. Every \verb+try+ can have a
\verb+finally+ block that is always executed after the \verb+try+
block; if an exception is thrown and caught, the \verb+finally+
method will be executed after the code in the \verb+catch+ block is
executed. The resulting code would look like this: 

\begin{verbatim}
01    File file = new File("file.csv");
02    try {
03        BufferedReader in = new BufferedReader(new FileReader(file); 
04        while ((String line = in.readLine()) != null) {
05            // ... do things with the data here
06        }
08    } catch (FileNotFoundException ex) {
09        System.out.println("File " + file + " does not exist.");
10    } catch (IOException ex) {
11        ex.printStackTrace();
11    } finally { 
12        in.close()
13    }
\end{verbatim}


\subsubsection{Writers}
\label{sec:writers}

Class \verb+Writer+ is the dual opposite of \verb+Reader+, and the
class from where all writers descend. Writers are used to write on
files on disk. 

Most of the reader classes have an equivalent writer class. There is a
\verb+Writer+, a \verb+BufferedWriter+, a \verb+FileWriter+, and a
\verb+StringWriter+. They are used mostly in the same way, although
there is one important difference: there is no \verb+writeLine()+
method. Apart from this, they are used in a very similar way to how
readers have been used in the preceding section. 

There is an additional class that has no dual reading counterpart but
is very convenient to use: \verb+PrintWriter+. This class provides a
constructor to build an object from a \verb+File+, and it also provides 
convenient methods for writing into files, including: 
\verb+append(char)+, 
\verb+print(String)+, 
\verb+printf(...)+ (similar to C's), 
and 
\verb+println(String)+.
The code below shows a typical use of \verb+PrintWriter+: 

\begin{verbatim}
    File file = new File("file.csv");
    try {
        PrintWriter out = new PrintWriter(file); 
        out.write(...);
    } catch (FileNotFoundException ex) {
        // This happens if file does not exist and cannot be created, 
        // or if it exists, but is not writable
        System.out.println("Cannot write to file " + file + ".");
    } catch (IOException ex) {
        ex.printStackTrace();
    } finally { 
        out.close()
    }    
\end{verbatim}


\paragraph{Flushing}
\label{sec:flushing}

When you read from a file, your program will not continue until the
read operation finishes. This is not exactly the case with writing
operations. When you execute one of the \verb+write()+ suite of methods, this only
puts the information in some structure of the operating system. This
may or may not be put immediately on disk, depending in different
factors. 

This may produce strange effects, like data being lost because the
program crashed after the \verb+write()+ call but before the data was
actually committed to disk. If you want to ensure that data has been
written on disk, you can use the method \verb+flush()+. 

Closing a data sink using \verb+close()+ 
automatically flushes it before closing to prevent data loss. 


\subsection{Binary files}
\label{sec:binary-files}

Text files contain text. They contain only printable characters; they
can be seen as a big string on disk. You are encouraged to
use only text files: they are simpler, they are clearer, and they
allow you to examine them when there are bugs in your code that
corrupt data. In the
preceding notes we have seen how to interact with text files. 

Old applications sometimes use \emph{binary} files. Binary files
contain non-printable characters and cannot be read by
people. Compiled class files are binary files. Try
opening a \verb+.class+ file in your favourite text editor to get a
feel of what they look like (hint: gibberish). Old applications use
binary files usually because they use less space on disk, and disk (and
memory!) space was limited in old times. 

In order to work with binary files, the classes \verb+InputStream+ and
\verb+OutputStream+ are used, instead of \verb+Reader+ and
\verb+Writer+. If you ever need to work with binary files in your
future career as a programmer, you are prepared at this point to read
the documentation directly from the Java Doc. The main ideas are the
same: open resource, read and write, close resource. There is also a
class \verb+RandomAccessFile+ that allows you to read from and write
to arbitrary positions in a file, instead of using it as a stream. 

Unless you are dealing with old legacy code, you should always use
plain text in your programs. Memory and disk space are much 
cheaper\footnote{If you are really concerned about space, use text
  files compressed with ZIP ---this is what LibreOffice does---. Look
  at package \verb+java.util.zip+, where you will find many classes
  that take care of the compressing / decompressing process.} 
than sleepless nights trying to debug binary data with an
hexadecimal editor. 

% TODO: preferences

% TODO: try-with-resources

% TODO:  and the new way of doing finally in Java 7
% 
% Exercises: 
%
% - ls
% - mkdir 
% - cat
% - copy
% - tr
% - uniq (*)
% - sort (*)
% - Read temperatures and output the averages

%%% Local Variables:
%%% mode: latex
%%% TeX-master: "d16"
%%% End:
