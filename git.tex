\section{Source code version control}
\label{sec:source-code-version}

What is version control? It is something as simple (and as difficult
to make right) as keeping track of changes in some piece of work. You
are probably familar with some very rudimentary version of version
control. Have you ever listed the documents in a folder and seen
something similar to this?

\begin{itemize}
\item myDocument
\item myDocument-2
\item myDocument-3
\item myDocument-final
\item myDocument-final2
\item myDocument-final-final
\item myDocument-definitive
\item myDocument-defitive-USE-THIS-ONE
\end{itemize}

If so, you already understand the most important idea behind version
control: our work is never created in one go, it changes over time and
sometimes we want to make sure we can go back in time to a former
version of it\ldots just in case. 

Many modern programs have version control embedded into them,
e.g. word processors like Microsoft Word, OpenOffice Write, or Google
Docs (also known as Google Drive). Very often they just track the
changes made to the document, sometimes they allow the user to go back
and forth in time to review, accept, or discard changes. This is also
very common in wiki sites like the Wikipedia. 

% TODO: add a screenshot of the version control in wikipedia here. 

Version control systems were initially created to track changes in
\emph{source code}. They were created by programmers for
programmers. We have come a long way since those days, and now version
control is spread over many different applications, but they still aim
at two main features: 

\begin{description}
\item[Reversibility: ] the capacity of going back in time if you mess
  up and introduce bugs in your code; sorry, \emph{when} you introduce
  bugs in your code.
\item[Concurrency: ] the capacity of working together with other
  people on the same project, on the same file. 
\end{description}

These two capacities are basic for any modern programmer, and that is
why version control is (or should be) part of every programmer's daily
life. Modern programs are big and complex, and several programmers
work on them. Without appropriate version control, they cannot work at
the same time: they need to take turns, pass the baton\ldots this is
really unproductive, good programmers do not work like
that. Additionally, programmers ---even good programmers, as long as
they are human--- make mistakes all the time, sometimes serious
mistakes that break their programs completely, and they need to go
back in time to the point where everything was working fine and start
again (in large and complicated programs, this can be a long time
before).

In this chapter we will learn to make version control on your source
code files \emph{right}, not as shown above (myProgramOLD.groovy,
etc). And we will use a program for doing so called \emph{Git}.

\section{Git}
\label{sec:git}

There are many programs for performing source code ...