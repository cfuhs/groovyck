\section{Introduction}

%Simple input/output with integers, outputting words and ends of lines, assignment and initialisation}

%% Why should YOU want to write computer programs?
\athena{Computing increasingly permeates everyday life,
from the apps on our phones to the social networks that connect us.
Thus, there is a need for people to
% have the skill of not just using
%be able to
not just use
such systems made by others, but
%also
to be empowered to create their own.
%  and modify them, and ultimately
% to shape the future of our society.
}

To get a computer to do something, you have to tell it what you want it
to do.  You give it a series of instructions in the form of a \emph{program}.
You write a program in a \emph{programming language}.
Many programming languages have been devised \athena{over the decades};
well-known ones include
Fortran, BASIC, Cobol, Algol, Lisp, Ada, C++, Python, Ruby, Scala, and Java.

% The language we will use in this document is named \emph{Groovy}, and
% it is \emph{very similar to the Java programming language but with a
% simplified syntax} that better fits the purposes of this introduction. 

The language we will use in this document is named \emph{Python}.
More precisely, we are using version 3 of the Python language.
\athena{Python hits a sweet spot: it makes writing simple programs easy
and is robust enough for a wide uptake in a variety of industries.}

Programming languages vary a lot, but
an instruction in a typical programming language might contain
some English words (such as \verb!while! or \verb!return!), perhaps a mathematical
expression (such as \verb!x + 2!) and some punctuation marks used in a special way.
The following three lines are in BASIC, Cobol and C++ respectively:
\begin{Verbatim}
      FOR I% = 1 TO 100 STEP 5 DO

      PERFORM RECORD-TRANSACTION UNTIL ALL-DONE = "T"

      if (x > 100) cout << "Big"; else cout << "Small";
\end{Verbatim}

Programs can vary in length from a few lines to thousands of lines.
Here is a complete, short program written in Python:

\VerbatimInput[frame=single,label=Example]{src/Example1.py}

\pagebreak

When you write a program, you have to be very careful to keep to the
syntax of the programming language, i.e., to the grammar rules of the language.
For example, the above Python program would
behave incorrectly if in the 12$^{th}$ line we wrote
%\pagebreak

\begin{Verbatim}
    if maxi = 0:
\end{Verbatim}

instead of 

\begin{Verbatim}
    if maxi == 0:
\end{Verbatim}

or if in the 9$^{th}$ line we omitted the colon in

\begin{Verbatim}
        if len(word) > maxi:
\end{Verbatim}

The computer would refuse to accept the program if we added dots or colons
in strange places
or if any of the parentheses (or colons) were missing, or if we wrote
\verb!length! instead of \verb!len! or even \verb!Len! instead of \verb!len!.

% Moved to earlier:
% 
%The language we will use in this document is named \emph{Groovy}, and
%it is \emph{very similar to the Java programming language but with a
%simplified syntax} that better fits the purposes of this introduction. 

\subsection{Input and output}

\athena{Almost all programs need to communicate with the outside world to
be useful. In particular, they need to read input and to remember it. They
also need to respond with an answer.}

To get the computer to take some input \athena{as text} and to store it
in its memory, we write in Python:

\begin{Verbatim}
      word = input()
\end{Verbatim}

\verb!input()!
% (pronounced \emph{system dot console dot read line})
is a phrase which has a special meaning to
the computer. The combination of these instructions
means
% \footnote{More precisely, it means ``Take the current system
%   --- this computer, take its console --- its keyboard and screen,
% and read a line of character input from it as entered by the user'' (from the keyboard,
% not the screen because the user cannot type a line onto the screen... or
% could not before they invented touchscreens!).}
``Take some input which is a sequence of characters and put it into the computer's memory.''

\verb!word! by contrast, is a word that I (the programmer) have chosen.
I could have used \verb!characters! or \verb!thingy! or \verb!breakfast!
or just \verb!s! or almost any word I wanted. (There are some
restrictions which I will deal with in the next section.)  
% You may occasionally see the semi-colon used in a program and that signals the
% end of this instruction.

Computers can take in all sorts of things as input --- numbers,
letters, words, records and so on --- but, to begin with, we will
write programs that generally handle
% strings
\athena{text as sequences (or \emph{strings})} of characters (like
"I", "hi", "My mother has 3 cats", or "This is AWESOME!").
We'll also assume that the computer is taking its input from the
keyboard, i.e., when the program is executed, you key in one or more
words at the keyboard and these are the characters that the computer 
puts into its memory.

You can imagine the memory of the computer as consisting of lots of
little boxes.  Programmers can reserve some of these boxes for
use by their programs and they refer to these boxes by giving
them names.  

\begin{Verbatim}
      word = input()
\end{Verbatim}

means ``Take a string of characters input using the keyboard and
terminated when the user presses RETURN, and
then put this string into the box called \verb!word!.''  When the program runs and
this instruction gets executed, the computer will take the words which
you type at the keyboard (whatever you like) and will put them into
the box which has the name \verb!word!.

Each of these boxes in the computer's memory can hold only one string at a time.  If a box is
holding, say, "hello", and you put "Good bye!"\ into it, the
"Good bye!"\ replaces the "hello".  In computer parlance these
boxes are called \emph{variables} because the content inside the box
can vary; you can put an "I" in it to start with and later change it
to a "you" and change it again to "That was a hurricane" and so on
as often as you want.
Your program can have as many variables as you want.


% In Groovy, you have to tell the computer that you want to use a variable
% with a certain name before you use it.  You can't just pitch in using \verb!word!
% without telling the computer what \verb!word! is.
%   Most of the time you also have to
% tell the computer what {\em type} of variable this is, i.e., what sort of thing you are going
% to put into it. In this case we are going to put strings of characters into \verb!word!. 
% Strings of characters, or simply \emph{strings} as they are called in programming, are
% known in Groovy as \verb!String!, with capital S.  To tell the computer that we want to use a
% variable of type \verb!String! called \verb!str!, we write in our program:
%
% \begin{Verbatim}
%       String str
% \end{Verbatim}
%
% If we wanted to declare more than one variable, we could use two lines:
%
% \begin{Verbatim}
%       String myName
%       String yourName
% \end{Verbatim}
%
% or we could declare them both on one line, separated by a comma:
% \begin{Verbatim}
%       String myName, yourName
% \end{Verbatim}
%
In Python (and many other programming languages), you have to make sure
that you have put something into a variable before you read from it,
e.g., to print its contents. If you try to take something out of a box
into which you have not yet put anything, it is not clear what that
``something'' should be. This is why Python will complain if you write
a program that reads from a variable into which nothing has yet been put.


%   In Groovy you don't
% have to declare all your variables at the start of the program, as is the
% case in some other languages.  You can declare them in the middle of a
% program, but you mustn't try to use a variable before you've declared it.

If you want the computer to display (on the screen) the contents of one of
its boxes, you use \athena{\texttt{print(thingy)}} where instead of \athena{\texttt{thingy}}
you write the name of the box.  For example, we can print the contents of
\verb!word! by writing:

\begin{Verbatim}
      print(word)
\end{Verbatim}

If the contents of the box called \verb!word! happened to be "Karl", then when the
computer came to execute the instruction \verb!print(word)! the word "Karl" would appear
on the screen.  



% Arithmetic expressions such as \verb!num + 5! can also appear in a \verb!print!
% line.  For example, if \verb!num! had the value 7, then \verb!print num + 5!
% would output 12.
So we can now write a program in Python (not a very exciting program,
but it's a start):

\VerbatimInput[frame=single,label=Example]{src/Example2.py}

This program takes some text as input from the keyboard and displays it
back on the screen.

In many programming languages it is customary to lay out programs like this
or in our first example with each instruction on a line of its own.
In these languages, the indentation, if there is any, is just for the benefit
of human readers, not for the computer which ignores any indentations.

Python goes a step further and uses indentation as a part of the language
to structure the programs. We will see more about this a bit later.

\pagebreak

\subsection{Running a program}

You can learn the rudiments of Python from these notes just by doing the
exercises with pencil and paper.
\athena{However, it is a good idea to test your programs using a computer
after you have written them so you can validate that they do what you
intended. If}
%   It is not essential to run your programs
% on a computer.
%However, if
you have a computer and are wondering how
you run
% the
\athena{your}
programs, you will need to know the following, and, even if you don't have
a computer, it will help if you have some idea of how it's done.

First of all you type your program into the computer using a text
editor.
Then, to run your program, you use a piece of software called
an \emph{interpreter}.
The interpreter first checks whether your program is acceptable according
to the syntax of Python and then runs it.
If it isn't acceptable according to the syntax of Python,
the interpreter issues one or more error messages telling you
what it objects to in your program and where the problem lies.
You try to see what the problem is, correct it and try again.  You keep
doing this until the interpreter runs your program without further
objections.
%  Then, 
% before you can run the program, you have to \emph{compile} it.  This means that
% you pass it through a piece
% of software called a \emph{compiler}.  The compiler checks whether your program
% is acceptable according to the syntax of Groovy.  If it isn't, the compiler
% issues one or more error messages telling you
% what it objects to in your program and where the problem lies.
% You try to see what the problem is, correct it and try again.  You keep
% doing this until the program compiles successfully.  You now have an
% {\em executable} version of your program, i.e., your program has been translated
% into the internal machine instructions of the computer and the computer
% can run your program.

To make the computer run your program using the interpreter, you need to
issue a command. You can issue commands\footnote{In
    a modern operating system, you can click on an icon to execute a
    program. However, this only makes sense for graphical applications
    and not for the simple programs that you will write at first.}
from the command prompt in Windows (you can find the command prompt under 
\emph{Start $\rightarrow$
    Accessories}), or from the terminal in Linux and Mac OS/X.


If you are lucky, your program does what you expect it 
to do first time.  Often it doesn't.  You look at what your program is
doing, look again at your program and try to see why it is not doing what
you intended.  You correct the program and run it again.
You might have to do this many times before the program behaves in the way
you wanted. \athena{This is normal and nothing to worry about at this stage.}

As I said earlier, you can study this introduction without running your
programs on a computer.  However, it's possible that you have a PC with
a Python interpreter and will try to run some of the programs given in these
notes. 
If you have a PC but you don't have a
Python interpreter, I attach a few notes telling
you how you can obtain one (see Section~\ref{sec:obta-inst-runn}).

\athena{If you do not wish to install Python on your machine, you can use
the online Python editor and interpreter at
\begin{center}
\url{https://repl.it/languages/python3}
\end{center}
Have a look at the intro video at
\begin{center}
\url{https://replit.github.io/media/quick-start/simple-repl.mp4}
\end{center}
and the quick-start guide
\begin{center}
\url{https://repl.it/site/docs/misc/quick-start}
\end{center}
}

%\pagebreak

\subsection{Outputting words and ends of lines}

% \athena{So far, we've been able to talk to the computer.
% For the computer to talk back to us, we need
% }

Let's suppose that you manage to %compile
get your program to run.
%and that you then run it.
Your running of the above program
would produce something like this on the screen if you typed in
the word Tom followed by RETURN: 

\begin{Verbatim}
Tom
Tom
\end{Verbatim}

The first line is the result of you keying in the word Tom.  The system ``echoes''
the keystrokes to the screen, in the usual way.  When you hit RETURN, the
computer executes the \verb!word = input()! instruction, 
i.e., it reads the word or words that have been input.  Then it
executes the \verb!print! instruction and the word or words that were
input appear on the screen again.

We can also get the computer to display additional words by putting
them in quotes\footnote{In Python, we can choose between \texttt{"} and \texttt{'}, but we cannot use both together.}
in the parentheses after the \verb!print!, for example:

\begin{Verbatim}
      print("Hello")
\end{Verbatim}

We can use this to
% improve the above program:
\athena{make the above program more user-friendly:}

\VerbatimInput[frame=single,label=Example]{src/Example3.py}

An execution, or ``run'', of this program might appear on the screen thus:
\begin{Verbatim}
Please key in a word:
Tom
The word was:
Tom
\end{Verbatim}

Here each time the computer executed the \verb!print! instruction,
it also went into a new line.
If we use \verb!print(word, end = "")! instead of \verb!print(word)!
then no extra line is printed after the value.\footnote{The reason is that
 with \Verb!end = ""! we are telling Python that after printing the contents
 of \Verb!word!, we want to print nothing at the end instead of going to a
 new line.}
So we can improve \athena{the formatting of the output of our program
on the screen:}
% our program a bit further:

\VerbatimInput[frame=single,label=Example]{src/Example3a.py}

Now a ``run'' of this program might appear on the screen thus:
\begin{Verbatim}
Please key in a word: Tom
The word was: Tom
\end{Verbatim}

Note the spaces in lines 1 and 3 of the program after \verb!word:! and \verb!was:!.
This is so that what appears on the screen is \verb!word: Tom! and
\verb!was: Tom! rather than \verb!word:Tom! and \verb!was:Tom!.

It's possible to output more than one item with a single \verb!print! instruction.
For example, we could combine the last two lines of the above program into one:

\begin{Verbatim}
print("The word was " + word)
\end{Verbatim}

and the output would be exactly the same. The symbol ``\texttt{+}'' does not
represent addition in this context, it represents concatenation, i.e.,
writing one string after the other. We will look at addition of numbers in the next section.


Let's suppose that we now added three lines to the end of our program, thus:

\VerbatimInput[frame=single,label=Example]{src/Example4.py}

% After running this program, the screen would look something like this:
% \begin{Verbatim}
% Please key in a word: Tom
% The word was TomNow please key in another: Jack
% And this one was Jack
% \end{Verbatim}

% which is probably not what we wanted.  If we want a new line after the
% first word is printed, we need to use the \verb!println! instruction we mentioned earlier:

% \VerbatimInput[frame=single,label=Example]{src/Example5.groovy}

% Now we would get:

After running this program, the screen would look something like this:
\begin{Verbatim}
Please key in a word: Tom
The word was Tom
Now please key in another: Jane
And this one was Jane
\end{Verbatim}

\subsection*{Exercise A}

\athena{
For those of you who prefer a ``learning by doing'' approach,
the University of Waterloo provides an interactive
tutorial to programming in Python 3 with a similar structure to
this booklet:
\begin{center}
\url{https://cscircles.cemc.uwaterloo.ca/}
\end{center}
This tutorial gives you further exercises and checks
whether the output obtained from your solutions is correct.
We will occasionally refer to exercises in this on-line
tutorial.
You are welcome to create an account if you wish to record your progress.
However, as this booklet is designed for self-study, we will not
provide any assistance for this system.\\
\indent
As your first exercise, read through Section ``0: Hello'' at the
above link and attempt the exercises in your browser as you
go along. The web page will provide you with direct feedback.
}

\subsection*{Exercise B}

Now pause and see if you can write:
\begin{enumerate}
\item
a Python instruction which would output a blank line.
\item
an instruction which would output

\begin{Verbatim}
Hickory, Dickory, Dock
\end{Verbatim}

\item
a program which reads in two words, one after the other, and then displays them
in reverse order. For example, if the input was 

\begin{Verbatim}
First
Second
\end{Verbatim}

the output should be

\begin{Verbatim}
Second
First
\end{Verbatim}

\end{enumerate}

%To check your answers, click on  \verb!Answers to the exercises!. (TO BE COMPLETED)

\subsection{Assignment and initialisation}

There is another way to get a string into a box apart from using
\verb!input()!.  We can write, for instance:

\begin{Verbatim}
    word = "Some text"
\end{Verbatim}

This has the effect of putting the string "Some text" into the
\verb!word! box.  Whatever content was in \verb!word! before is obliterated; the
new text replaces the old one. 

In programming, this is called \emph{assignment}.  We say that the
value "Some text" is
assigned to the variable \verb!word!, or that the variable \verb!word! takes the
value "Some text".  The ``\texttt{=}'' symbol is the assignment operator in Python.  We are not
testing whether \verb!word! has the value "Some text" or not, nor are we stating that
\verb!word! has the value "Some text"; we are \emph{giving} the value
"Some text" to \verb!word!.

% Unneeded. Probably should delete.
%
% If we want, we can have arithmetic expressions on the right-hand side of
% the "=", for example:
% \begin{Verbatim}
% num = count + 10
% \end{Verbatim}
% This instruction means, "Take whatever number is in \verb!count!,
% add 10 to it and put the result into \verb!num!."

An assignment instruction such as this:

\begin{Verbatim}
    word = word + " and some more"
\end{Verbatim}

looks a little strange at first but makes perfectly good sense.
Let's suppose the current value of \verb!word! (the contents of the
box) is "Some text".
The instruction says, ``Take the value of \verb!word!
("Some text"), add " and some more"  to it (obtaining "Some text and some
more") 
and put the result into \verb!word!''.
So the effect is to put "Some text and some more" into \verb!word! in
place of the earlier "Some text".


% We also say that the first time we put something into a variable \verb!foo!,
% we \emph{define} \verb!foo!.


%%% Python does not to uninitialised variables :)
The \verb!=! operator is also used to ``initialise'' variables.
In Python, a variable is \emph{defined} by the first assignment to
that variable, the first time you put something into it. This first
assignment is often also called \emph{initialisation}.
You can read from a variable only after it has been initialised.

% The reason for allowing to read from a variable only after something
% is put into it is as follows.
% When the computer allocates a portion of memory to store one of your
% variables, it does not clear it for you; the variable holds whatever
% value this portion of memory happened to have the last time it was
% used.
% %  Its value is said to be \emph{undefined.}
% To prevent strange values coming from such an ``uninitialised''
% portion of memory, Python will complain if you try to read from a
% variable into which you have not put anything yet.

% Earlier I said that in Python you have to make sure that you have put
% something into a variable before you read from it, e.g., to print its
% contents. 

% Using undefined values is a common cause of program
% bugs.  
% % Suppose a program uses the variable \verb!str! without giving it an
% % initial value and suppose that, on the computer the programmer is using,
% % the initial value in \verb!num! happens to be zero and that, by happy chance,
% % zero is just what the programmer wants it to be.
% % The program seems to work fine.  Then the program is compiled and run on
% % a different computer.  On this second computer, the initial value of \verb!num!
% % does not happen to be zero.  The program, which has worked OK on the first
% % computer, does not work on the second one.
% %
% In Python, you have to make sure that you have put something into a
% variable before you read from it, e.g., to print its contents.

% To prevent yourself from using undefined values, you can give a variable
% an initial value when you declare it.  For example, if you wanted \verb!str! to begin
% with empty, you should declare it thus:

% \begin{Verbatim}
%       String str = ""
% \end{Verbatim}

% This is very like assignment since we are giving a value to \verb!str!. But
% this is a special case where \verb!str! did not have any defined value before,
% so it is known as \emph{initialisation.}

Finally a word about terminology.  I have used the word ``instruction''
to refer to lines such as \verb!print(word)! and \verb!word = "Some text"!.
It seems a natural word to use since we are giving the computer instructions.
But the correct word is actually ``statement''.  \verb!print(word)! is
an output statement, and \verb!word = "Hello"! is an assignment statement.
% The lines in which we tell the computer about the variables we intend to
% use, such as \verb!String str! or \verb!String str = ""! are called
% variable \emph{definitions}.  They are also referred to as 
% variable \emph{declarations}.  When you learn more about Groovy you will
% find that you can have declarations which are not definitions, but the
% ones in these introductory notes are both definitions and declarations.


\subsection*{Exercise C}

Now see if you can write a program in Python that takes
two words from the keyboard and outputs one after the other on the same line. 
E.g., if you keyed in ``Humpty'' and ``Dumpty'' 
it would reply with ``Humpty Dumpty'' (note the space in between).
A run of the program should look like this:

\begin{Verbatim}
    Please key in a word: Humpty
    And now key in another: Dumpty
    You have typed: Humpty Dumpty
\end{Verbatim}

%To check your answers, click on \verb!Answers to the exercises!. (TO BE COMPLETED)

 	

%%% Local Variables:
%%% TeX-master: "primer"
%%% End:

