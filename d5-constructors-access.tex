
\section{More on clases}
\label{sec:more-clases}

There are two important aspects of classes that we have to learn
about. I am talking about constructor methods and levels of access. 

\subsection{Constructor methods}
\label{sec:constructor-methods}

When classes have been used in the preceding sections, they were used
in two steps: first the memory was reserved for them using \verb+new+
and then their fields were initialised. 

\begin{verbatim}
    Point corner = new Point();
    corner.x = 4;
    corner.y = 0;
\end{verbatim}

This is not too cumbersome unless you have a class that has much more
than two fields that need initialising. For example, if you wanted to
create a \verb+Person+ with first name, family name, gender, age, job,
nationality, etc, you would need a lot of code every time you created
a new \verb+Person+. 

\begin{verbatim}
    Person john = new Person();
    john.firstName = "John";
    john.familyName = "Smith";
    john.gender = Gender.MAN; // This is an enumerated type
    john.age = 30;
    // the rest of the parameters would come here...
    // ...
    Person mary = new Person();
    mary.firstName = "Mary";
    mary.familyName = "Jordan";
    mary.gender = Gender.WOMAN;
    mary.age = 33;
    // the rest of the parameters would come here...
    // ...
\end{verbatim}

We have written a lot of code and we have just created two
objects. This is really boring. On top of that, it looks like we are
repeating code over and over again by having to initialise all fields
manually. There is a better way of doing this. 

Every class (or complex type) can have one or more \emph{constructor
  methods}. A constructor method is a special type of method that is
used to initialise an object of a class when it is first created, and
it is executed every time a new method is created using \verb+new+. In
other words, the constructor method is a way of telling Java to use
\verb+new+ to allocate the memory \emph{and} initialise the object
at the same time. Simpler. Clearer. 

A constructor method looks like a method without a return type (not
even \verb+void+). Have a look at this example:

\begin{verbatim}
    class Point {
        double x;
        double y;
        
        Point(double x, double y) {
            this.x = x;
            this.y = y;
        }
        
        double moveTo(Point remote) {
            this.x = remote.x;
            this.y = remote.y;
        }
        
        // more methods here...
    }
    Point point = new Point(1,1);
    println "The point is now at " + point.x + "," + point.y
    Point remotePoint = new Point(10,20);
    point.moveTo(remotePoint);
    println "The point is now at " + point.x + "," + point.y
\end{verbatim}

No need to initialise the points after creating them. The constructor
method does it for both of them. 
If a class has more than one constructor method, Java chooses the
right one according to the positional parameters. For example, we
could create a class Rectangle that has one or two parameters, and
then create different instances (i.e.~objects) of it: 

\begin{verbatim}
    class Rectangle {
        int length;
        int width;
        Rectangle(int length, int width) {
            this.length = length;
            this.width  = width;
        }
        // This method creates a square, all sides equal
        Rectangle(int length) {
            this.length = length;
            this.width  = length;
        }
    }
    Rectangle dominoRectangle = new Rectangle(1,2);
    Rectangle pitagoreanRectangle = new Rectangle(3,4);
    Rectangle goldenRectangle = new Rectangle(1618,1000);
    Rectangle square = new Rectangle(5); 
\end{verbatim}

Every class has 
\textbf{at least one} constructor method. 
If it is not explicit ---as it has been
the case with all classes in the previous sections---, Java 
implicitly adds an empty
constructor: a constructor method with no parameters that does not
initialise any field\footnote{Strictly speaking, the constructor is
  not empty: it calls the constructor of the parent class. We will see
  this when we learn about inheritance.}. 

\begin{verbatim}
    // If there is no constructor for Rectangle, Java 
    // will add an "implicit empty constructor" like this
    public Rectangle() {
    }
\end{verbatim}


\subsection{Levels of access (public, private) and information hiding}
\label{sec:levels-access-inform}

A real programming project can have thousands of classes, and millions
of live objects in memory at any given time (remember that we create a
new object every time we use \verb+new+). Keeping track of all the
interactions of these objects becomes very complex very quickly. That
is why programmers make an effort to keep 
the complexity as low as possible. One
strategy to achieve that objective is 
to hide as much information as possible
and leave visible only what is strictly necessary. 

Imagine that you want to find out the total population of the European
Union and you request data from every country. Governments of each
country can give you either their population figure or a listing with
the names of every citizen; it is clear that the former option makes
your life much easier. This is the principle of \emph{information
  hiding}. 

Information hiding is achieved in Java by means of two\footnote{There
  is a third keyword but it is 
less important and we will come to it when we talk about class
hierarchies.} new keywords:\verb+public+ and \verb+private+. 
The keyword \verb+private+ specifies that the field (i.e.~variable) or
method that comes after it will be visible only inside its own
class. This should be your default option as a programmer, especially
for fields. The keyword \verb+public+ specifies that the field or
methods that comes after it will be visible from outside the class. 

Classes can also be \verb+public+ or \verb+private+. In the same way
that public fields and methods are visible out of their own class, a
public class is visible out of its file, and viceversa. 


\subsubsection*{Rules of thumb for access levels}
\label{sec:rules-thumb-access}

After many decades, programmers have developed some rules of thumb
to know what should be public and what should be private. The
following rules cover 90\% of all cases, but do not expect them to be
valid in absolutely every situation. 

\begin{itemize}
\item If a class has methods, its fields must be private. This is the
  most common case. 
\item If-and-only-if a class does not have any methods (not counting
  constructors), its fields must be public.
\item Methods of a class must be private unless their purpose is to be
  called from outside the class, in which case they must be public.
\item Constructor methods must be public. 
\end{itemize}

How would class \verb+Point+ be if we were doing things right and
using \verb+public+ and \verb+private+ to specify the visibility of
everything? Like this: 

\begin{verbatim}
    public class Point {
        private double x;
        private double y;
        public Point(double x, double y) {
            this.x = x;
            this.y = y;
        }
        public int getX() {
            return x;
        }
        public int getY() {
            return y;
        }
        public double moveTo(Point remote) {
            this.x = remote.x;
            this.y = remote.y;
        }
        // more methods here...
    }
    Point point = new Point(1,1);
    println "The point is now at " + point.getX() + "," + point.getY();
    Point remotePoint = new Point(10,20);
    point.moveTo(remotePoint);
    println "The point is now at " + point.getX() + "," + point.getY();
\end{verbatim}

This is a bit more complex at first sight, but now we can understand
all its components. What can we see?

\begin{itemize}
\item The class is \verb+public+, because we probably want it to be
  used from other files. This is the usual case.
\item The constructor is \verb+public+. If it was not, it would be
  difficult to create new objects of type \verb+Point+ using
  \verb+new+. Actually, it would be impossible.
\item The methods are all \verb+public+ (at least the methods we can
  see in the example), because all of them are to be used by other
  code that uses objects of type \verb+Point+.
\item The fields \verb+x+ and \verb+y+ are \verb+private+. Fields
  should be private and be accessed through methods in 99\% of all
  programs.
\item As fields are private, two \emph{getters} or \emph{accessor
    methods} have been added. If we want to be able to modify the
  dimensions of objects of type \verb+Rectangle+ after they are
  created, we will need to add \emph{setters} too. 
\end{itemize}

%%% Local Variables:
%%% mode: latex
%%% TeX-master: "main"
%%% End:
