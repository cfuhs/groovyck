\section{Obtaining, installing and running Groovy on a PC}
\label{sec:obta-inst-runn}
 
You can download a free copy of the Groovy compiler and runtime from the web:

\begin{verbatim}
    http://groovy.codehaus.org/Download
\end{verbatim}

Groovy can run in any computer with Java installed, including the main
operating systems like Windows, Linux, and Mac. If you do not have
Java installed in your computer, you can install it from the web:

\begin{verbatim}
    http://www.java.com/en/download/index.jsp
\end{verbatim}

Java will be used intensively throughout the MSc in Computer Science
at Birkbeck, so it is a good idea to install it if you do not have it
yet (which is usually the case).

If you do not know whether you have Java or Groovy installed in your
system, you can open a command prompt\footnote{You can find it in Windows in
  ``Accesories''} and type \verb+java -version+ or \verb+groovy -version+. 

If they are installed, the result should be something similar to this:

\begin{verbatim}
    > java -version
    java version "1.6.0_23"
    OpenJDK Runtime Environment (IcedTea6 1.11pre) (6b23~pre11-0ubuntu1.11.10.1)
    OpenJDK Server VM (build 20.0-b11, mixed mode)

    > groovy -version
    Groovy Version: 1.7.10 JVM: 1.6.0_23
\end{verbatim}



\subsection*{Compiling and running programs (scripts)}

The Groovy compiler is run from the command line.

You first type your program in a text editor. Notepad (in Windows) or
gedit (in Linux) are good options, but any simple editor will
do. Note: If you use a word-processor (like Microsoft Word or
LibreOffice Writer), make sure you save the file as text.

Give the filename a \texttt{.groovy} extension.

Open the command prompt. 
Go to the folder where you have saved your Groovy
program using the command "cd" (change directory). (You don not have to
do this but it is easier if you do, otherwise you have to type the full
path for your source code file.)

Suppose your program is in a file called 
\texttt{Myprog.groovy}. You can compile and run your program in one step typing 

\begin{Verbatim}
    groovy  Myprog.groovy
\end{Verbatim}

If the compilation was successful, you will see the result of your
program on the screen.
%
If there were errors you would get error messages and you would need
to go back to the text editor, correct and save the program again,
then recompile.



%%% Local Variables:
%%% TeX-master: "primer"
%%% End:
