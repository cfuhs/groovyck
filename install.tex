 
You can download a free copy of Python 
from the web:

\begin{verbatim}
https://www.python.org/downloads/
\end{verbatim}

It is important that you download version 3.\ldots, which is what we
use in this introduction.
Python is available for many
operating systems, including Windows, Linux, and Mac.%  If you do not have
% Java installed in your computer, you can install it from the web:

% \begin{verbatim}
%     http://www.java.com/en/download/index.jsp
% \end{verbatim}

Python will be used intensively in several of the MSc courses 
at Birkbeck, so it is a good idea to install it if you do not have it
yet.
If you do not know whether you have Python installed in your
system, you can open a command prompt\footnote{You can find it in Windows in
  ``Accessories''.} and type
% \footnote{In some systems, you can type
%   ``-v'' instead of ``-version''.}
\verb+python3 --version+.
If Python is installed, the result should be something similar to this:

\begin{verbatim}
    > python3 --version
    Python 3.4.3
\end{verbatim}



\subsection*{Running Python programs}

Python is run from the command line.

You first type your program in a text editor. Notepad (in Windows) or
gedit (in Linux) are good options, but any simple editor will
do. Note: If you use a word-processor (like Microsoft Word or
LibreOffice Writer), make sure you save the file as plain text.

Give the filename a \texttt{.py} extension.

Open the command prompt. 
Go to the folder where you have saved your Python
program using the command ``cd'' (change directory). (You do not have to
do this but it is easier if you do, otherwise you would have to type the full
path for your source code file to run it.)

Suppose your program is in a file called
\texttt{Myprog.py} in the current folder.
You can run your program by typing

\begin{Verbatim}
    python3  Myprog.py
\end{Verbatim}

If the %compilation is successful,
Python interpreter accepts your program,
you will see the result of your
program on the screen.
%
If there are errors, you will get error messages and you need
to go back to the text editor, correct the program and save it again,
then try to run your program again.



\subsection*{If you cannot install Python\ldots}
\label{sec:if-you-cannot}

If you have problems installing Python, you may find this webpage useful:

\begin{verbatim}
   https://repl.it/languages/python3
\end{verbatim}

The page provides you with an editor on the left and a console on the
right. It will allow you to test your programs online. It may be quite
slow compared to using a local machine, but it may help if you find
problems installing Python. 

Additionally, if you have any problem installing or running Python,
feel free to write to Dr.\ Carsten Fuhs
(carsten@dcs.bbk.ac.uk). We will come back to you as soon as possible.

%%% Local Variables:
%%% TeX-master: "primer"
%%% End:

