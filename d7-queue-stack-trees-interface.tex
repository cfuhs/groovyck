
\section{Interfaces}
\label{sec:interfaces}

Classes in Java have member fields and member methods. The fields are
variables stored inside objects for that class, and contain
information about the object. For example, they can represent the age
of a person or the number of patients in a hospital. The set of all
member fields of an object is usually called the \emph{state} of the
object: they say how the object \emph{is} at the current moment. 

On the other hand, methods (at least public ones) say what the object
can do. For example, a person can say something or a hospital can take
a new patient. The set of (public) methods of a class is usually
called the \emph{behaviour} of the class. 

The behaviour of a class is more important for a programmer than its
state. The state is an implementation detail that can change without
affecting its behaviour, and is usually not important for the task at
hand: this is why fields should be \verb+private+ in most
situations. If I want an object of type \verb+Person+ to say
something, I do not really mind if the person internally uses some
object of type \verb+VocalCord+, or some \verb+Vocoder+, or something
else: I just want the method to do what it is supposed to do. I do not
mind either if a future version of \verb+Person.say(String)+ is
implemented internally in a different way or with different
fields. Actually, it is quite common that the implementation changes
over time (to make it faster, or to use less memory, or to fix bugs)
so I should not worry about it; I should assume it will happen sooner
or later. 

This is why the behaviour of a class is more important than its
state. There are two consequences of this. First, the state should
always be private. Second, the behaviour must be easy to know and
understand without the need to read the whole code of my class. This
where interfaces come in. 

An interface in Java is just a way of showing and explaining the
behaviour of your class. An interface does not contain any information
at all about the implementation or the state of your class. It only
describes some (sometimes all) of the public methods of your
class. Let's see an example: 

\VerbatimInput[frame=single,label=Example]{src/Person.java}

As you can see, the definition of an interface is very similar to the
definition of a class, it is even defined in a java file. 
The difference (and it is a big one!) is that
there are no implementation details at all in the definition of the
interface. It only declares the methods: their names, and their
parameters. Every other detail about the class is left for the class
to be defined. 

Note that there is no need to say that the methods are public because
methods defined on an interface are public by definition. 



%              interfaces
%              queues and stacks, 
%              trees, binary and non-binary trees, 
%              basics of memory allocation, gargage collection, 
