\documentclass{article}
\usepackage[margin=2cm]{geometry}
\usepackage[dvips]{graphicx}
\begin{document}

\section*{Learning goals}
\label{sec:learning-goals}

Before the next day, you should have achieved the following learning
goals: 

\begin{itemize}
\item Understand what exceptions are and how they are used.
\item Create your own exceptions, both checked and runtime.
\item Catch exceptions by means of try/catch. 
\end{itemize}

\section{Code flow}
\label{sec:code-flow}

Look at the following code and write the code flow (use the line
numbers to indicate which lines are executed in which order) in the
following situations: 

\begin{itemize}
\item \verb+userInput+ is 0.
\item \verb+userInput+ is 2.
\item \verb+userInput+ is 4.
\item \verb+userInput+ is 6.
\end{itemize}

\begin{verbatim}
01   public void launch(int userInput) {
02       List<Integer> intList = new ArrayList<Integer>();
03       intList.add(1);
04       intList.add(2);
05       intList.add(3);
06       intList.add(4);
07       intList.add(5);
08       intList.add(6);
09       try {
10          intList.remove(0);
11          System.out.println(intList.get(userInput));
12          intList.remove(0);
13          System.out.println(intList.get(userInput));
14          intList.remove(0);
15          System.out.println(intList.get(userInput));
16          intList.remove(0);
17          System.out.println(intList.get(userInput));
18          intList.remove(0);
19          System.out.println(intList.get(userInput));
20          intList.remove(0);
21          System.out.println(intList.get(userInput));
22          intList.remove(0);
23          System.out.println(intList.get(userInput));
24       } catch (IndexOutOfBoundsException ex) {
25          ex.printStackTrace();
26       }
27    }
\end{verbatim}

Incorporate this code into a simple class to verify your answers. 

\section{Exception}
\label{sec:exception}

Read the following code and check whether there is anything wrong with
it. Then write some similar code and check your answer. 

\begin{verbatim}
     // Some code here
     try {
       // more code here
       list.add(newElement);
       // more code here
     } catch (Exception ex) {
       ex.printStackTrace();
     } catch (NullPointerException ex) {
       ex.printStackTrace();    
     }
\end{verbatim}


\section{Error handling on user input}
\label{sec:error-handling-user}

\subsection*{a)}

Write a program that reads 10 numbers from the user and then prints
the mean average. If the user inputs something that is not a number,
the program should complain and ask for a number again until 10
numbers have been introduced. 

\subsection*{b)}

Modify the program so that it first asks how many numbers the user
wants to enter, and then asks for the numbers. The computer should
complain if the user enters something that is not a number in both
cases. Use methods to prevent code repetitions. 

% Create exception and throw it
%      both checked and runtime

\section{More patients}
\label{sec:more-patients}

Write a class that asks for data (name and year of birth) about new patients in a
hospital and keeps them in a list of \verb+Patient+. The constructor
of \verb+Patient+ must throw an \verb+IllegalArgumentException+ if the
age of the patient is negative or greater than 130. 

\section{Prime divisors}
\label{sec:list-without-null}

Create a class \verb+PrimeDivisorList+. Integers (as in class
\verb+Integer+) can be added to~/~removed from the list. If a
\verb+null+ number is
passed to the \verb+add(Integer)+ method, a
\verb+NullPointerException+ must be thrown. If a non-prime number is
added, an \verb+IllegalArgumentException+ must be thrown. 

Override the method \verb+toString()+ so that it returns something
like:

\begin{verbatim}
    [ 2 * 3^2 * 7 = 126 ]
\end{verbatim}

for a list containing one 2, two 3, and one 7. 

Use the TDD methodology to create the class (interface,
tests,implementation).  
You can base your class on classes and interfaces from the Java
Collections Library.

\section{Your first exceptions}
\label{sec:your-first-exception}

Create two exceptions, one checked exception and one runtime
exception. Then create a simple class that will throw each of them in
two different situations, according to user input: 

\begin{enumerate}
\item inside a \verb+try+ block.
\item outside a \verb+try+ block.
\end{enumerate}

Note: Two exceptions times two situations means four different inputs from
users. Create the new exceptions with four different messages (using
the appropriate constructor), e.g.~''I am a checked exception and I
have been thown out of a try block''. 

Assuming you do all of the above inside the \verb+launch()+ method of
your class, did you have to make any changes to the method's
declaration? 

\section{Hierarchies of classes, hierarchies of exceptions}
\label{sec:hier-class-hier}

(Borrowed from Bruce Eckel.) Create a three-level hierarchy of
exceptions (i.e.~an exception extends some exceptions that extends
some exception). Now create a base-class A with a method that throws an
exception at the base of your hierarchy. Inherit B from A and override
the method so it throws an exception at level two of your
hierarchy. Repeat by inheriting class C from B. In the \verb+launch()+
method of another class, create a C
and upcast it to A, then call the method.

\section{Exaggerating list (*)}
\label{sec:exaggerating-list}

Create a class \verb+ExaggeratingList+ that implements 
\verb+java.util.List+ for type integer. 
Implement all methods in the interface, with
the following peculiarities: 

\begin{itemize}
\item When a small element is added to the list, it is transformed
  into a bigger element. What ``small'' means is decided at
  construction time (i.e.~as a parameter of the constructor of
  \verb+ExaggeratingList+).
\item The output of \verb+size()+ is actually twice the number of
  elements in the list.
\end{itemize}

Pay special attention to the exceptions that must be thrown at each
method (according to the interface documentation), especially in the
following two cases:

\begin{itemize}
\item No \verb+null+ elements can be added. Bear this in mind and
  implement the appropriate exceptions to be thrown (according to the
  interface documentation).
\item The list does not support methods \verb+removeAll()+ and
  \verb+retainAll()+. 
\end{itemize}


\section{Bonding (**)}

As a curiosity, look at the following code. Can you understand how it
works and predict its output? What would happen if method
\verb+gotBored()+ returned always \verb+false+? Test yours predictions
against reality by compiling and executing the code. 

Note: this code is \textbf{not} an example of well-written code and is provided
only as a mental exercise. 

\begin{verbatim}
    public class Player {
    
       /*
        * This is the only method worth looking. The rest is boilerplate. 
        */
       public void aim(Ball theBall) {
          try {
             throw theBall;
          } catch (Ball incomingBall) {
             if (gotBored()) {
    	           System.out.println(getName() + ": I got bored. Let's play something else. ");
             } else {
    	           System.out.println(getName() + ": good throw, " + target.getName());
    	           target.aim(incomingBall);
             }
          }
       }

       /*
        * All code below is basically boilerplate and it is
        * shown only for completeness so the code is compilable. 
        */
    
       private Player target;
       private String name;
       
       public Player(String name, Player target) {
          this.name = name;
          this.setTarget(target);
       }
       
       public void setTarget(Player target) {
          this.target = target;
       }
       
       public String getName() {
          return name;
       }
    
       public boolean gotBored() {
          return (Math.random() < 0.001);
       }
       
       public static void main(String...args) {
          Player parent = new Player("Dad", null);
          Player child = new Player("Kid", parent);
          parent.setTarget(child);
          parent.aim(new Ball());
       }
    }
    
    class Ball extends Throwable {}
\end{verbatim}

Acknowledgement: thanks to Randall Munroe for the idea.     
    
\end{document}