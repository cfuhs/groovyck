\section{Extra exercises}
\label{sec:additional-exercises}

This section provides some additional exercises for you, for more
practice with Python.  Use only the concepts presented in this
booklet to solve the exercises.

\subsection*{Exercise X1}

What does %this
the following
program do? This question is not about describing
the instructions in the program one by one. Instead, try to find a
\emph{descriptive name} as a ``summary'' to tell the user of the
program what output it will compute based on the user's inputs.

For example, for the first program in
Section \ref{subsec:backtoloops}, we could give the summary:
\emph{``This program prints
the sum of the numbers that the user has entered.''}
A good name for that program could be \emph{``sum''}.

Try to answer this question without using a computer.

\VerbatimInput[frame=single,label=Example]{src/s3Example6.py}

\subsection*{Exercise X2}

If you have worked out what the above program does, can you
see that, for certain series of numbers, it will not produce the correct
output?  In what circumstances will it not work correctly, and how could
you change the program to make it work properly?
You may assume that the user will enter at least one number that is
not 0 before the final 0.

\subsection*{Exercise X3}

Write a program that takes a series of numbers (ending in 0) and counts
the number of times that the number 100 appears in the list.  For example,
for the series 2, 6, 100, 50, 101, 100, 88, 0, it would output 2.

\subsection*{Exercise X4}

Write a program that takes a series of lines of text (ending by 
an empty string) and,
at the end, outputs the longest line.
You may assume there is at least one line in the input.

\pagebreak


\subsection*{Exercise X5 (this one is a bit harder)}

Write a program that takes
a series of numbers (ending in 0).  If the
current number is the same as the previous number, it says `Same';
if the current number is greater than the previous one, it says `Up',
and if it's less than the previous one, it says `Down'.
It makes no response at all to the
very first number.  For example, its output for the list 9, 9, 8, 5,
10, 10, 0,
would be Same, Down, Down, Up, Same
(comparing, in turn, 9 and 9, 9 and 8, 8 and 5, 5 and 10, 10 and 10).
You may assume there are at least two numbers in the input.

\begin{verbatim}
Enter the first number: 9
Enter the next number (0 to finish): 9
Same
Enter the next number (0 to finish): 8
Down
Enter the next number (0 to finish): 5
Down
Enter the next number (0 to finish): 10
Up
Enter the next number (0 to finish): 10
Same
Enter the next number (0 to finish): 0
\end{verbatim}

\subsection*{Exercise X5bis (still a bit harder)}

Write a solution for exercise X5 that prints all the `Down',
`Same', and `Up' messages together at the end. 

\begin{verbatim}
Enter the first number: 3
Enter the next number (0 to finish): 5
Enter the next number (0 to finish): 4
Enter the next number (0 to finish): 4
Enter the next number (0 to finish): 6
Enter the next number (0 to finish): 8
Enter the next number (0 to finish): 2
Enter the next number (0 to finish): 6
Enter the next number (0 to finish): 7
Enter the next number (0 to finish): 5
Enter the next number (0 to finish): 6
Enter the next number (0 to finish): 6
Enter the next number (0 to finish): 7
Enter the next number (0 to finish): 0
Up Down Same Up Up Down Up Up Down Up Same Up 
\end{verbatim}

%%% Local Variables:
%%% mode: latex
%%% TeX-master: "main"
%%% End:

