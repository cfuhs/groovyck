\documentclass{article}
\usepackage[margin=2cm]{geometry}
\usepackage[dvips]{graphicx}
\begin{document}

\section*{Learning goals}
\label{sec:learning-goals}

Before the next day, you should have achieved the following learning
goals: 

\begin{itemize}
\item Use the main classes from package \verb+java.util.concurrent+.
\item Understand the concepts of \emph{thread pool} and \emph{graceful
  degradation}.
\end{itemize}

\section{Graceful degradation everywhere}
\label{sec:grac-degr-everywh}

Think of three examples of concurrent applications that you use every
week and that benefit from graceful degradation, i.e. becoming
gradually more slow under high load rather than crashing unexpectedly.

Once you have thought of three examples, check them with one of your
colleagues. Have you thought of the same examples?

\section{Text loops re-executed}
\label{sec:text-loops}

Modify the code of the exercise ``Text Loops'' from last day to use
one of the \verb+Executor+ instead of plain threads. 

\section{Responsive UI (that degrades gracefully)}
\label{sec:responsive-ui-that}

Modify your solution of exercise ``Responsive UI'' so that: 

\begin{itemize}
\item It has two classes: one represents the application and one
  represents its users.
\item The former uses an \verb+Executor+ with a thread of pools
  instead of using plain threads.
\item The latter will use another \verb+Executor+ to have a lot of
  threads representing users, and the threads will programatically
  create new tasks instead of the human user doing it by hand,
  i.e. there is no need to ask the user to enter data at any point in
  this version of the exercise. Threads should create tasks as
  fast as the time they need to be run (e.g.~if a task that will make
  the thread sleep for one second, the next task should be created
  a second later). 
\item The ``application'' class should implement a method
  \verb+getMinimumWait()+ that returns the sum of the time needed to
  execute all tasks already in the queue.
\item The ``users'' (threads) will ask for the minimum waiting time
  before they assign a new task. If the wait is above a certain
  threshold, the user should print ``The site is down! I will
  come back later...'' and then wait for a long time before sending
  more tasks. 
\end{itemize}

Try your implementation with different numbers of users and see how
many users it can handle. 

\section{Self-ordering list (*)}
\label{sec:self-ordering-list--}

Re-implement your solutions for the exercise ``Self-ordering list'' by
using an \verb+Executor+. 

\section{Dining philosophers (*)}
\label{sec:dining-philosophers}

Re-implement your solutions for the exercise ``Self-ordering list'' by
using an \verb+Executor+. 

\section{Implementing Executor (**)}
\label{sec:impl-exec-}

Create your own implementation of Executor, with a thread pool of $n$,
where $n$ is set at construction time. Use your implementation in the
former exercises. 


\end{document}