\section{Conditionals (\texttt{if} statements)}

To write anything more than very straightforward programs we need some way
of getting the computer to make choices.
We do this in Python with the keyword \verb!if!.
We can write, for example:

\VerbatimInput[frame=single,label=Example]{src/s2Example4.py}

When the computer executes this, it first sees whether the variable \verb!num!
currently has the value 180 or not.  If it does, the computer displays
its message; if it doesn't, the computer ignores the  \verb!print! line
and goes on to the next line.

Note that the conditional expression (\verb!num == 180!) has to be
followed by a colon (``\verb!:!''), and that
we must put some additional space in front of the code that
we want to execute if the conditional expression is true.
This additional space is also called \emph{indent},
and it tells Python that this code is ``inside'' the \texttt{if}
statement. The usual number of spaces to add to the indent is 4.

%  use curly brackets to indicate what to do if
% the conditional expression is true. 

Note also that, to test whether  \verb!num! has the value 180
or not, we have to write \verb!if num == 180:! and not \verb!if num = 180:!.
We have to remember to hit the ``\texttt{=}'' key twice.
This can be a serious nuisance in Python, especially for beginners.
It comes about because the language uses the ``\texttt{=}'' operator for a different
purpose, namely assignment.  \verb!num = 180! does not mean ``\verb!num! is
equal to 180'', it means ``Give the value 180 to  \verb!num!''.  You may feel
that it is obvious that assignment is not what is intended in
\verb!if num = 180:!, but unfortunately that is
how the computer will interpret it -- and will complain.
You have been warned.

The following program takes two numbers and displays a message if they happen
to be the same:

\VerbatimInput[frame=single,label=Example]{src/s2Example5.py}

\pagebreak

\subsection{Conditional expressions}
\label{sec:cond-expr}

Conditional expressions -- the kind that follow an \verb!if! --
can be formed using the following operators:
\medskip

\begin{tabular}{ll}
\verb+==+ &is equal to\\
\verb+!=+ &is not equal to\\
\verb+>+ &is greater than\\
\verb+<+ &is less than\\
\verb+>=+ & is greater than or equal to\\
\verb+<=+ &is less than or equal to\\
\end{tabular}
\medskip

% Conveniently, Python and Groovy use the same policy.
When these operators are used with integers, their meaning is obvious,
but they can also be used with strings.  Here their meaning corresponds
to something like alphabetical order (the order used for entries in a
printed dictionary).  For instance, \verb!if s < t:!,
where  \verb!s! and \verb!t! are strings, means ``If  \verb!s! comes before \verb!t!
in alphabetical order''.  So it would be true if  \verb!s! had the value
`\verb!Birkbeck!' and \verb!t! had the value `\verb!College!'.  All the
upper-case letters come before the lower-case, so \verb!s < t! would
still be true if  \verb!s! had the value `\verb!Zebra!' and \verb!t! had the
value `\verb!antelope!' (upper-case 'Z' comes before lower-case 'a').

But what about strings that contain non-alphabetic characters?  Would
\verb!s!  come before \verb!t! if \verb!s! had the value ``\verb!#+*!''
and \verb!t! had the value ``\verb+$&!+''?  To find the answer we have
to consult the \emph{UNICODE table} -- the Universal Character Set.
UNICODE defines a particular ordering of all the characters on the
keyboard.  (There are other orderings in use, notably EBCDIC which is
used on IBM mainframe computers, and ASCII, which was adopted by PCs and
became the de facto standard for English-based languages.)
The UNICODE table tells us that the character '\#' comes before the
character '\verb!$!', for instance.  The latest version of Unicode\footnote{The full list can be downloaded from many
places, including Wikipedia
(\texttt{http://en.wikipedia.org/wiki/List\_of\_Unicode\_characters}).}
consists of a repertoire of more than 120,000 characters covering 129
different scripts (including Latin, Cyrillic, Arabic, all the Japanese
ones, and many more). Some points worth remembering are:

\begin{itemize}
\item
The space character comes before all the printable characters.
\item
Numerals come in the order you'd expect, from '0' to '9'.
\item
Letters come in the order you'd expect, from 'A' to 'Z' and from 'a' to 'z'.
\item
Numerals come before upper-case letters and upper-case letters come before lower-case.
\end{itemize}

\pagebreak

\subsection*{Exercise A}

Say, for each of the following pairs of strings, whether \verb!s < t! would be
true or false, assuming that  \verb!s! had the value on the left and \verb!t!
had the value on the right:
\medskip

\begin{tabular}{ll}
\verb!'A'! & \verb!'9'!\\
\verb!'Zurich'! & \verb!'acapulco'!\\
\verb!'Abba'! & \verb!'ABBA'!\\
\verb!'long_thing_with_a_$'! & \verb!'long_thing_with_a_&'!\\
\verb!'King@example.invalid'! & \verb!'King Kong'!\\
\end{tabular}
\medskip

\subsection{Two-way branches (\texttt{if ... else})}
\label{sec:two-way-branches}

The following program fragment tells students whether they have passed
their exam: 

\VerbatimInput[frame=single,label=Example]{src/s2Example6.py}

What happens, in the case of this program, if a student's mark is
less than 50?  The program does nothing.  This kind of \verb!if! statement is a
one-way branch.  If the condition is true, we do something; if not, we do
nothing.  But in this case
this seems unsatisfactory.  If the exam mark is less than 50, we would like
it to display `Sorry, you have failed.'  We could arrange this by
including another test -- \verb!if examMark < 50:!  -- or, better, we could
do it by using the keyword \verb!else!, thus:

\VerbatimInput[frame=single,label=Example]{src/s2Example7.py}

The  \verb!else! turns a one-way branch into a two-way branch.  If the
condition is true, do this; otherwise, do that.

% Let's suppose we want to extend the exam mark program so that candidates who
% have passed get two lines of output, thus:

% \VerbatimInput[frame=single,label=Example]{src/s2Example8.groovy}

% Unfortunately the compiler will object to this.  It will say that it has
% encountered an  \verb!else! in an unexpected place.  What is the problem?

% Although the layout of this program suggests that the "satisfactory" line
% and the "proceed with project" line go together, there is nothing to
% indicate this to the compiler.  As I pointed out earlier, the compiler
% pretty much ignores the layout.  So far as the compiler is concerned,
% we have a one-way  \verb!if! statement (the kind without an  \verb!else!) which ends
% at the first \verb!endl;!  Then there is another  \verb!cout! line (proceed with
% project) which is not part of the  \verb!if! statement; it's just the first
% line of the rest of the program.  And then, unexpectedly,
% appears an  \verb!else!.

% We need some way of bracketing together
% the "satisfactory" line and the "proceed with project" line so as to make
% it clear to the compiler that both of these lines come under the  \verb!if!.
% This is how we do it in Groovy:

%A note about the curly braces.
A note about the indent (the spaces we put before the statements
``inside'' the \texttt{if} statement). It has the effect of
grouping all the statements with the same indentation level (the same
amount of whitespace before them) into a programming unit called a
\emph{block}.
% \footnote{A block
%   consisting only of one statement does not need curly
%   brackets. However, I recommend to use them always for
%   clarity.}.
Look at this example: 
\pagebreak

\VerbatimInput[frame=single,label=Example]{src/s2Example9.py}

If the exam mark is greater than or equal to 50, 
the whole of the block is executed and both
lines `A satisfactory result!' and `You may proceed with your
project.' will be printed.  If the mark is lower than 50, 
the computer skips to
the \verb!else! and executes the `Sorry' line.

We can reduce the indent level again to tell Python where e.g.\
the block inside the \verb!else! ends. We might add
a line to the previous example to print a `Good bye.'\ message
in all cases:

\VerbatimInput[frame=single,label=Example]{src/s2Example9a.py}


% You will find that different programmers, and different textbooks, have
% different ideas about the precise placement of the curly braces.  Some would
% set out the above fragment as:

% \VerbatimInput[frame=single,label=Example]{src/s2Example10.groovy}

% and there are other variations. Personally I prefer the first version
% because it is more compact but at the same time the structure of the
% program is very clear. 

Suppose now that we wanted to give a different message to candidates who
had done exceptionally well. Our first thought might be as follows:

\VerbatimInput[frame=single,label=Example]{src/s2Example11.py}

But this would not work quite right.  It's OK for candidates with
marks below 70,
but candidates with marks greater than or equal to 70 
would give the following output:

\begin{Verbatim}
    An exceptional result
    We expect a first-class project from you.
    A satisfactory result
    You may proceed with your project.
\end{Verbatim}

The problem is that if a mark is greater than or equal to~70, it is
also greater than~50.  The first
condition is true, so we get the ``exceptional'' part, but then the second
condition is also true, so we get the ``satisfactory'' part.  We want to
proceed to the greater than~50 test only if the mark is below~70.
Here Python offers us the keyword \verb!elif!, which is short for
``else if'' and which allows us to avoid excessive indentation:

\VerbatimInput[frame=single,label=Example]{src/s2Example12.py}

We can write more complex conditional expressions by joining them together
using the operator \textbf{or}.  For example, the following additional
condition at the start of the last program checks if the exam mark keyed in
is invalid, i.e., less than 0 or greater than 100:

\VerbatimInput[frame=single,label=Example]{src/s2Example13.py}


\subsection*{Exercise B}

Write a program that takes two numbers as input, one representing a
husband's salary and the other representing the wife's, and prints out
whether or not their combined income makes them due for tax at the higher rate
(i.e., it exceeds \pounds 40,000).

\subsection*{Exercise C}

Extend the program about students' marks so that all the candidates
get two lines of output, the unsuccessful ones getting `Sorry, you
have failed.'\ and `You may re-enter next year.'

\subsection*{Exercise D}

Write a program which takes two integers as input.
If the first is exactly divisible
by the second (such as 10 and 5 or 24 and 8, but not 10 and 3 or 24 and 7)
it outputs `Yes', otherwise `No', except when the second
is zero, in which case it outputs `Cannot divide by zero'. Remember
you can use the modulo operator (``\texttt{\%}'') to find out whether one
number is divisible by another. 

\subsection*{Exercise E}

Write a program which takes an integer as its input, representing the time
using the 24-hour clock.  For example, 930 is 9.30 am; 2345 is 11.45 pm.  Midnight is
zero. The program responds with a suitable greeting for the time of day
(e.g., `Good morning', `Good afternoon', or `Good evening').
If you want to make this a bit harder, make the program respond with a '?'
if the time represented by the number is impossible, such as 2400, -5 or
1163.

% \subsection{The UNICODE character set}

% The portion of the UNICODE table dealing with printable characters is as
% follows.  The numbers indicate the character's position in the UNICODE
% character set;  upper-case 'A', for example, is character number 65.
% Character number 32 is a space.


%%% Local Variables:
%%% mode: latex
%%% TeX-master: "main"
%%% End:

