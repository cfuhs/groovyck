\documentclass[11pt,a4paper]{article} 
\usepackage[pdftex]{geometry}
\usepackage{fancyvrb}
%%%\usepackage[dvips]{graphicx}
\usepackage{graphicx}
%\usepackage{moreverb} % env: comment
%\usepackage{alltt} 
%\usepackage{pifont} 
%\usepackage{color} 
%\usepackage{subfigure}
%\usepackage{listings}
%\usepackage{amssymb}
%\usepackage{epstopdf}
%\DeclareGraphicsRule{.tif}{png}{.png}{`convert #1 `dirname #1`/`basename #1 .tif`.png}
%\usepackage{draftwatermark} 
%\SetWatermarkLightness{0.9}
%\SetWatermarkFontSize{20cm}
%\SetWatermarkScale{5.0} 

\newcommand{\todo}[1]{\marginpar{\textsf{!!! #1}}}


\title{A Short Introduction to Computer Programming\\
  Using Python}
\author{Carsten Fuhs\\
  (based on earlier documents by\\
  Sergio Gutierrez-Santos, Keith Mannock, and Roger Mitton)\\
  Birkbeck, University of London}

\date{\InputIfFileExists{version.txt}
     {v}
     {\emph{``version.txt'' not found}}}

\pagestyle{plain}

\begin{document}

%%%%%%%%%%%%%%%%%%%%%%%%%
%      TITLE PAGE       %
%%%%%%%%%%%%%%%%%%%%%%%%%

\maketitle

\thispagestyle{empty}

\vfill 

\begin{figure*}[h!]  % BBK logo
  \centering
%  \includegraphics[height=2cm]{bbk.eps}
  \includegraphics[height=2cm]{Birkbeck-Logo-Colour-330x104.jpg}
\end{figure*}

\vfill

\noindent This document forms part of the pre-course reading for
several MSc courses at Birkbeck.
% , and it is required pre-interview
% reading for the MSc Computer Science and the MSc Data Science.

%\vfill 

\noindent (Last revision: \today.)

%\vfill 

\newpage

~\vspace{8cm}

\textbf{Important note: } If anything is unclear, or you have any 
sort of problem (e.g.,
downloading or installing Python on your computer), please send an
email to Carsten Fuhs (\emph{carsten@dcs.bbk.ac.uk}).
\newpage

\section{Introduction}

%Simple input/output with integers, outputting words and ends of lines, assignment and initialisation}

%% Why should YOU want to write computer programs?
\athena{Computing increasingly permeates everyday life,
from the apps on our phones to the social networks that connect us.
Thus, there is a need for people to
% have the skill of not just using
%be able to
not just use
such systems made by others, but
%also
to be empowered to create their own.
%  and modify them, and ultimately
% to shape the future of our society.
}

To get a computer to do something, you have to tell it what you want it
to do.  You give it a series of instructions in the form of a \emph{program}.
You write a program in a \emph{programming language}.
Many programming languages have been devised \athena{over the decades};
well-known ones include
Fortran, BASIC, Cobol, Algol, Lisp, Ada, C++, Python, Ruby, Scala, and Java.

% The language we will use in this document is named \emph{Groovy}, and
% it is \emph{very similar to the Java programming language but with a
% simplified syntax} that better fits the purposes of this introduction. 

The language we will use in this document is named \emph{Python}.
More precisely, we are using version 3 of the Python language.
\athena{Python hits a sweet spot: it makes writing simple programs easy
and is robust enough for a wide uptake in a variety of industries.}

Programming languages vary a lot, but
an instruction in a typical programming language might contain
some English words (such as \verb!while! or \verb!return!), perhaps a mathematical
expression (such as \verb!x + 2!) and some punctuation marks used in a special way.
The following three lines are in BASIC, Cobol and C++ respectively:
\begin{Verbatim}
      FOR I% = 1 TO 100 STEP 5 DO

      PERFORM RECORD-TRANSACTION UNTIL ALL-DONE = "T"

      if (x > 100) cout << "Big"; else cout << "Small";
\end{Verbatim}

Programs can vary in length from a few lines to thousands of lines.
Here is a complete, short program written in Python:

\VerbatimInput[frame=single,label=Example]{src/Example1.py}

\pagebreak

When you write a program, you have to be very careful to keep to the
syntax of the programming language, i.e., to the grammar rules of the language.
For example, the above Python program would
behave incorrectly if in the 12$^{th}$ line we wrote
%\pagebreak

\begin{Verbatim}
    if maxi = 0:
\end{Verbatim}

instead of 

\begin{Verbatim}
    if maxi == 0:
\end{Verbatim}

or if in the 9$^{th}$ line we omitted the colon in

\begin{Verbatim}
        if len(word) > maxi:
\end{Verbatim}

The computer would refuse to accept the program if we added dots or colons
in strange places
or if any of the parentheses (or colons) were missing, or if we wrote
\verb!length! instead of \verb!len! or even \verb!Len! instead of \verb!len!.

% Moved to earlier:
% 
%The language we will use in this document is named \emph{Groovy}, and
%it is \emph{very similar to the Java programming language but with a
%simplified syntax} that better fits the purposes of this introduction. 

\subsection{Input and output}

\athena{Almost all programs need to communicate with the outside world to
be useful. In particular, they need to read input and to remember it. They
also need to respond with an answer.}

To get the computer to take some input \athena{as text} and to store it
in its memory, we write in Python:

\begin{Verbatim}
      word = input()
\end{Verbatim}

\verb!input()!
% (pronounced \emph{system dot console dot read line})
is a phrase which has a special meaning to
the computer. The combination of these instructions
means
% \footnote{More precisely, it means ``Take the current system
%   --- this computer, take its console --- its keyboard and screen,
% and read a line of character input from it as entered by the user'' (from the keyboard,
% not the screen because the user cannot type a line onto the screen... or
% could not before they invented touchscreens!).}
``Take some input which is a sequence of characters and put it into the computer's memory.''

\verb!word! by contrast, is a word that I (the programmer) have chosen.
I could have used \verb!characters! or \verb!thingy! or \verb!breakfast!
or just \verb!s! or almost any word I wanted. (There are some
restrictions which I will deal with in the next section.)  
% You may occasionally see the semi-colon used in a program and that signals the
% end of this instruction.

Computers can take in all sorts of things as input --- numbers,
letters, words, records and so on --- but, to begin with, we will
write programs that generally handle
% strings
\athena{text as sequences (or \emph{strings})} of characters (like
"I", "hi", "My mother has 3 cats", or "This is AWESOME!").
We'll also assume that the computer is taking its input from the
keyboard, i.e., when the program is executed, you key in one or more
words at the keyboard and these are the characters that the computer 
puts into its memory.

You can imagine the memory of the computer as consisting of lots of
little boxes.  Programmers can reserve some of these boxes for
use by their programs and they refer to these boxes by giving
them names.  

\begin{Verbatim}
      word = input()
\end{Verbatim}

means ``Take a string of characters input using the keyboard and
terminated when the user presses RETURN, and
then put this string into the box called \verb!word!.''  When the program runs and
this instruction gets executed, the computer will take the words which
you type at the keyboard (whatever you like) and will put them into
the box which has the name \verb!word!.

Each of these boxes in the computer's memory can hold only one string at a time.  If a box is
holding, say, "hello", and you put "Good bye!"\ into it, the
"Good bye!"\ replaces the "hello".  In computer parlance these
boxes are called \emph{variables} because the content inside the box
can vary; you can put an "I" in it to start with and later change it
to a "you" and change it again to "That was a hurricane" and so on
as often as you want.
Your program can have as many variables as you want.


% In Groovy, you have to tell the computer that you want to use a variable
% with a certain name before you use it.  You can't just pitch in using \verb!word!
% without telling the computer what \verb!word! is.
%   Most of the time you also have to
% tell the computer what {\em type} of variable this is, i.e., what sort of thing you are going
% to put into it. In this case we are going to put strings of characters into \verb!word!. 
% Strings of characters, or simply \emph{strings} as they are called in programming, are
% known in Groovy as \verb!String!, with capital S.  To tell the computer that we want to use a
% variable of type \verb!String! called \verb!str!, we write in our program:
%
% \begin{Verbatim}
%       String str
% \end{Verbatim}
%
% If we wanted to declare more than one variable, we could use two lines:
%
% \begin{Verbatim}
%       String myName
%       String yourName
% \end{Verbatim}
%
% or we could declare them both on one line, separated by a comma:
% \begin{Verbatim}
%       String myName, yourName
% \end{Verbatim}
%
In Python (and many other programming languages), you have to make sure
that you have put something into a variable before you read from it,
e.g., to print its contents. If you try to take something out of a box
into which you have not yet put anything, it is not clear what that
``something'' should be. This is why Python will complain if you write
a program that reads from a variable into which nothing has yet been put.


%   In Groovy you don't
% have to declare all your variables at the start of the program, as is the
% case in some other languages.  You can declare them in the middle of a
% program, but you mustn't try to use a variable before you've declared it.

If you want the computer to display (on the screen) the contents of one of
its boxes, you use \athena{\texttt{print(thingy)}} where instead of \athena{\texttt{thingy}}
you write the name of the box.  For example, we can print the contents of
\verb!word! by writing:

\begin{Verbatim}
      print(word)
\end{Verbatim}

If the contents of the box called \verb!word! happened to be "Karl", then when the
computer came to execute the instruction \verb!print(word)! the word "Karl" would appear
on the screen.  



% Arithmetic expressions such as \verb!num + 5! can also appear in a \verb!print!
% line.  For example, if \verb!num! had the value 7, then \verb!print num + 5!
% would output 12.
So we can now write a program in Python (not a very exciting program,
but it's a start):

\VerbatimInput[frame=single,label=Example]{src/Example2.py}

This program takes some text as input from the keyboard and displays it
back on the screen.

In many programming languages it is customary to lay out programs like this
or in our first example with each instruction on a line of its own.
In these languages, the indentation, if there is any, is just for the benefit
of human readers, not for the computer which ignores any indentations.

Python goes a step further and uses indentation as a part of the language
to structure the programs. We will see more about this a bit later.

\pagebreak

\subsection{Running a program}

You can learn the rudiments of Python from these notes just by doing the
exercises with pencil and paper.
\athena{However, it is a good idea to test your programs using a computer
after you have written them so you can validate that they do what you
intended. If}
%   It is not essential to run your programs
% on a computer.
%However, if
you have a computer and are wondering how
you run
% the
\athena{your}
programs, you will need to know the following, and, even if you don't have
a computer, it will help if you have some idea of how it's done.

First of all you type your program into the computer using a text
editor.
Then, to run your program, you use a piece of software called
an \emph{interpreter}.
The interpreter first checks whether your program is acceptable according
to the syntax of Python and then runs it.
If it isn't acceptable according to the syntax of Python,
the interpreter issues one or more error messages telling you
what it objects to in your program and where the problem lies.
You try to see what the problem is, correct it and try again.  You keep
doing this until the interpreter runs your program without further
objections.
%  Then, 
% before you can run the program, you have to \emph{compile} it.  This means that
% you pass it through a piece
% of software called a \emph{compiler}.  The compiler checks whether your program
% is acceptable according to the syntax of Groovy.  If it isn't, the compiler
% issues one or more error messages telling you
% what it objects to in your program and where the problem lies.
% You try to see what the problem is, correct it and try again.  You keep
% doing this until the program compiles successfully.  You now have an
% {\em executable} version of your program, i.e., your program has been translated
% into the internal machine instructions of the computer and the computer
% can run your program.

To make the computer run your program using the interpreter, you need to
issue a command. You can issue commands\footnote{In
    a modern operating system, you can click on an icon to execute a
    program. However, this only makes sense for graphical applications
    and not for the simple programs that you will write at first.}
from the command prompt in Windows (you can find the command prompt under 
\emph{Start $\rightarrow$
    Accessories}), or from the terminal in Linux and Mac OS/X.


If you are lucky, your program does what you expect it 
to do first time.  Often it doesn't.  You look at what your program is
doing, look again at your program and try to see why it is not doing what
you intended.  You correct the program and run it again.
You might have to do this many times before the program behaves in the way
you wanted. \athena{This is normal and nothing to worry about at this stage.}

As I said earlier, you can study this introduction without running your
programs on a computer.  However, it's possible that you have a PC with
a Python interpreter and will try to run some of the programs given in these
notes. 
If you have a PC but you don't have a
Python interpreter, I attach a few notes telling
you how you can obtain one (see Section~\ref{sec:obta-inst-runn}).

\athena{If you do not wish to install Python on your machine, you can use
the online Python editor and interpreter at
\begin{center}
\url{https://repl.it/languages/python3}
\end{center}
Have a look at the intro video at
\begin{center}
\url{https://replit.github.io/media/quick-start/simple-repl.mp4}
\end{center}
and the quick-start guide
\begin{center}
\url{https://repl.it/site/docs/misc/quick-start}
\end{center}
}

%\pagebreak

\subsection{Outputting words and ends of lines}

% \athena{So far, we've been able to talk to the computer.
% For the computer to talk back to us, we need
% }

Let's suppose that you manage to %compile
get your program to run.
%and that you then run it.
Your running of the above program
would produce something like this on the screen if you typed in
the word Tom followed by RETURN: 

\begin{Verbatim}
Tom
Tom
\end{Verbatim}

The first line is the result of you keying in the word Tom.  The system ``echoes''
the keystrokes to the screen, in the usual way.  When you hit RETURN, the
computer executes the \verb!word = input()! instruction, 
i.e., it reads the word or words that have been input.  Then it
executes the \verb!print! instruction and the word or words that were
input appear on the screen again.

We can also get the computer to display additional words by putting
them in quotes\footnote{In Python, we can choose between \texttt{"} and \texttt{'}, but we cannot use both together.}
in the parentheses after the \verb!print!, for example:

\begin{Verbatim}
      print("Hello")
\end{Verbatim}

We can use this to
% improve the above program:
\athena{make the above program more user-friendly:}

\VerbatimInput[frame=single,label=Example]{src/Example3.py}

An execution, or ``run'', of this program might appear on the screen thus:
\begin{Verbatim}
Please key in a word:
Tom
The word was:
Tom
\end{Verbatim}

Here each time the computer executed the \verb!print! instruction,
it also went into a new line.
If we use \verb!print(word, end = "")! instead of \verb!print(word)!
then no extra line is printed after the value.\footnote{The reason is that
 with \Verb!end = ""! we are telling Python that after printing the contents
 of \Verb!word!, we want to print nothing at the end instead of going to a
 new line.}
So we can improve \athena{the formatting of the output of our program
on the screen:}
% our program a bit further:

\VerbatimInput[frame=single,label=Example]{src/Example3a.py}

Now a ``run'' of this program might appear on the screen thus:
\begin{Verbatim}
Please key in a word: Tom
The word was: Tom
\end{Verbatim}

Note the spaces in lines 1 and 3 of the program after \verb!word:! and \verb!was:!.
This is so that what appears on the screen is \verb!word: Tom! and
\verb!was: Tom! rather than \verb!word:Tom! and \verb!was:Tom!.

It's possible to output more than one item with a single \verb!print! instruction.
For example, we could combine the last two lines of the above program into one:

\begin{Verbatim}
print("The word was " + word)
\end{Verbatim}

and the output would be exactly the same. The symbol ``\texttt{+}'' does not
represent addition in this context, it represents concatenation, i.e.,
writing one string after the other. We will look at addition of numbers in the next section.


Let's suppose that we now added three lines to the end of our program, thus:

\VerbatimInput[frame=single,label=Example]{src/Example4.py}

% After running this program, the screen would look something like this:
% \begin{Verbatim}
% Please key in a word: Tom
% The word was TomNow please key in another: Jack
% And this one was Jack
% \end{Verbatim}

% which is probably not what we wanted.  If we want a new line after the
% first word is printed, we need to use the \verb!println! instruction we mentioned earlier:

% \VerbatimInput[frame=single,label=Example]{src/Example5.groovy}

% Now we would get:

After running this program, the screen would look something like this:
\begin{Verbatim}
Please key in a word: Tom
The word was Tom
Now please key in another: Jane
And this one was Jane
\end{Verbatim}

\subsection*{Exercise A}

\athena{
For those of you who prefer a ``learning by doing'' approach,
the University of Waterloo provides an interactive
tutorial to programming in Python 3 with a similar structure to
this booklet:
\begin{center}
\url{https://cscircles.cemc.uwaterloo.ca/}
\end{center}
This tutorial gives you further exercises and checks
whether the output obtained from your solutions is correct.
We will occasionally refer to exercises in this on-line
tutorial.
You are welcome to create an account if you wish to record your progress.
However, as this booklet is designed for self-study, we will not
provide any assistance for this system.\\
\indent
As your first exercise, read through Section ``0: Hello'' at the
above link and attempt the exercises in your browser as you
go along. The web page will provide you with direct feedback.
}

\subsection*{Exercise B}

Now pause and see if you can write:
\begin{enumerate}
\item
a Python instruction which would output a blank line.
\item
an instruction which would output

\begin{Verbatim}
Hickory, Dickory, Dock
\end{Verbatim}

\item
a program which reads in two words, one after the other, and then displays them
in reverse order. For example, if the input was 

\begin{Verbatim}
First
Second
\end{Verbatim}

the output should be

\begin{Verbatim}
Second
First
\end{Verbatim}

\end{enumerate}

%To check your answers, click on  \verb!Answers to the exercises!. (TO BE COMPLETED)

\subsection{Assignment and initialisation}

There is another way to get a string into a box apart from using
\verb!input()!.  We can write, for instance:

\begin{Verbatim}
    word = "Some text"
\end{Verbatim}

This has the effect of putting the string "Some text" into the
\verb!word! box.  Whatever content was in \verb!word! before is obliterated; the
new text replaces the old one. 

In programming, this is called \emph{assignment}.  We say that the
value "Some text" is
assigned to the variable \verb!word!, or that the variable \verb!word! takes the
value "Some text".  The ``\texttt{=}'' symbol is the assignment operator in Python.  We are not
testing whether \verb!word! has the value "Some text" or not, nor are we stating that
\verb!word! has the value "Some text"; we are \emph{giving} the value
"Some text" to \verb!word!.

% Unneeded. Probably should delete.
%
% If we want, we can have arithmetic expressions on the right-hand side of
% the "=", for example:
% \begin{Verbatim}
% num = count + 10
% \end{Verbatim}
% This instruction means, "Take whatever number is in \verb!count!,
% add 10 to it and put the result into \verb!num!."

An assignment instruction such as this:

\begin{Verbatim}
    word = word + " and some more"
\end{Verbatim}

looks a little strange at first but makes perfectly good sense.
Let's suppose the current value of \verb!word! (the contents of the
box) is "Some text".
The instruction says, ``Take the value of \verb!word!
("Some text"), add " and some more"  to it (obtaining "Some text and some
more") 
and put the result into \verb!word!''.
So the effect is to put "Some text and some more" into \verb!word! in
place of the earlier "Some text".


% We also say that the first time we put something into a variable \verb!foo!,
% we \emph{define} \verb!foo!.


%%% Python does not to uninitialised variables :)
The \verb!=! operator is also used to ``initialise'' variables.
In Python, a variable is \emph{defined} by the first assignment to
that variable, the first time you put something into it. This first
assignment is often also called \emph{initialisation}.
You can read from a variable only after it has been initialised.

% The reason for allowing to read from a variable only after something
% is put into it is as follows.
% When the computer allocates a portion of memory to store one of your
% variables, it does not clear it for you; the variable holds whatever
% value this portion of memory happened to have the last time it was
% used.
% %  Its value is said to be \emph{undefined.}
% To prevent strange values coming from such an ``uninitialised''
% portion of memory, Python will complain if you try to read from a
% variable into which you have not put anything yet.

% Earlier I said that in Python you have to make sure that you have put
% something into a variable before you read from it, e.g., to print its
% contents. 

% Using undefined values is a common cause of program
% bugs.  
% % Suppose a program uses the variable \verb!str! without giving it an
% % initial value and suppose that, on the computer the programmer is using,
% % the initial value in \verb!num! happens to be zero and that, by happy chance,
% % zero is just what the programmer wants it to be.
% % The program seems to work fine.  Then the program is compiled and run on
% % a different computer.  On this second computer, the initial value of \verb!num!
% % does not happen to be zero.  The program, which has worked OK on the first
% % computer, does not work on the second one.
% %
% In Python, you have to make sure that you have put something into a
% variable before you read from it, e.g., to print its contents.

% To prevent yourself from using undefined values, you can give a variable
% an initial value when you declare it.  For example, if you wanted \verb!str! to begin
% with empty, you should declare it thus:

% \begin{Verbatim}
%       String str = ""
% \end{Verbatim}

% This is very like assignment since we are giving a value to \verb!str!. But
% this is a special case where \verb!str! did not have any defined value before,
% so it is known as \emph{initialisation.}

Finally a word about terminology.  I have used the word ``instruction''
to refer to lines such as \verb!print(word)! and \verb!word = "Some text"!.
It seems a natural word to use since we are giving the computer instructions.
But the correct word is actually ``statement''.  \verb!print(word)! is
an output statement, and \verb!word = "Hello"! is an assignment statement.
% The lines in which we tell the computer about the variables we intend to
% use, such as \verb!String str! or \verb!String str = ""! are called
% variable \emph{definitions}.  They are also referred to as 
% variable \emph{declarations}.  When you learn more about Groovy you will
% find that you can have declarations which are not definitions, but the
% ones in these introductory notes are both definitions and declarations.


\subsection*{Exercise C}

Now see if you can write a program in Python that takes
two words from the keyboard and outputs one after the other on the same line. 
E.g., if you keyed in ``Humpty'' and ``Dumpty'' 
it would reply with ``Humpty Dumpty'' (note the space in between).
A run of the program should look like this:

\begin{Verbatim}
    Please key in a word: Humpty
    And now key in another: Dumpty
    You have typed: Humpty Dumpty
\end{Verbatim}

%To check your answers, click on \verb!Answers to the exercises!. (TO BE COMPLETED)

 	

%%% Local Variables:
%%% TeX-master: "primer"
%%% End:


\newpage

\section{Variables, identifiers and expressions}

In this section we will discuss integer numbers, arithmetic expressions, 
identifiers, comments, and strings.

%\subsection{Integer variables}
\subsection{Integer numbers}

As well as strings, we can also have integer numbers (often just
\emph{integers}) as values: whole numbers.
%Integers
So variables can contain not only strings, but also any
% \footnote{Actually, a computer's memory
%   is finite and therefore they put limits to the possible values for an
%   integer variable, but they are big enough for most programs.}
integer value like~0,~-1,~-200, or~1966.\footnote{In contrast to many
  other programming languages, Python does not have a limit to how
  large an integer value can become.}

% We would declare an \verb!int! variable called \verb!i! as follows:

% \begin{Verbatim}
%     int i
% \end{Verbatim}

We can assign an integer value, say, \verb!0!,
to a variable, say, \verb!i!:

\begin{Verbatim}
    i = 0
\end{Verbatim}

% We can change the value of an integer with an assignment:

% \begin{Verbatim}
%     i = 1
% \end{Verbatim}

or, if we had two variables \verb!i! and \verb!j!, we could assign:

\begin{Verbatim}
    i = j
\end{Verbatim}

We can even say that \verb.i. takes the result of adding two numbers
together:

\begin{Verbatim}
    i = 2 + 2
\end{Verbatim}

which results in \verb.i. having the value 4. 

\subsection*{Integers and strings}
\label{sec:intstr}

You have probably noticed that, when dealing with integer
values the symbol ``\texttt{+}'' represents addition of numbers, while -- as
we saw in the last section -- when dealing with strings the same
symbol represents concatenation. Therefore, the statement

\begin{Verbatim}
    word = 'My name is ' + 'Inigo Montoya'
\end{Verbatim}

results in \verb+word+ having the value `My name is Inigo Montoya',
while the statement

\begin{Verbatim}
    n = 10 + 7
\end{Verbatim}

results in \verb+n+ having the value 17 (not 107). What happens if we
mix integer values and string values when using ``\texttt{+}'', say in the
following statement?

\begin{Verbatim}
    answer = 'Answer: ' + 42
\end{Verbatim}

In that case, Python will give us an error message similar to the
following one:

\begin{Verbatim}
TypeError: cannot concatenate 'str' and 'int' objects
\end{Verbatim}

This means that Python does not know how to connect (``concatenate'')
the string `Answer: '\ (Python uses the name ``str'' for the
\emph{type} of strings) and the integer 42 (``int'').
To clarify that we want to use ``\texttt{+}'' to concatenate two strings (and
not to use ``\texttt{+}'' to add two values), we can convert the ``int'' value
42 to the ``str'' value `42' with the command \verb!str()!:

\begin{verbatim}
    answer = 'Answer: ' + str(42)
\end{verbatim}

%  converts the integers to strings and performs
% concatenation.
% For example,
Similarly, the small program

\begin{Verbatim}
    word = 'My name is ' + 'Inigo Montoya'
    n = 10 + 7
    text = word + ' and I am ' + str(n)
    print(text)
\end{Verbatim}

will print on the screen ``My name is Inigo Montoya and I am 17''. It
is important to know that the same symbol can be used for different
things, but we will come to this later again; now let's go back to
writing programs with a bit of maths in them using integers.

\subsection{Reading integers from the keyboard}
\label{sec:intkeyboard}

In the last section, we saw how we can read a string of characters
from the keyboard, 
using \verb+input()+. We can use the same command
to read a number\ldots but the computer will not know it is a number,
it will think it is a string of characters. If we want to tell the
computer that a sequence of characters \emph{is} a number,
we need to convert it.
This is similar to the way we needed to tell the computer that we
wanted to treat a number as a character string in the previous example.
We can do this easily by \emph{parsing} it using the command
\verb+int()+: 

\VerbatimInput[frame=single,label=Example]{src/s2Example1.py}

When we parse a string that contains an integer we obtain an integer
with the correct value. If we try to parse a string that is not an
integer (for example, the word `Tom') the program will terminate with an error
message on the screen. If we do not parse the string and use it as if
it was an integer, the results will be unpredictable. This
is a common source of errors in programs. You can check for yourself
what happens if you do not parse the string in the former example,
e.g., what happens with this program:

\VerbatimInput[frame=single,label=Example]{src/s2Example1_noParsing.py}

Now, assuming that you read your integers and always remember to parse
them, what maths can you do with them?

\subsection{Operator precedence}
\label{sec:prec}

Python uses the following arithmetic operators (amongst others):
\bigskip

\begin{tabular}{ll}
\verb!+! & addition\\
\verb!-! & subtraction\\
\verb!*! & multiplication\\
\verb!//! & division\\
\verb!%! & modulo\\
\end{tabular}
\bigskip

The last one is perhaps unfamiliar.  The result of \verb!x % y! (``x mod y'')
is the remainder that you get after dividing the integer x by the integer y.
For example, \verb!13 % 5! is \verb!3!; \verb!18 % 4! is \verb!2!; 
\verb!21 % 7! is \verb!0!, and \verb!4 % 6 is 4! (6 into 4 won't go, 
remainder 4).  \verb!num = 20 % 7! would assign the value 6
to \verb!num!.

How does the computer evaluate an expression containing more than one
operator? For example, given \verb!2 + 3 * 4!, does it do the
addition first, thus getting \verb!5 * 4!, which comes to 20, or does
it do the multiplication first, thus getting \verb!2 + 12!, which
comes to 14?  Python, in common with other programming languages and
with mathematical convention in general, gives precedence to \verb!*!, \verb!//! and
\verb!%! over \verb!+! and \verb!-!.  This means that, in the example, it does the
multiplication first and gets 14.

If the arithmetic operators are at the same level of precedence, it
takes them left to right.  \verb!10 - 5 - 2! comes to 3, not 7.  You
can always override the order of precedence by putting brackets into
the expression; \verb!(2 + 3) * 4! comes to 20, and \verb!10 - (5 - 2)! comes
to 7.

Some words of warning are needed about division. First, remember that
integer values are whole numbers.
So the result of a division with \verb-//- is
only the integer part without the decimal part. Try to
understand what the following program does:
\pagebreak

\VerbatimInput[frame=single,label=Example]{src/s2Example.py}

A computer would get into difficulty if it tried to divide by zero.
Consequently, the system makes sure that it never does.
If a program tries to get the computer to divide by zero, the program
is unceremoniously terminated, usually with an error message on the
screen.

% You may also have wondered why we write \verb!//! instead of \verb!/!
% for integer division in Python. The reason is that \verb!/! already
% stands for an operation called ``floating point division''. This
% operation does not cut off the decimal part of the result, but it will
% still need to ``round'' the result because not all decimal numbers
% (say, $1/3 = 0.333333...$) can be represented as a decimal fraction.

\subsection*{Exercise A}

Write down the output of the above program without executing it.

% \VerbatimInput[frame=single,label=Example]{src/s2Example.groovy}

Now execute it and check if you got the right values. Did you get them
all right? If you got some wrong, why was it?


\subsection{Identifiers and comments}

I said earlier that you could use more or less any names for your variables.
I now need to qualify that.

The names that the programmer invents are called \emph{identifiers}.  The
rules for forming identifiers are that the first character can be a letter
(upper or lower case, usually the latter) and subsequent characters
can be letters or digits 
or underscores.  (Actually the first character can be an underscore, but
identifiers beginning with an underscore are often used by system programs
and are best avoided.)  Other characters are not allowed.  Python is
case-sensitive, so \verb!Num!, for example, is a different identifier from
\verb!num!.

The only other restriction is that you cannot use any of the language's keywords
as an identifier.  You couldn't use \verb!if! as the name of a variable,
for example. There are many keywords but most of them are words that you
are unlikely to choose. 
Ones that you might accidentally hit upon are
\texttt{assert}, \texttt{break}, \texttt{class}, \texttt{continue},
\texttt{def}, \texttt{del}, \texttt{except}, \texttt{finally},
\texttt{global}, \texttt{import}, \texttt{pass}, \texttt{raise},
\texttt{return}, \texttt{try} and \texttt{yield}. You should also avoid 
using words which, though not technically keywords, have special significance
in the language, such as \verb!int! and \verb!print!.

Programmers often use very short names for variables, such as
\verb!i!, \verb!n!, or 
\verb!x! for integers.  There is no harm in this if the variable is used to
do an obvious job, such as counting the number of times the program goes round
a loop, and its purpose is immediately clear from the context.  If, however, its
function is not so obvious, \textbf{it should be given a name that
gives a clue as to the role it plays in the program}.  If a variable is
holding the total of a series of integers and another is holding the
largest of a series of integers, for example, then call them \verb!total!
and \verb!maxi! rather than \verb!x! and \verb!y!.

The aim in programming is to write programs that are ``self-documenting'',
meaning that a (human) reader can understand them without having to read
any supplementary documentation.  A good choice of identifiers helps to
make a program self-documenting.

Comments provide another way to help the human reader to understand a
program.  Anything on a line after ``\texttt{\#}'' is ignored by the interpreter,
so you can use this to annotate your program.  You might summarise
what a chunk of program is doing:

\begin{Verbatim}
    # sorts numbers into ascending order
\end{Verbatim}

or explain the purpose of an obscure bit:

\begin{Verbatim}
    x = x * 100 // y   # x as percent of y
\end{Verbatim}

Comments should be few and helpful.  Do not clutter your programs with
statements of the obvious such as:

\begin{Verbatim}
    num = num + 1   # add 1 to num
\end{Verbatim}

Judicious use of comments can add greatly to a program's readability, but they
are second-best to self-documentation. Note that comments are %all but
ignored by the computer: their weakness is that it is all too
easy for a programmer to modify the program but forget to make any
corresponding modification to the comments, so the comments no longer quite
go with the program. At worst, the comments can become so out-of-date as
to be positively misleading, as illustrated in this case: 

\begin{verbatim}
    num = num + 1  # decrease num
\end{verbatim}

What is wrong here? Is the comment out of date? Is there a bug in the
code and the plus should be a minus? Remember, use comments sparingly
and make them matter. A good rule of thumb is that comments should
explain ``why'' and not ``how'': the code already says \emph{how}
things are done! 

\subsection*{Exercise B}

Say for each of the following whether it is a valid identifier
in Python and, if not, why not:
\begin{Verbatim}
BBC, Python, y2k, Y2K, old, new, 3GL, a.out, 
first-choice, 2nd_choice, third_choice
\end{Verbatim}

\subsection{A bit more on strings}

We have done already many things with strings: we have
created them, printed them on the screen, even concatenated several of
them together to get a longer value. But there are many other useful
things we can do with strings.

For example, if you want to know how long a string is, you can find
out using the \emph{function} \verb!len()!.\footnote{A function
in Python is a named sequence of instructions. We 
\emph{call} the function using its name in order to execute its instructions.
We can call a function without knowing what instructions it consists of, so
long as we know what it does.}
% {\em method}\footnote{A method is a named sequence of instructions. We 
% {\em call} the method using its name in order to execute its instructions.
% We can call a method without knowing what instructions it consists of, so
% long as we know what it does.}. 
% If  \verb!str! is the string, then \verb!str.length()! gives its 
% length. Do not forget the dot or the brackets: methods are always 
% prefixed with a dot and followed by brackets\footnote{Now you can see that
%   console() and readLine() are also methods.} (sometimes empty,
% sometimes not).
You could say, for example:

\begin{Verbatim}
    print('Please enter some text: ', end = '')
    word = input()
    length = len(word)
    print('The string ' + word + ' has ' + str(length) + ' characters.')
\end{Verbatim}

You can obtain a substring (or a ``slice'')
of a string as in the %\verb!substring! method.
following example:
If the variable \verb!s! has the string value `\verb!Silverdale!', then
\verb!s[0:6]! will give you the first six letters, i.e., the string
`\verb!Silver!'.  The first number in square brackets
after the string
says where you want the substring to begin, and the second number says
where you want it to end (to be precise, the second number is the
position of the first character that we \emph{don't} want in the
substring). \emph{Note that the initial character of the string
is treated as being at position~0, not position~1}.
Similarly, \verb!s[2:6]! will give you the string \verb!lver!.

If you leave out the second number, you get the rest of the string.
For example, \verb!s[6:]! would give you `\verb!dale!', i.e., the
tail end of the string beginning at character 6 ('d' is character 6,
not 7, because the 'S' character is character 0).

You can output substrings with \verb!print! or assign them to other
strings or combine them with the ``\texttt{+}'' operator.  For example:

\VerbatimInput[frame=single,label=Example]{src/s2Example2.py}

will output `soft ices'.

\subsection*{Exercise C}

Say what the output of the following program fragment would be:

\VerbatimInput[frame=single,label=Example]{src/s2Example3.py}

Then run the program and see if you were right. 


%%% Local Variables:
%%% mode: latex
%%% TeX-master: "main"
%%% End:


\newpage

\input{section3-If,else}
\newpage

\section{Loops and booleans}

How would you write a program to add up a series of numbers?
If you knew that there were, say, four numbers, you might write
this program:

\VerbatimInput[frame=single,label=Example]{src/s3Example.py}

But a similar program to add up 100 numbers would be very long.
More seriously, each program would need to be tailor-made for a particular
number of numbers.  It would be better if we could write one program
to handle \emph{any} series of numbers.  We need a \emph{loop}.

One way to create a loop is to use the keyword \texttt{while}.
For example:

\VerbatimInput[frame=single,label=Example]{src/s3Example2.py}

Similar to the \texttt{if} statement we saw earlier,
it is essential to indent the lines inside the loop.
This allows Python to see which lines are inside the loop and which
lines are ``after'' the loop.
% It is not essential to indent the lines inside the loop, but it makes the
% program easier to read and it is a good habit to get into.


Having initialised the variable \texttt{num} to zero, the program checks
whether the value of \texttt{num} is less than 100.  It is, so it
enters the loop.  Inside the loop, it adds 5 to the value of
\texttt{num} and then outputs this value (so the first thing that
appears on the screen is a \texttt{5}).  Then it goes back to the
  \texttt{while} and checks whether the value of \texttt{num} is less than
100.  The current value of \texttt{num} is 5, which is less than 100,
so it enters the loop again.  It adds 5 to \texttt{num}, so
\texttt{num} takes the value 10, and then outputs this value.  It goes back
to the \texttt{while}, checks whether 10 is less than 100 and enters
the loop again.  It carries on doing this with \texttt{num} getting
larger each time round the loop.  Eventually \texttt{num} has the
value 95. As 95 is still less than 100, it enters the loop again, adds 5 to
\texttt{num} to make it 100 and outputs this number.  Then it goes
back to the \texttt{while} and this time sees that \texttt{num} is not
less than 100.  So it stops looping, goes on to the line after the
end of the loop, and prints `Done looping!'.
The output of this program is the numbers 5,
10, 15, 20 and so on up to 95, 100, followed by `Done looping!'.

Note the use of %curly braces
indent to mark the start and end of the loop.
Each time round the loop the program does everything inside
the indented block.
% the curly braces.
When it decides not to execute the loop again, it jumps to the point
%beyond the closing brace.
beyond the indented block.

What would happen if the \texttt{while} line of this program was
\texttt{while num != 99:}?
The value of \texttt{num} would eventually reach 95.  The computer would
decide that 95 was not equal to 99 and would go round the loop again.
It would add 5 to \texttt{num}, making 100.
It would now decide that 100 was not equal to 99 and would go round the
loop again.  Next time \texttt{num} would have the value 105, then 110,
then 115 and so on.  The value of \texttt{num} would never be equal to 99
and the computer would carry on for ever.  This is an example of
an \emph{infinite loop}.

Note that the computer makes the test \emph{before} it enters the loop.
What would happen if the \texttt{while} line of this program was
\texttt{while num > 0:}?
\texttt{num} begins with the value zero and the computer would first
test whether this value was greater than zero.  Zero is not greater than
zero, so it would not enter the loop.  It would skip straight to the end
of the loop and finish, producing no output.

% It is not essential to indent the lines inside the loop, but it makes the
% program easier to read and it is a good habit to get into.

\subsection*{Exercise A}

Write a program that outputs the squares of all the numbers from 1 to
10, i.e., the output will be the numbers 1, 4, 9, 16 and so on up to 100.

\subsection{Booleans (True/False expressions)}

So far we have just used integer and string values.  But we can
have values of other types and, specifically, we can have
\emph{boolean} values.\footnote{The word ``boolean'' was coined in honour of an
Irish mathematician of the nineteenth century called \emph{George Boole}}
%In Groovy, these are variables of type \texttt{boolean}.
In Python, this type is called ``bool''. Its values are also called
``truth values'', and they are
%A variable of type \texttt{boolean} does not hold numbers;
%There are only two values for the type bool:
\texttt{True} and \texttt{False}.

\begin{Verbatim}
    positive = True
\end{Verbatim}

Note that we do not have quote marks around the word \texttt{True}.  
The word \texttt{True}, without quote marks, is not a string; it's the name of a
boolean value.  Contrast it with:

\begin{Verbatim}
    positive = 'True'
\end{Verbatim}

Here \texttt{positive} is assigned
the four-character string `\texttt{True}', not the truth value
\texttt{True}. (This is similar to the difference between e.g.\ the
number 42 and the string `42' that we saw earlier.)
%  By contrast, \texttt{positive}
% is a boolean variable and we cannot assign strings to it.
% It can hold only the values \texttt{true} or \texttt{false}.

You have already met boolean expressions.  They are also called conditional
expressions and they are the sort of expression you have
after \texttt{if} or \texttt{while} before the ``\texttt{:}''.
When you evaluate a boolean expression,
you get the value \texttt{True} or \texttt{False} as the result.

Consider the kind of integer assignment statement with which you are
now familiar:

\begin{Verbatim}
    num = count + 5
\end{Verbatim}

The expression on the right-hand side, the \texttt{count + 5}, is an
integer expression.  That is, when we evaluate it, we get an integer value
as the result.
%   And of course an integer value is exactly the right kind of
% thing to assign to an integer variable.

Now consider a similar-looking boolean assignment statement:

\begin{Verbatim}
    positive = num >= 0
\end{Verbatim}

The expression on the right-hand side, the \texttt{num >= 0}, is a
boolean expression.  That is, when we evaluate it, we get a boolean value
(True/False) as the result.
% And of course a boolean value is exactly the
% right kind of thing to assign to a boolean variable.
You can achieve
the same effect by the following more long-winded statement:

\begin{Verbatim}
    if num >= 0:
        positive = True
    else:
        positive = False
\end{Verbatim}

The variable \texttt{positive} now stores a simple fact about the value of
 \texttt{num} at this point in the program.  (The value of  \texttt{num} might
subsequently change, of course, but the value of \texttt{positive} will not
change with it.)  If, later in the program, we wish to test
the value of \texttt{positive}, we need only write

\begin{Verbatim}
    if positive:
\end{Verbatim}

You can write \texttt{if positive == True:} if you prefer, but the
\texttt{== True} 
is redundant.  \texttt{positive} itself is either True or False.  
%Once the computer has evaluated \texttt{positive} 
%(established whether it is true or false) there is nothing more to do.  
We can also write

\begin{Verbatim}
    if not positive:
\end{Verbatim}

%(pronounced \emph{if not positive})
that is exactly the same as \texttt{if positive == False}. If
\texttt{positive} is True, then \texttt{not positive} is False, and
vice-versa.

Boolean variables are often called \emph{flags}.  The idea is that
% , leaving aside subtleties such as half-mast, 
a flag has basically two states -- either it's flying or it isn't.

So far we have constructed simple boolean expressions using the operators
introduced in the last chapter -- \texttt{x == y}, \texttt{s >= t} and so on --
now augmented with negation~(\verb+not+).
We can make more complex boolean expressions by joining simple ones
with the operators \verb!and! and \verb!or!.  For example,
we can express ``if x is a non-negative odd number'' as
\verb+if x >= 0 and x % 2 == 1:+. We can express ``if the name begins
with an A or an E'' as \texttt{if name[0:1] == 'A' or
  name[0:1] == 'E':}.  The rules for evaluating \emph{and}
and \emph{or} are as follows:

\vspace{1em}
\begin{tabular}{|l|lcr||lcr|}
\hline
& left & \verb!and! & right & left & \verb+or+ & right\\
\hline
\hline
1 & True & \textbf{True} & True&True&\textbf{True}&True\\
2 & True & \textbf{False} & False&True&\textbf{True}&False\\
3 & False & \textbf{False} & True&False&\textbf{True}&True\\
4 & False & \textbf{False} & False&False&\textbf{False}&False\\
\hline
\end{tabular}
\vspace{1em}

Taking line 2 as an example, this says that, given that you have two simple
boolean expressions joined by \emph{and} and that the one on the left is
True while the one on the right is False, the whole thing is False.  If,
however, you had the same two simple expressions joined by \emph{or}, the
whole thing would be True. As you can see, \emph{and} is True if and
only if both sides are True, otherwise it's False; \emph{or} is False
if and only if both sides are False, otherwise it's True.

\subsection*{Exercise B}

Given that \texttt{x} has the value 5, \texttt{y} has the value 20,
and \texttt{s} has the value `\texttt{Birkbeck}', decide whether these
expressions are True or False:

\begin{verbatim}
  x == 5 and y == 10
  x < 0 or y > 15
  y % x == 0 and len(s) == 8
  s[1:3] == 'Bir' or x // y > 0
\end{verbatim}

\subsection{Back to loops}

Returning to the problem of adding up a series of numbers, have a
look at this program:

\VerbatimInput[frame=single,label=Example]{src/s3Example3.py}

If we want to input a series of numbers, how will the program know
when we have put them all in?  That is the tricky part, which accounts
for the added complexity of this program.

The variable
\texttt{finished} is being used to help us detect when there are no more
numbers. It is initialised to \texttt{False}.  When the computer detects
that the last number (``0'') is introduced, it will be set to \texttt{True}.
When \texttt{finished} is True, it means that we have finished reading in
the input. The \texttt{while} loop begins by testing whether \texttt{finished}
is true or not.  If \texttt{finished} is not true, there is some more input
to read and we enter the loop. If \texttt{finished} is true, there are
no more numbers to input and we skip to the end of the loop.

The variable \texttt{total} is initialised to zero.
Each time round the loop, the computer reads a new value into \texttt{num}
and adds it to \texttt{total}. So \texttt{total} holds the total of all
the values input so far.

Actually the program only adds  \texttt{num} to \texttt{total} if a (non-zero) 
number has been entered. 
% The user will signal that there are no more numbers by keying in a special
% character.  (On Unix this is a Control-D; on PCs it is usually a Control-Z.
% If this is gobbledygook to you, just imagine that the user strikes a special
% key on the keyboard.)  
If the user enters something other than an integer 
-- perhaps a letter or a punctuation mark -- 
then the parsing of the input will fail 
and the program will stop, giving an error message.
% If the program is expecting an integer
% and instead receives something like ``abc'' or ``W'' or ``**!'', the input will fail.
% We can test whether the input has finished with the line \texttt{if (sc.hasNext())}.
% If this is false, then there is no more input available and we set \texttt{finished} to \texttt{true}
% in order to terminate the loop.

A real-life program ought not to respond to a user error by aborting
with a terse error message, though regrettably many of them do.
However, dealing with this problem properly would make this little program
more complicated than I want it to be at this stage.

\textbf{You have to take some care in deciding whether a line should go in the
loop or outside it}.  This program, for example, is only slightly different
from the one above but it will perform differently:

\VerbatimInput[frame=single,label=Example]{src/s3Example4.py}

It resets \texttt{total} to zero \emph{each time round the loop.}  So
\texttt{total} gets set to zero,
has a value added to it, then gets set to zero, has another value added to
it, then gets set to zero again, and so on.  When the program finishes,
\texttt{total} does not hold the total of all the numbers, just the value
zero.

\pagebreak

Here is another variation:

\VerbatimInput[frame=single,label=Example]{src/s3Example5.py}

This one has the \texttt{print} line inside the loop, so it outputs the value
of \texttt{total} each time round the loop.  If you keyed in the numbers
4, 5, 6, 7 and 8, then, instead of just getting the total (30) as the output,
you would get 4, 9, 15, 22 and then 30.

\subsection*{Exercise C}

Write a program that reads a series of numbers, ending with 0, and then tells you how
many numbers you have keyed in (other than the last 0).  For example, if you keyed in the
numbers~5, -10, 50, 22, -945, 12, 0 it would output `You have entered 6
numbers.'.


% \subsection{System libraries}

% The purpose of libraries is to provide the programmer with extensions
% to the language which can be pulled in, as and when needed. 
% Some libraries, known as the standard libraries, are provided with every
% Groovy implementation.  Other libraries can be created for special purposes.

% One team of programmers might produce a library of routines for another
% team of programmers to use.  When teams of programmers divide up the
% work on a large program in this way, it is all too easy
% for programmers in team A to choose a name for some item in their library
% and for programmers in team B to choose
% the same name for some quite unrelated item elsewhere in the program.
% This can be a serious nuisance.  To help contain the problem, Groovy
% provides \emph{packages}.  Team A declares one namespace and team B
% declares another.  Now it doesn't matter if they accidentally choose the
% same name.  Team B can still use names from A's package, but they have to
% tell the compiler specifically which names they are going to use.




%%% Local Variables:
%%% mode: latex
%%% TeX-master: "main"
%%% End:


\newpage

\input{extra}
\newpage

\section{Summary of the language features mentioned in this introduction}

%\begin{tabular}{@{\ttfamily}l l}
\begin{tabular}{p{6cm}p{7.2cm}}
\texttt{\#} & Introduces a comment\\
% int i & Defines an integer variable called \texttt{i}\\
% int x, y & Defines integer variables called \texttt{x} and \texttt{y}\\
% boolean b & Defines a boolean variable called \texttt{b}\\
% String s & Defines a string variable called \texttt{s}\\
\texttt{num = 99} & Assigns the integer value 99 to a variable called \texttt{num}\\
\texttt{word = input()} & Takes a string from the input and puts it
into the variable \texttt{word} \\
\texttt{n = int(word)} & Parses the number in the variable
\texttt{word} and puts its integer value into the variable \texttt{n}
\\
\texttt{word = str(n)} & Converts the number in the variable
\texttt{n} to its string representation and puts it into the variable \texttt{word} \\
\texttt{print(word)} & Outputs the value of \texttt{word}\\
\texttt{print(word, end = '')} & The same, but without the return-of-line at the end\\
\texttt{print('The answer is ' + x)} & Outputs `The answer is ' followed by the string \texttt{x}\\
%x = 99 & Assigns the value 99 to \texttt{x}\\
\texttt{+}, \texttt{-}, \texttt{*}, \texttt{//}, \texttt{\%} & Arithmetic operators.  \texttt{*}, \texttt{//} and \texttt{\%} take precedence over \texttt{+} and \texttt{-}.\\
\texttt{len(s)} & Gives length of string \texttt{s}\\
\texttt{s[x:y]} & Gives substring of \texttt{s} starting with the \\
                  & character at position \texttt{x} and ending with \\
                  & the last character before position \texttt{y}.\\
                  & (The first character of the string is at \\ 
                  & position zero.)\\
\\
\verb!if A:!    & If \texttt{A} is True,\\
\verb!    B!    & \quad do \texttt{B},\\
\verb!else:!    & else\\
\verb!    C!    & \quad do \texttt{C}.\\
\\
\texttt{x == y}          & Tests whether \texttt{x} is equal to
\texttt{y}.  Note the double ``\texttt{=}''.\\
\verb+!=+, \verb+>+, \verb+<+, \verb+>=+, \verb+<=+ 
                  & Not equal to, greater than, less than, etc.\\
% \{ & Used to bracket together separate \\ 
% \} & statements to make a block\\
\\
\verb!while A:! & While \texttt{A} is True, do \texttt{B}. \texttt{B} can be a block\\
\verb!    B!   &     with several statements.\\
\\
\verb!True!, \verb!False! & The boolean values \emph{True} and \emph{False}.\\
\verb+not+, \verb+and+, \verb+or+  & The boolean operators \emph{not}, \emph{and} and \emph{or}.
\end{tabular}
\vspace*{5pt}


%%% Local Variables:
%%% TeX-master: "cheatmain"
%%% End:


\newpage

% \input{groovyOnIDEs}
% \newpage

\section{Obtaining, installing and running Python on a PC}
\label{sec:obta-inst-runn}
 
You can download a free copy of Python 
from the web:

\begin{verbatim}
https://www.python.org/downloads/
\end{verbatim}

It is important that you download version 3.\ldots, which is what we
use in this introduction.
Python is available for many
operating systems, including Windows, Linux, and Mac.%  If you do not have
% Java installed in your computer, you can install it from the web:

% \begin{verbatim}
%     http://www.java.com/en/download/index.jsp
% \end{verbatim}

Python will be used intensively in several of the MSc courses 
at Birkbeck, so it is a good idea to install it if you do not have it
yet.
If you do not know whether you have Python installed in your
system, you can open a command prompt\footnote{You can find it in Windows in
  ``Accessories''} and type
% \footnote{In some systems, you can type
%   ``-v'' instead of ``-version''.}
\verb+python3 --version+.
If Python is installed, the result should be something similar to this:

\begin{verbatim}
    > python3 --version
    Python 3.4.3
\end{verbatim}



\subsection*{Running Python programs}

Python is run from the command line.

You first type your program in a text editor. Notepad (in Windows) or
gedit (in Linux) are good options, but any simple editor will
do. Note: If you use a word-processor (like Microsoft Word or
LibreOffice Writer), make sure you save the file as text.

Give the filename a \texttt{.py} extension.

Open the command prompt. 
Go to the folder where you have saved your Python
program using the command ``cd'' (change directory). (You do not have to
do this but it is easier if you do, otherwise you would have to type the full
path for your source code file to run it.)

Suppose your program is in a file called
\texttt{Myprog.py} in the current folder.
You can run your program by typing

\begin{Verbatim}
    python3  Myprog.py
\end{Verbatim}

If the %compilation is successful,
Python interpreter accepts your program,
you will see the result of your
program on the screen.
%
If there are errors, you will get error messages and you need
to go back to the text editor, correct the program and save it again,
then try to run your program again.



\subsection*{If you cannot install Python\ldots}
\label{sec:if-you-cannot}

If you have problems installing Python, you may find this webpage useful:

\begin{verbatim}
   https://repl.it/languages/python3
\end{verbatim}

The page provides you with an editor on the left and a console on the
right. It will allow you to test your programs online. It may be quite
slow compared to using a local machine, but it may help if you find
problems installing Python. 

Additionally, if you have any problem installing or running Python,
feel free to write to Dr.\ Carsten Fuhs
(carsten@dcs.bbk.ac.uk). We will come back to you as soon as possible.

%%% Local Variables:
%%% TeX-master: "primer"
%%% End:



\end{document}  

