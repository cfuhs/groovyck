% TODO: add fork-join

% TODO: add example of use of BlockingQueue
%       e.g. ExecutorImpl with and without BlockingQueue

% TODO: Add CSP using java.util.concurrent

\section{More concurrency}
\label{sec:more-concurrency}

Last unit we covered the basics of concurrent programming. We used the
low-level elements that we can use in Java to create programs where
different tasks are being executed at the same time: threads and
locks (and in particular, the \verb+synchronized+ keyword). 

Although any kind of concurrent program can be created using these
elements, in reality they are too basic for big programs and/or
advanced tasks. It is very easy to make mistakes that will make a
program run slower or ---worse--- block completely by using threads
and locks directly. In the current context, in which multi-core
computers have become the norm and applications need to be able to
benefit from this inherent parallelism provided by the hardware,
higher-order structures are needed to create reliable concurrent
programs. 

Java 5 introduced several features that aimed at covering this
gap. The two most imporant, executors and concurrent collections, are
explained in the following sections.

% TODO: do fork/join

\section{Executors}
\label{sec:executors}

An \verb+Executor+ is an object that executes tasks. These tasks are
defined by other objects. 
%
This is similar to what we have been doing with threads. We have
created threads that ran tasks that were defined by a \verb+Runnable+
object, with a \verb+run()+ method. 

The problem with this approach is that the creation of the thread, its
lifetime, and the task it performs are closely linked. There is no
separation between thread and task, between container and
content. The thread lives for as long as the task is running, and as
soon as the task finishes the thread dies. This has several
disadvantages: 

\begin{itemize}
\item Thread management is done manually and, therefore, is an
  error-prone process. The programmer must take care not only of the
  tasks that need to be performed (sometimes called the \emph{business
  logic}), but also of creating 
  the threads to run them, starting them, managing their
  interruptions, etc. 
\item Thread creation takes time. In a highly concurrent application,
  creation and destruction (i.e. garbage collection) of threads can
  result in a performance penalty.
\item Thread are memory structures that use a non-trivial amount of
  memory. If threads are created manually one by one there is no
  easy and scalable way to measure the load of the application and to
  manage  
  it carefully to prevent running out of resources.
\end{itemize}

In a large-scale application, it makes sense to separate thread
creation and management from the rest of the application. Executors
are objects that encapsulate these operations hiding them from the
rest of the program. 

\subsection{Three types of executors}
\label{sec:three-types-exec}

Executors are defined by three interfaces in the
\verb+java.util.concurrent+ package; the package also provides some
implementations for these interfaces. 

\begin{description}
\item[Executor: ] This is the basic executor. It just supports
  launching new tasks.
\item[ExecutorService: ] This is an extension of the former one, and
  adds features that help manage the lifecyle of the tasks to be run
  and also the executor itself.
\item[ScheduledExecutorService: ] An extension of the former one, this
  interface adds methods to execute tasks in the future and at
  scheduled times. 
\end{description}

As usual, variables that represent executor objects should be declared
as one of these interfaces and not as one of the classes that
implement them, for example: 

\begin{verbatim}
    Executor executor = new ThreadPoolExecutor();
\end{verbatim}

\subsubsection{Executor}
\label{sec:executor}

The \verb+Executor+ interface defines only one method:
\verb+execute(Runnable)+. This interface is rarely used compared to
the other two, but it was designed to be an easy replacement when
fixing legacy code that had been implemented using pure threads and
locks. If \verb+r+ is a \verb+Runnable+ object and \verb+e+ is an
\verb+Executor+ object, than we can replace:

\begin{verbatim}
    Thread t = new Thread(r);
    t.start();
\end{verbatim}

with 

\begin{verbatim}
    e.execute(r);
\end{verbatim}

Simple, cleaner, and ---more important--- it opens the door to using
one of the other two more powerful executors. Additionally, depending
on the class that implements the \verb+Executor+, it may be able to
reuse threads from a thread pool or put the runnable tasks in a queue
if the system is busy; these things are not possible when using plain
threads. Thread pools and queues of tasks are two constructs that
are very convenient to ease the management of threads and solve some
of the problems of using plain threads as explained above. More on
that below. 

\subsubsection{Executor Services}
\label{sec:executor-services}

The \verb+ExecutorService+ interface extends the plain \verb+Executor+
with the \verb+submit(...)+ method, which is similar to
\verb+execute(Runnable)+ but more versatile because it accepts both
\verb+Runnable+ and \verb+Callable+ tasks (i.e.~described by
objects). Callable objects are similar to runnable objects but they
can return a value when they finish their computation. 

The \verb+submit(...)+ method returns a \verb+Future+, which represents
the result of the computation; it is called \emph{future} because it
is only known after some indeterminate time has passed since the service
was called. The \verb+Future+ object can be used to retrieve the
\verb+Callable+ return value, 
and to manage the status of both \verb+Callable+ and
\verb+Runnable+ tasks. 

% TODO: more on Futures

The interface \verb+ScheduledExecutorService+ adds methods to schedule
the execution of \verb+Runnable+ and \verb+Callable+ tasks at specific
points in time: \verb+schedule(...)+ executes the tasks after some
time delay has passed, while \verb+scheduleAtFixedRate(...)+ and
\verb+scheduleWithFixedDelay+ execute specified tasks repeatedly at 
defined intervals. 


\subsection{Thread pools}
\label{sec:thread-pools}

The classes in \verb+java.util.concurrent+ that implement the executor
interfaces described in the former section use in most cases a
\emph{thread pool}. A thread pool is a collection or worker threads
to which tasks (defined by \verb+Runnable+ and \verb+Callable+
objects) can be asigned. This kind of thread exists separately from
the tasks it executes and is often used to execute multiple tasks at
different times. 

Using thread pools reduces the overhead due to thread creation. Thread
objects use a non-trivial amount of memory; in large projects
allocation and disposal of a lot of thread objects can result in a
noticeable performance cost. Having a pool of threads already created
that can be reused results in better performance. 

One common type of thread pool is the fixed thread pool. A pool of
this type has a specific number of threads running at all times; if
for any reason a thread terminates while it is still in use, another
thread will be created to replace it. If there are more tasks to be
executed than threads, tasks will be added to an internal queue. As
soon as thread finishes its task and becomes available, new tasks are
retrieved from the queue and assigned to the thread. In the uncommon
situation in which there are more tasks than threads ---or no tasks at
all--- then the ``idle'' threads will \verb+wait()+.

Having a fixed thread pool is a way of managing the load of your
application. By capping the number of parallel tasks that the system
will be running at any given time (i.e.~by the size of the pool), 
the programmer makes sure that the
system will not crash unexpectdely under stress. Without a limit on
the number of threads to be created, the application would 
create threads until it ran out of memory, and then it would crash,
which is bad. By limiting the number of tasks that can be executing at
the same time, and putting the new incoming tasks in a queue until
there is a thread available for them, the application \emph{degrades
  gracefully}; this means that it simply becomes gradually more slow
in its response time, instead of becoming unresponsive suddenly and
unexpectedly, and this is good.

\section{Concurrent Collections}
\label{sec:conc-coll}

We have already used several collections from the Java Collections
Framework (JCF): \verb+List+, \verb+Queue+, \verb+Stack+, \verb+Map+,
\verb+Set+\ldots However, all these interfaces are not defined to be
thread-safe, and therefore the classes that implement them (such as
\verb+ArrayList+ and \verb+HashSet+) are not expected to be. This is OK
for single-threaded applications, but in concurrent applications we
need to make sure that our data structures remain consistent. 

We can do this by using threads and locks (i.e.~\verb+synchronized+),
but the \verb+java.util.concurrent+ package also includes some nice
additions to the JCF. The two more interesting ones are described
below: 

\begin{description}
\item[BlockingQueue: ] This is like a normal queue with two
  exceptions. When you try to insert an element into an already full
  queue, it will wait until there is space instead of failing; it can
  timeout if the queue remains full for too long. A similar behaviour
  applies when you try to retrieve an element from an empty queue: it
  will wait until there is something to retrieve\ldots or timeout.
\item[ConcurrentMap: ] This is an extension of \verb+Map+ that adds
  atomicity to the usual put-replace-delete operations of maps. These
  operations remove or replace a key-value pair only if the key is
  present, or add a key-value pair only if the key is absent. Making
  these operations atomic helps avoid external synchronization.
\end{description}



%%% Local Variables:
%%% mode: latex
%%% TeX-master: "main"
%%% End:
