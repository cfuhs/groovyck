\section{More concurrency}
\label{sec:more-concurrency}

Last unit we covered the basics of concurrent programming. We used the
low-level elements that we can use in Java to create program where
different tasks are being executed at the same time: threads and
locks (and in particular, the \verb+synchronized+ keyword). 

Although any kind of concurrent program can be created with these
elements, in reality they are too basic for big programs and/or
advanced tasks. It is very easy to make mistakes that will make a
program run slower or ---worse--- block completely by using threads
and locks directly. In the current context, in which multi-core
computers have become the norm and applications need to be able to
benefit from this inherent parallelism provided by the hardware,
higher-order structures are needed to create reliable concurrent
programs. 

Java 5 introduced several features that aimed at covering this
gap. The two most imporant, executors and concurrent collections, are
explained in the following sections.

% TODO: do fork/join

\section{Executors}
\label{sec:executors}

