
\section{Exception handling}
\label{sec:exception-handling}

If you try to parse as an integer something that is not an integer
number, you get a \verb+NumberFormatException+. If you try to call the
object of an object but the object is \verb+null+, you get a
\verb+NullPointerException+. But what are exceptions? And is there a
way to handle them so that they do not always crash a program? 

Exceptions represent exceptional circumstances that should never occur
in the normal (some would say \emph{ideal}) execution of a
program. When one of these exceptional circumstances ---like trying to
follow a \verb+null+ pointer--- happens, the Java Virtual Machine
throws an exception to indicate it. 

An exception interrupts the normal execution of the method where it
happened. It breaks the method where it was happening and moves up to
the calling method, where it may be caught. If it is not caught in the
calling method, it will go up again, and again, until it is caught. If
it is not caught, not even in the original \verb+main+ method, it will
be caught by the Java Virtual Machine and it will be printed on the
standard output as you have seen several times already. 

\begin{verbatim}
    Exception in thread "main" java.lang.NullPointerException
      at LinkedListNode.addNode(LinkedListNode.java:49)
      at LinkedList.addNode(LinkedList.java:30)
	at HospitalManager.launch(HospitalManager.java:71)
	at HospitalManager.main(HospitalManager.java:13)
\end{verbatim}

\subsubsection*{Reading a stack trace}

An exception contains information about the state of the Java Virtual
Machine when it was thrown. In particular, it contains the whole stack
of methods calls. This is very useful because it helps programmers to
find what went wrong (and maybe why). 

The most useful line of the stack trace is usually the first one,
because it says exactly where the exceptional situation happened. For
example, in the stack trace above we can observe that a
\verb+NullPointerException+ was thrown at line 49 of the code of class 
\verb+LinkedListNode+. With this information a programmer can look at
that specific line and try to understand why a null pointer was
accessed when this should not happen. 

If we want more information we just need to follow the stack trace
line by line. We can see that the program started at the main method
of class \verb+HospitalManager+, that the \verb+launch()+ method was
called on like 13, and that this method called the method
\verb+addNode()+ in class \verb+LinkedList+ (on line 71), which in
turn called the method \verb+addNode()+ of class \verb+LinkedListNode+
(on line 30), where the exception was thrown. 

There is no limit to the length of a stack trace. Then can have as
many steps as the size of your stack, which is several thousands of
calls deep (depending on the number of local variables per method
call). 

Sometimes the stack trace goes into code that has not been written by
us, as in the following example (comments ommitted for brevity): 

\begin{verbatim}
01    public class ExceptionThrower {
02       public static void main(String[] args) {
03          ExceptionThrower et = new ExceptionThrower();
04          et.launch();
05       }
06       private void launch() {
07          System.out.print("Write a number: ");
08          int n = getNumber();
09          String evenness = (n % 2 == 0) ? "even" : "odd";
10          System.out.println("You entered " + n + ", an " + evenness + " number."$
11       }
12       private int getNumber() {
13          String str = System.console().readLine();
14          int result = Integer.parseInt(str);
15          return result;
16       }
17    }
\end{verbatim}

If you enter numbers like 2 or 19, all will work well. However, if you
enter something like ``three'' you will get a detailed stack trace to
blame you for your incompetence typing integer numbers: 

\begin{verbatim}
    Exception in thread "main" java.lang.NumberFormatException: For input string: "three"
	at java.lang.NumberFormatException.forInputString(NumberFormatException.java:48)
	at java.lang.Integer.parseInt(Integer.java:449)
	at java.lang.Integer.parseInt(Integer.java:499)
	at ExceptionThrower.getNumber(ExceptionThrower.java:14)
	at ExceptionThrower.launch(ExceptionThrower.java:8)
	at ExceptionThrower.main(ExceptionThrower.java:4)
\end{verbatim}

We can see that the program started in the \verb+main+ method, that is called
the method \verb+launch()+, that this method called the method
\verb+getNumber()+, and then this method called
\verb+Integer.parseInt()+. From that point on, the stack trace shows
method calls happening inside classes of the basic Java Library, like
\verb+Integer+ and \verb+NumberFormatException+, but you can see that
the information is exactly the same: method calls and line numbers. If
you have access to the source code of those external classes, you will
see the line numbers and will be able to trace the stack of method
calls in detail. If you work with external libraries from which you do
not have the source code, you will only see the names of the methods
but not the line number. 

The example allows us to observer the ``interrupting'' nature of
exceptions. The \verb+NumberFormatException+ interrupted the execution
of method \verb+getNumber()+ (line 15 was not executed), then
interrupted the caller method \verb+launch()+ (lines 09--10 were never
executed), and then interrupted the \verb+main+ method. Since the
exception was never \emph{caught}, it went all the way up to the
\verb+main+ method and then was shown by the Java Virtual Machine. 

\subsection{Catching exceptions}
\label{sec:catching-exception}

Exceptions are handles by using try/catch constructs: you ``try'' some
code where exceptions may be thrown; if exceptions are thrown, you
``catch'' them and do something about them. See the following example: 

\begin{verbatim}
12    private int getNumber() {
13       int result = 0; // default
14       try { 
15          String str = System.console().readLine();
16          result = Integer.parseInt(str);
17          System.out.println("You entered " + result + ".");
18       } catch (NumberFormatException ex) {
19          System.out.println("You entered something that is not an integer number.");
20       }
21       return result;
22    }
\end{verbatim}

If a \verb+NumberFormatException+ is thrown at line 16, the normal
execution of the method is interrupted. Program execution jumps out of
the current \verb+try+ clause (i.e.~until the next closing curly
bracket). It is caught there ---there is a \verb+catch+ statement for
that type of exception---, and the code inside the \verb+catch+ clause is 
executed. Then the execution proceeds normally. 

If the exception jump up to the end of the scope and is not caught, it
will jump up to the end of the next scope. If it is still not caught,
it will jump up to the end of the next scope and so on. This usually
means that the exception is moving up the method stack from calling
method to calling method until it is caught. If it is never caught, it
will crash your program and appear on screen. 

In a way, you can think
of this as a huge \verb+catch+ statement just out of your main method
that only prints the stack trace.

\begin{verbatim}
    // WARNING: this is only a metaphore, not real code
    try {
        mainMethod();
    } catch (AnyException ex) {
        ex.printStackTrace().
        System.exit(0); // Stop the Java Virtual Machine
    }
\end{verbatim}

The method \verb+printStackTrace()+ is very useful, and is almost
always used in \verb+catch+ statements because it really helps in
debugging your program when a exception is thrown in an unexpected
place. The method prints the name of every single method that had been
called when the exception was thrown, including the line that was
being executed (if the source code for the class is known). 

A \verb+try+ clause can have many \verb+catch+ clauses associated to
it, as in the following example: 

\begin{verbatim}
12    private int getNumber() {
13       int result = 0; // default
14       try {
15           System.out.print("Enter a number with at least 3 digits: ");
16           String str = System.console().readLine();
17           result = Integer.parseInt(str);
18           System.out.println("You entered " + result + ".");
19           char thirdDigit = str.charAt(2);
20           System.out.println("The third digit is " + thirdDigit + ".");
21       } catch (NumberFormatException ex) {
22           System.out.println("ERROR 1: You entered something that is not an integer.");
23       } catch (IndexOutOfBoundsException ex) {
24           System.out.println("ERROR 2: You entered integer with less than three digits.");
25       }
26       return result;
27     }
\end{verbatim}

If a \verb+NumberFormatException+ is thrown on line~17, the execution
of the program will jump to line~21, where it will be caught. After
that \verb+catch+ clause is executed, the program will continue at the
end of the try/catch construct at line~26. If a
\verb+IndexOutOfBoundsException+ is thrown at line~19, the execution
of the program will jump to line~21, and then to line~23, where it
will be caught. After that \verb+catch+ clause is executed, the
program will continue at the end of the try/catch construct at
line~26.  
Once an exception is caught, it does not go ``up'' anymore. It
disappears. 


A \verb+try+ clause can also have a \verb+finally+ clause, which can
used for cleaning up resources. We will see more of this when we learn
about input/output. 

\subsection{Which methods throw exceptions?}
\label{sec:which-methods-throw}

The exceptions thrown by a method should be specified in the
documentation, in its JavaDoc comments. This is done by means of
\verb+@throws+. This is as important as describing the arguments for
the method or the return type. (Check the methods in the core Java
Library: they all document the exceptions they can throw). 

\begin{verbatim}
    /**
     * Gets a number as introduced by the user on the console, which
     * must be at least three digits long. Otherwise, an 
     * IndexOutOfBounds will be thrown.
     *
     * @return the number introduced by the user
     *
     * @throws NumberFormatException if the user types a not-number
     * @throws IndexOutOfBoundsException if the user types less than three digits
     */
\end{verbatim}

Additionally, many exceptions must be documented in the code
too. These are called \emph{checked exceptions}. 

\subsection{Checked and runtime exceptions}
\label{sec:check-runt-except}

All exceptions in Java extend class \verb+Exception+, in package
\verb+java.lang+. Many exceptions, including all you have met so far,
are in the same package. Classes from this package are automatically
imported, so you do not need to do import them explicitly. 

A subclass of \verb+Exception+ is \verb+RuntimeException+. Descendants
of the latter are called (unsurprisingly) \emph{runtime exceptions},
while all other descendants of \verb+Exception+ are called
\emph{checked exceptions}. The difference between these two types is
that checked exceptions \emph{must} be either explicitly declared or
explicitly caught. 

If a checked exception is thrown in some line of a method it must be
caught (by using a try/catch construct). If it is not caught, it must
be declared as (possibly) thrown up, as in the following example:

\begin{verbatim}
    /**
     * ...
     * @throws SomeCheckedException when that exceptional situation happens
     */
    public void doggyMethod() throws SomeCheckedException {
        // do things here that throw some checked exception
    }
\end{verbatim}

By declaring that the method throws \verb+SomeCheckedException+, the
programmer is telling the compiler to check that any method calling
\verb+doggyMethod+ is catching the exception\ldots or throwing it up
once again. If a checked exception can be thrown in a line, the line
must be inside a \verb+try+ block or the exception must be declared as
thrown. Otherwise, the compiler will complain. 

This does not happen with \verb+RuntimeException+. The reason is that
they are so common that it would be a lot of work to always declare
them or catch them.


\subsection{Mishandling exceptions}
\label{sec:mish-except}

There are two common mistakes when handling exceptions, and both of
them are related to laziness. As laziness is an essential part of
being human (or at least very common), it is important to be warned to
avoid the same mistakes that thousands of programmers have done in the
past (sometimes with terrible consequences). 

The first common mistake is not being specific enough when catching
exceptions. When you code throws three or four different exceptions,
it is quite tempting to use a quick \verb+catch(Exception)+ that will
work for all of them. This is usually a bad idea in the long run
because different exceptions happen in different circumstances and
should be dealt with in different ways. If you use the same code for
all of them, this may be confusing further down the line. As a rule of
thumb, it is better to catch different exceptions separately. 

The second ---and more dangerous--- mistake is not doing anything when
catching an exception. This is really bad. Once an exception is
caught, it disappears, it does not move up anymore. Unless you do
something about it, nobody will ever know that an exceptional
situation happened. For all your users know, your program may be
deleting money from their bank accounts without a hint that something
is wrong until it is too late. This is so important that I will write
it in big letter so that you do not ever forget. Repeat with me: 

\begin{center}  
{\large \bf Do not write empty \verb+catch+ blocks. Ever. }
\end{center}

Sometimes it is boring to write a \verb+catch+ block because you are
interested in the main algorithm and not the exceptional situations
along the borders of your problem. If you follow the Test Driven
Methodology we have seen in past weeks this will happen less often,
but it will happen, and you will be tempted many times to just write
an empty \verb+catch+ block and ``come later to write it
properly''. 

Believe me when I tell you that ``later'' never comes. If you cannot
think of anything reasonable for your \verb+catch+ block ---or do not
want to---, copy and paste either of the following two lines (or
both) in there and ``come later to write it properly'': 

\begin{verbatim}
    ex.printStackTrace();
    throw new RuntimeException(ex); // See more about this in next section
\end{verbatim}

If you leave empty \verb+catch+ blocks in your code you will face many
sleepless nights hunting missing exceptions and ghost bugs. You have
been warned. 

\subsection{Throwing your own exceptions}
\label{sec:throwing-your-own}

Creating your own exceptions is easy. Just create a class that extends
either \verb+Exception+ (for checked exceptions) or
\verb+RuntimeException+ (for runtime exceptions). It is rare that you
need to create additional exceptions, but in some cases it can be
useful to be more precise in your exception handling. 

Exceptions usually have three constructors: one without parameters,
one that takes a String (the message of the exception), and one that
takes another exception (so the new exception acts as a wrapper of the
old exception). 

Throwing an exception is also very easy. It is done by means of the
keyword \verb+throw+. 
(See the example below.)
Any exception can be thrown. 

\begin{verbatim}
    //...
    throw new SomeException();
    // ...
\end{verbatim}

This line creates a new exception (an object of class
\verb+RuntimeException+) and throws it. The exception can be created
in a different line, of course, and assigned to a variable. 



\paragraph{Re-throwing an exception}
\label{sec:re-throwing-an}

When an exception is caught, it disappears and does not go further up
in the method call stack. Unless we want it to happen, in which case
we can just either wrap it inside another exception and throw the
latter: 

\begin{verbatim}
    //...
    } catch(SomeException e) {
        throw new RuntimeException(e);
    }
    // ...
\end{verbatim}

or ---even simpler--- re-throw it directly (it is still an exception,
after all, so it can be thrown): 

\begin{verbatim}
    //...
    } catch(SomeException e) {
        throw e;
    }
    // ...
\end{verbatim}

The first option is quite better because it provides additional
information. I would not recommend to use the latter \emph{in any
  case}. It is only included here to illustrate the
relationship between exceptions and the \verb+throws+ keyword.

% \subsection{Making exceptions safe}
% \label{sec:making-except-safe}

% TODO: talk about Abraham's guarantees?

\subsection{Afterthoughts on exception handling}
\label{sec:afterth-except-handl}

\subsubsection{Should I create checked or runtime exceptions?}
\label{sec:should-i-create}

This is a difficult question, and there is not a clear answer for
it. Personally, I would recommend to make all new exceptions
\verb+RuntimeException+. 

Runtime exceptions will make your code easier to write
and not more difficult to read. They will go up your stack when thrown
and crash your program, which provides a good incentive to fix
the bug (probably by catching them at the right place). 
Not having checked and runtime exceptions (only the latter) seems
to be the route taken by more modern languages, including some very
popular ones like Python.

\subsubsection{Throw throws throwables}
\label{sec:throw-throws-throw}

The keyword \verb+throw+ can be used to throw other classes, not just
exceptions. In particular, any descendant of class \verb+Throwable+
can be thrown (and caught). 

There are two types of throwables in Java: \verb+Exception+ and
\verb+Error+. We already know exceptions. Errors are used to indicate
serious problems, like the \verb+StackOverflowError+ that is thrown
when the program runs out of stack space (e.g.~due to a buggy infinite
recursive call). It is not expected from normal applications to catch
errors, but it is possible. 

% \subsubsection{Exception handling error-checking}
% \label{sec:exception-handling-1}

% TODO: leave this section for next year



% Exercises: 
%  - Sequence flow with or without exception
%  - Read numbers. If not a number, show error message and start
%    again.



%%% Local Variables:
%%% mode: latex
%%% TeX-master: "d15"
%%% End:
