
\documentclass{article}
\usepackage[margin=2cm]{geometry}
\begin{document}

\section*{Learning goals}
\label{sec:learning-goals}

Before the next day, you should have achieved the following learning
goals: 

\begin{itemize}
\item Understand how to use interfaces in Java, and use them in your
  programs. 
\item Understand how stacks, queues, and trees work. 
\item Strengthen your understanding of pointers, and how they are used
  in dynamic data structures. 
\end{itemize}

You should be able to finish most of non-star exercises in the lab. 
Remember that star exercises are more difficult. 
\textbf{Do not try star-exercises unless the other ones are clear to
  you}.  

\section{Supermarket queue}
\label{sec:queues}

Use the interface \verb+PersonQueue+ that represents a queue of
people waiting at the supermarket and then implement it. You can use
the definition and the implementations of \verb+StringStack+ as an
example. You can reuse any version of class \verb+Person+ from past
days. For example, depending on your implementation, you may want to
use a version of \verb+Person+ with or without internal pointers. 

\begin{verbatim}
    public interface PersonQueue {
        /**
         * Adds another person to the queue.
         */
        void insert(Person person);

        /**
         * Removes a person from the queue.
         */ 
         Person retrieve();
    }
\end{verbatim}

Then create a class \verb+Supermarket+ that has two methods:
\verb+addPerson(Person)+ and \verb+servePerson()+. These methods must
call the appropriate methods of \verb+PersonQueue+. 

\section{Supermarket queue revisited (*)}
\label{sec:superm-queue-revis}

Implement your interface queue in a different way. Then check that it
works exactly the same without changing either the interface or your
class \verb+Supermarket+. 

\section{Foreign people, different queues (*)}
\label{sec:fore-people-diff}

Get a queue implementation from one of your colleagues. Use it and
check that it works exactly the same without changing either the
interface or your class \verb+Supermarket+. 

If it does not work, why is this?

\section{Unfair queue (*)}
\label{sec:unfair-queue-}

\subsection{Simple}
\label{sec:simple}

Implement the interface queue in a way that the person at the end
(i.e.~the person that is retrieved next time the method
\verb+retrieve()+ is called) is always the oldest person. 

\subsection{Clustered}
\label{sec:clustered}

Implement the interface queue in a way that the next person retrieved
is always a person over 65, if there is any in the queue; if not, it
must be a person over 18, if there is any in the queue. Inside each
age category, the behaviour of the queue is typical FIFO: first in,
first out. 

These two queues are examples of \emph{priority queues}, and are not
strictly FIFO: old people will always come out of the queue than
younger people, even if the youngsters came to the queue
first. Priority queues are more difficult to implement, but they are
also important in computing. For example, your hard disk uses a
priority queue to decide where to move next: if the disk's head is at
position 555 and the queue of requests is

$$4, 99, 234, 500, 101, 43, 881, 77$$

your disk may decide to move to position
500 to reduce movement, time, and energy consumption.



\end{document}