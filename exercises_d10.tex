\documentclass{article}
\usepackage[margin=2cm]{geometry}
\usepackage[dvips]{graphicx}
\begin{document}

\section*{Learning goals}
\label{sec:learning-goals}

Before the next day, you should have achieved the following learning
goals: 

\begin{itemize}
\item Strengthen your knowledge of JUnit4, practicing the use of the
  new annotations.
\item Understand the Test-Driven Development methodology. 
\item Understand the use of \verb+final+.
\item Understand how JavaDoc is used.
\end{itemize}

You should be able to finish most of non-star exercises in the lab. 
Remember that star exercises are more difficult. 
\textbf{Do not try star-exercises unless the other ones are clear to
  you}.  

\section{Final is final}
\label{sec:final-final}

Look at the following code. Is everything OK? How many errors can you spot?

\begin{verbatim}
    public class Person {
        private final String name;

        public Person(String name) {
            this.name = name;
        }

        public String getName()
            return this.name;
        }

        public void setName(String name) {
            this.name = name;
        }
    }
\end{verbatim}

Once you think you have found all the errors, try to compile the code
and see if your intuition was right. 

\section{Practicing TDD: }
\label{sec:jj}

At every stage, you must follow the following steps: 

\begin{enumerate}
\item Generate the tests for the functionality required for that
  stage. Check that the new tests fail while the old ones pass.
\item Generate the minimum code that passes all the new tests.
\item Refactor the code to make it clearer, if needed.
\item Document the
  code if it has not been done yet. Update the JacaDoc documentation
  (on a separate \verb+www+ folder). 
\end{enumerate}

Bank
  makeDeposit
  takeMoney
  (has map: name -> amount)

Client
  getMoney

----
  
Library(String name)
  takeBook(String title)
  returnBook(Book book)
  addBook(String title, String author)
  getBookCount()
  getBookBorrowedCount()
  addReader(Person p)
  getReaderCount()
  getName()

Person(String name)
  getBooks() 
  getName()
  getLibrary()
  register(Library)

Book(String title, String author)
  isTaken()
  setTaken(boolean)
  getTitle()
  getAuthor()

\end{document}