\documentclass{article}
\usepackage[margin=2cm]{geometry}
\usepackage[dvips]{graphicx}
\begin{document}

% TODO: re-write or even redesign the exercise with the library, some
% students find it confusing.

\section*{Learning goals}
\label{sec:learning-goals}

Before the next day, you should have achieved the following learning
goals: 

\begin{itemize}
\item Strengthen your knowledge of JUnit4, practicing the use of the
  new annotations.
\item Understand the Test-Driven Development methodology. 
%\item Understand the use of \verb+final+.
%\item Understand how JavaDoc is used.
\end{itemize}

You should be able to finish the first exercise and most of the second
one in the lab.

\section{Practicing TDD: }
\label{sec:jj}

Write a simple application for keeping track of the books in a
library. The functionality will be described incrementally by
stages. At every stage, you must follow the following steps:

\begin{enumerate}
\item Write (or update) the interface(s) for the new functionality
  required. 
\item Generate the tests for the functionality required at that
  stage. The tests may not even compile at this point.
\item Write the bare minimum code of the classes implementing the
  interfaces to make the test-class(es) compile. 
  Check that the new tests fail while the old ones pass. 
\item Write the minimum code in the methods of the implementing
  class(es) that makes all the new tests pass.  
\item Refactor the code to make it clearer, if needed. Your first
  implementation may not be as clear as possible. Think of the next
  programmer that will come after you: will they understand the code
  easily? Are variable names clear and descriptive? Can you simplify
  those \verb+for+ loops and/or those \verb+if...else+ branches?
\item Document the
  code if it has not been done yet. Update the JavaDoc documentation
  (on a separate \verb+www+ folder) using the command
  \verb+javadoc+. Use your browser to check that it works and is easy
  to read. 
\end{enumerate}

\subsection{}

Create a class for books. Books have authors and titles. The class
should implement getters for both author name and title, but these
must be set at construction time and never be modified
afterwards\footnote{An object whose fields cannot be changed after
  construction is called an \emph{immutable} object.}.

\subsection{}

Create a class for the users of the library. Users have a name and a
library-ID (an \verb+int+), 
both of which must be unique in a library. The name is
set at construction time, but the library-ID is not. 
The class must implement methods to get the
name of the person and their ID, 
% TODO: Maybe setId() is not needed. Implement this to double-check. 
and to set the latter. % // User.setId() 

\subsection{}

Expand the class you have just created to allow users to register with
a library. You will need two methods \verb+register(Library)+
and \verb+getLibrary()+. The former method is the way to obtain the
user-ID. 

\textbf{Important.} As you do not have a Library class yet, you will need a fake
Library object to test your method \verb+register(Library)+. This is called a
\emph{mock} object, and it is a common practice when writing testing
code because it allows the programmer to test one class at a time
---instead of testing several classes at the same time, which is more
complex and thus error-prone---. 
The mock library object does not need to do anything apart from
providing a name (so that your class can return it when you call
\verb+getLibrary()+ and an ID when you call \verb+getID()+. Because it
is a mock object and not the real one, it can return trivial values
(i.e.~the name can be always ``Library name'' and the ID can be
always~13).  

\subsection{}

Create a class Library for the library. Libraries have a name, set at
construction time. They also have a ``maximum number of books borrowed
by the same person'' policy (e.g.~max three books per user), 
which can be updated at any time. Of
course, they also have a method to get the maximum number of books to
be borrowed at any time (e.g.~\verb+getMaxBooksPerUser()+). 

\subsection{}

Add a method \verb+getId(String)+ that returns the ID of a person for
a given name in this library. If the person does not have an ID yet,
a new unique ID is created and returned. 

Any subsequent calls to this
method with the same name argument should return the same ID. 

\subsection{}

Expand your library class a bit further. Add three new methods: 

\begin{description}
\item[addBook(String title, String author)] Adds a new book to the
  list of books in this library.
\item[takeBook(String title)] If the book is not taken, marks the book
  as taken and returns it. If the book is taken, null is returned.
\item[returnBook(Book book)] Marks the book as not taken. 
\end{description}

\textbf{Important}. At this point you have to make a \emph{design
  decision}: should the information about the books ``being taken'' be
stored in the book class or in the library class? 
If this information is part of the book, you must
add some functionality to the Book class: maybe methods like
\verb+isTaken()+ and \verb+setTaken(boolean)+, and some private
boolean field; if the responsibility lies on the library's camp, then you must
add the adequate memory structures.

\subsection{}

Expand your library once more to include the following methods: 

\begin{description}
\item[getReaderCount()] returns the number of users registered in this
  library. 
\item[getBookCount()] returns the number of books in this library.
\item[getBookBorrowedCount()] returns the number of borrowed books in
  this library. 
\end{description}

\subsection{}

Add a method to your users that returns a list (or an array) with the
titles of all the books they are borrowing at the moment. 

\subsection{}

Expand your library class one more time with two methods that return: 

\begin{itemize}
\item a list (or an array) with all the users that are borrowing books
  at the moment.
\item a list (or an array) with all the users, borrowing or not.
\item the name of the person that is borrowing a specific given title
  at the moment; if nobody is borrowing the book, or the book does not
  exist in the library, you must return null (not an empty String). 
\end{itemize}

\subsection{}

Modify your method \verb+setMaxBookPolicy(int)+ in your library class,
so that it does not return \verb+void+ anymore. Instead, it should
return an array of users that have more books than the new policy
allows. The array may be empty. 

For example, if the maximum policy is three books per user, Marge,
Lisa, and Maggie may have borrowed one, two, and three books. If the
policy is then set to a maximum of one book, the method must return
the names of Lisa and Maggie (so that the library can track them down
and ask them to return the excess books).

\subsection{}

Once you have finished implementing all this functionality, write a
small script that illustrates these situations: 

\begin{itemize}
\item A user borrowing one book and then returning it.
\item A user trying to borrow more books than allowed. And then
  returning one of the books they already have to borrow a new one.
\item Several users borrow books. The library is then asked who has
  some specific title, if anyone. 
\item The ``maximum books'' is reduced to one, then to zero, then it
  is increased to the original value.  
\end{itemize}

\section{Final is final}
\label{sec:final-final}

Look at the following code. Is everything OK? How many errors can you spot?

\begin{verbatim}
    public class Person {
        private final String name;

        public Person(String name) {
            this.name = name;
        }

        public String getName()
            return this.name;
        }

        public void setName(String name) {
            this.name = name;
        }
    }
\end{verbatim}

Once you think you have found all the errors, try to compile the code
and see if your intuition was right. 



\end{document}