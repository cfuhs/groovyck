
\documentclass{article}
\usepackage[margin=2cm]{geometry}
\begin{document}

\section*{Learning goals}
\label{sec:learning-goals}

Before the next day, you should have achieved the following learning
goals: 

\begin{itemize}
\item Become familiarised with the use of \verb+private+ and
  \verb+public+. All your classes, fields, and methods should specify
  explicitly wheter they are public or private according to the rules
  of thumb in the notes. If you make a decision of visibility that
  deviates from those rules, you should explain why in a comment.
\item Related to the former point, you should become used to use
  constructors in all your classes. The constructor method or methods
  should be used to initialise the fields of any new object of that
  class.
\item Be able to create classes in their own \verb+.java+ file, 
  compile them using
  \verb+javac+, and use those classes from 
  Groovy or Java Decaf programs. 
\item Be able to cast simple types from one type to another.
\item Be able to create and use arrays in one or more dimensions. 
\end{itemize}

You should be able to finish most of non-star exercises in the
lab. Remember that star exercises are more difficult. Do not try them
unless the normal ones are clear to you. 

\section{Dividing integers}
\label{sec:casting}

Create a Java class called \verb+Calculator+. The class should
implement the following methods, each of them printing the result on
the screen. 

\begin{itemize}
\item \verb+add(int, int)+
\item \verb+subtract(int, int)+
\item \verb+multiply(int, int)+
\item \verb+divide(int, int)+
\item \verb+modulus(int, int)+
\end{itemize}

Note that you will will need to cast the parameters into \verb+double+
to perform exact division. 

Write a small Groovy or Java Decaf program that uses all the methods of
\verb+Calculator+.



\section{Copying arrays}
\label{sec:copying-arrays}

Create a new Java class called ArrayCopier with a method called
\verb+copy+ that takes two arrays of integers as parameters. The
method should copy the elements of the first array (you can call it
\verb+src+, from ``source'') to the second one (\verb+dst+, from
``destination'') as much as possible.

If the second array is smaller, then only those elements that fit will
be copied. If the second array is larger, it will be filled with
zeroes. 

Write a program that creates an object of this class and uses this
method to copy some arrays in all three cases: 

\begin{itemize}
\item Both arrays are of the same size.
\item The source array is longer.
\item The source array is shorter. 
\end{itemize}

\section{Creating matrices}
\label{sec:creating-matrices}

Create a class Matrix that has a 2-D array of integers as a field. The
class should have methods for: 

\begin{itemize}
\item a constructor method \verb+Matrix(int,int)+ setting the size of
  the array as two 
  integers (not necessarily the same). All elements in the matrix
  should be initialised to 1. 
\item a method \verb+setElement(int,int,int)+ to modify one element
  of the array, given its position 
  (the first two integers) and the new value to be put in that 
  position (the third integer).
  The method must check that the indeces are valid
  before modifying the array to avoid an
  \verb+IndexOutOfBoundsException+. If the indeces are invalid, the
  method will do nothing and the third argument will be ignored.
\item a method \verb+setRow(int,String)+ that modifies one whole row
  of the array, given its position 
  as an integer and the list of numbers as a String like ``1,2,3''. 
  The method must check that the index is valid and the numbers are
  correct (i.e.~if the array has three columns, the String contains
  three numbers). If the index or the String is invalid, the
  method will do nothing. 
\item a method \verb+setColumn(int,String)+ that modifies one whole 
  column of the array, given its position 
  as an integer and the list of numbers as a String like ``1,2,3''. 
  The method must check that the index is valid and the numbers are
  correct (i.e.~if the array has four rows, the String contains
  four numbers). If the index or the String is invalid, the
  method will do nothing.
\item a method \verb+toString()+ that returns the values in the array
  as a String using square brackets, commas, and semicolons,
  e.g. ``[1,2,3;4,5,6;3,2,1]''. 
\item A method \verb+prettyPrint()+ that prints the values of the
  matrix on screen in a legible format. Hint: you can use the
  special character '\textbackslash t' (backslash-t) 
  to mark a tabulator so that all numbers are
  placed in the same column regardless of their size. You can think of
  a tabulator character as a move-to-the-next-column command when
  printing on the screen. 
\end{itemize}

Create a Groovy program that uses all those methods from the Matrix class:
creates matrices, modifies its elements (one-by-one, by rows, and by
columns), and prints the matrix on the screeen. 

\section{One-liners for matrices (*)}
\label{sec:creat-matr-one}

Extend your \verb+Matrix+ class with a method \verb+setMatrix(String)+
that takes a String representing the numbers to be put in the elements
of the array separated by commas, separating rows by semicolons,
e.g.~1,2,3;4,5,6;7,8,9. 

\section{Symmetry looks pretty}
\label{sec:symm-looks-pretty}

Make a class \verb+MatrixChecker+ with three methods: 

\begin{itemize}
\item \verb+isSymmetrical(int[])+ takes an array of \verb+int+ and
  returns true if the array is symmetrical and false otherwise. An
  array is symetrical if the element at [0] is the same as the
  element at [length-1], the element at [1] is the same as the
  element at [length-2], etc.
\item \verb+isSymmetrical(int[][])+ takes an bidimensional array of
  \verb+int+ and returns true if the matrix is symmetrical and false
  otherwise. An matrix is symmetrical if \verb+m[i][j] == m[j][i]+ for
  any value of \verb+i+ and \verb+j+.
\item \verb+isTriangular(int[][])+ takes an bidimensional array of
  \verb+int+ and returns true if the matrix is triangular\footnote{A
    matrix can be up-triangular or low--triangular, but just checking
    one of the two is fine for this exercise.} and false
  otherwise. An matrix is triangular if \verb+m[i][j] == 0+ for
  any value of \verb+i+ that is greater than \verb+j+.
\end{itemize}

Add some methods to your \verb+Matrix+ class from the other exercise to
perform tests on the matrices you create using the methods from
MatrixChecker. (Hint: these methods will need to create objects of
type \verb+MatrixChecker+). 


\section{Anti-aircraft aim (*)}
\label{sec:anti-aircraft-aim}

Create an enumerated type \verb+Result+ in its own file. The
\verb+enum+ must have 8 possible values: HIT, FAIL\_LEFT, FAIL\_RIGHT,
FAIL\_HIGH, FAIL\_LOW, FAIL\_SHORT, FAIL\_LONG, OUT\_OF\_RANGE. Hint:
the \verb+enum+ must be \verb+public+.

Then create a Java class \verb+Target+ with the following methods: 

\begin{itemize}
\item A constructor method \verb+Target(int)+ that creates a 3-D array 
  of integers of the
  proposed size in all three dimensions. All elements must be set to
  zero.
\item A method called \verb+init()+ 
  that sets all the elements in the matrix to 0 except one
  ---selected randomly--- that will be set to 1. 
  Hint: Remember that you can get a random
  integer between 0 and N (not including N) by
  using \verb+int numberToGuess = (int) Math.abs(N * Math.random())+.
\item \verb+fire(int,int,int)+ a method that checks whether the
  element determined by the indexes is 1 and returns a type
  \verb+Result+ according to the result: \verb+Result.HIT+ if the
  element is 1, \verb+Result.FAIL_LEFT+ if the element of value one is
  ``to the left'' (you must decide what left and right are in your 3-D
  array), etc. If any of the indeces is too big (or negative), the
  method must return \verb+Result.OUT\_OF\_RANGE+. Left--right
  information takes precedence over high--low, and this takes
  precendence over short--long. If the 1 is to the left and behind,
  the output should be \verb+Result.FAIL\_LEFT+.
\end{itemize}

Write a small program that tells the user they must hit a flying
target, and then let the user try to find it by introducing three
indeces. The program should use an object of class \verb+Target+ to
know whether the user hit or not, and provide feedback
accordingly. Here is a sample out of such a program in a space $10 x
10 x 10$. 

\begin{verbatim}
    Here they come! Try to bring the plane down!
    Enter a coordinate X: 30
    Enter a coordinate Y: 4
    Enter a coordinate Z: 5
    That shot is way out of range. Try harder!
    Enter a coordinate X: 3
    Enter a coordinate Y: 4
    Enter a coordinate Z: 5
    You missed! The target is to the right!
    Enter a coordinate X: 5
    Enter a coordinate Y: 4
    Enter a coordinate Z: 1
    You missed! The target is farther!
    Enter a coordinate X: 5
    Enter a coordinate Y: 4
    Enter a coordinate Z: 5
    You hit it! Well done!
    Would you like to play again? y
    Here they come! Try to bring the plane down!
    Enter a coordinate X: 
\end{verbatim}


% Leave for assignment
% \section{Sinking ships (classic)}
% \label{sec:sinking-ships}

% Create a class \verb+Board+ with the following methods: 

% \begin{itemize}
% \item \verb+init(int)+ to create a 2-D array of integers of the
%   proposed size in both dimensions. All elements must be set to
%   zero. 
% \end{itemize}

% Hundir la flota
%   classical
%     painting the board
%   ships have one location, but different hit points and can move
%      class ship with hit points and position
%      more complicated: ships cannot share the same position


\section{Big enough (*)}
\label{sec:big-enough}

Write a small program that asks for the names and IDs of all employees
in a small company, and store them in an array of integers and an
array of Strings. (You will need to create a Java class to hold the
arrays, and to access them). 

This is similar to the example from the notes, but you do not know the
number of employees in advance. Read the names and IDs of employees
until the user enters an empty name (i.e. length 0) or an ID equal to
0.

Once you have finished reading employee data, go through the employee
list and print the names and IDs of
those employees whose ID is even or their names start with ``S''. 

(Hint: As you do not know how many employees you need in advance, you
will need a growing array. Create a small array, if it gets full
create an array twice as big, copy all data to the new array, and
discard the old array, etc).



\end{document}