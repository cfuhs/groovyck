\section{Arithmetic expressions}

In this section we will discuss arithmetic expressions, identifiers
and comments, numeric (integer) variables, conditionals, two-way
branches, and the ASCII code set.

\subsection{Integer variables}

As well as variables of type \verb!String! we can have variables of
type \verb!int!. We call them integer variables or just
\emph{integers}. Integers can contain any\footnote{Actually, computers
  are finite and therefore they put limits to the possible values for an
  integer variable, but they are big enough for most programs.}
integer value like~0,~-1,~-200, or~1966.

We could declare an \verb!int! variable as follows:

\begin{Verbatim}
    int i
\end{Verbatim}

and also initialise it if we wanted to:

\begin{Verbatim}
    int i = 0
\end{Verbatim}

We can change the value of an integer with assignment:

\begin{Verbatim}
    i = 1
\end{Verbatim}

or, if we had two integers \verb!i! and \verb!j!:

\begin{Verbatim}
    i = j
\end{Verbatim}

We can even say that \verb.i. takes the result of adding two numbers
together:

\begin{Verbatim}
    i = 2 + 2
\end{Verbatim}

which results in \verb.i. having the value 4. 

\subsection*{Integers and strings}
\label{sec:intstr}

You have probably noted that, when dealing with integer
variables, the symbol ``+'' represents addition of numbers while ---as
we saw in the last section--- when dealing with strings the same
symbol represents concatenation. Therefore, the statement

\begin{Verbatim}
    String str = "My name is " + "Inigo Montoya"
\end{Verbatim}

results in \verb+str+ having the value ``My name is Inigo Montoya'',
while the statement

\begin{Verbatim}
    int n = 10 + 7
\end{Verbatim}

results in \verb+n+ having the value 17, not 107. What happens if we
mix integer and string variables? In that case, Groovy converts all
integers to strings and the small program

\begin{Verbatim}
    String str = "My name is " + "Inigo Montoya"
    int n = 10 + 7
    String text = str + " and I am " + n
    println text
\end{Verbatim}

will print on the screen ``My name is Inigo Montoya and I am 17''. It
is important to know that the same symbol can be used for different
things, but we will come to this later again; now let's go back to
making programs with a bit of maths in them using integer variables.

\subsection{Reading integers from the keyboard}
\label{sec:intkeyboard}

In the last section, we saw how we can read a string of characters
from the keyboard, 
using \verb+System.console().readLine()+. We can use the same command
to read a number\ldots but the computer will not know it is a number,
it will think it is a string of characters. If we want to tell the
computer that a number \emph{is} a number, we need to convert it. We
can do it easily by \emph{parsing} it with the command
\verb+Integer.parseInt()+: 

\VerbatimInput[frame=single,label=Example]{src/s2Example1.groovy}

When we parse a string that contains an integer we obtain an integer
with the right value. If we try to parse something that is not an
integer (for example, a word) the program will terminate with an error
message on the screen. If we do not parse the string and use it as if
it was an integer, the results will be crazy and unpredictable. This
is a common source of errors in programs. You can check for yourself
what happens if you do not parse the string in the former example. 

Now, assuming that you read your integers and always remember to parse
them, what maths can you do with them?

\subsection{Operator precedence}
\label{sec:prec}

Groovy uses the following arithmetic operators (amongst others):

\begin{tabular}{ll}
\verb!+! & addition\\
\verb!-! & subtraction\\
\verb!*! & multiplication\\
\verb!/! & division\\
\verb!%! & modulo\\
\end{tabular}

The last one is perhaps unfamiliar.  The result of \verb!x \% y! (``x mod y'')
is the remainder that you get after dividing the integer x by the integer y.
For example, \verb!13 % 5! is \verb!3!; \verb!18 % 4! is \verb!2!; 
\verb!21 % 7! is \verb!0!, and \verb!4 % 6 is 4! (6 into 4 won't go, 
remainder 4).  \verb!num = 20 \% 7! would assign the value 6
to \verb!num!.

How does the computer evaluate an expression if there is more than one
operator in it? For example, given \verb!2 + 3 * 4!, does it do the
addition first, thus getting \verb!5 * 4!, which comes to 20, or does
it do the multiplication first, thus getting \verb!2 + 12!, which
comes to 14?  Groovy, in common with other programming languages and
with mathematical convention in general, gives precedence to *, / and
\% over + and \verb!-!.  This means that, in the example, it does the
multiplication first and gets 14.

If the arithmetic operators are at the same level of precedence, it
takes them left to right.  \verb!10 - 5 - 2! comes to 3, not 7.  You
can always override the order of precedence by putting brackets into
the expression; (2 + 3) * 4 comes to 20, and \verb!10 - (5 - 2)! comes
to 7.

Some words of warning are needed about division. First, remember that
\verb-int- variables can only store integer values. If you try to
store the result of a division in an \verb-int- variable, you will
store only the integer part and lose the decimal part. Try to
understand what the following program does:

\VerbatimInput[frame=single,label=Example]{src/s2Example.groovy}

A computer would get into difficulty if it tried to divide by zero.
Consequently, the system makes sure that it never does.
If a program tries to get the computer to divide by zero, the program
is unceremoniously terminated, usually with an error message on the
screen.

\subsection*{Exercise A}

Write down the output of this program without executing it:

\VerbatimInput[frame=single,label=Example]{src/s2Example.groovy}

Now execute it and check if you got the right values. Did you get them
all right? If you got some wrong, why was it?


\subsection{Identifiers and comments}

I said earlier that you could use more or less any names for your variables.
I now need to qualify that.

The names that the programmer invents are called \emph{identifiers}.  The
rules for forming identifiers are that the first character can be a letter
(upper or lower case, usually the latter) and subsequent characters
can be letters or digits 
or underscores.  (Actually the first character can be an underscore but
identifiers beginning with an underscore are often used by system programs
and are best avoided.)  Other characters are not allowed.  Groovy is
case-sensitive, so \verb!Num!, for example, is a different identifier from
 \verb!num!.

The only other restriction is that you cannot use any of the language's keywords
as an identifier.  You couldn't use \verb!int! as the name of a variable,
for example. There are many keywords but most of them are words that you
are unlikely to choose. 
Ones that you might accidentally hit upon are \texttt{break},
\texttt{case}, \texttt{catch}, \texttt{class}, \texttt{const},
\texttt{continue}, \texttt{double}, \texttt{final}, \texttt{finally},
\texttt{float}, \texttt{import}, \texttt{long}, \texttt{new},
\texttt{return}, \texttt{short}, \texttt{switch}, \texttt{this},
\texttt{throw}  and \texttt{try}. You should also avoid 
using words which, though not technically keywords, have special significance
in the language, such as \verb!println! and \verb!String!.

Programmers often use very short names for variables, such as
\verb!i!, \verb!n!, or 
\verb!x! for integers.  There is no harm in this if the variable is used to
do an obvious job, such as counting the number of times the program goes round
a loop and its purpose is immediately clear from the context.  If, however, its
function is not so obvious, it should be given a name that
gives a clue as to the role it plays in the program.  If a variable is
holding the total of a series of integers and another is holding the
largest of a series of integers, for example, call them \verb!total!
and \verb!max! rather than \verb!x! and \verb!y!.

The aim in programming is to write programs that are ``self-documenting'',
meaning that a (human) reader can understand them without having to read
any supplementary documentation.  A good choice of identifiers helps to
make a program self-documenting.

Comments provide another way to help the human reader to understand a
program.  Anything on a line after ``//'' is ignored by the compiler,
so you can use this to annotate your program.  You might summarise
what a chunk of program is doing:

\begin{Verbatim}
    // sorts numbers into ascending order
\end{Verbatim}

or explain the purpose of an obscure bit:

\begin{Verbatim}
    x = x * 100 / y   // x as percent of y
\end{Verbatim}

Comments should be few and helpful.  Do not clutter your programs with
statements of the obvious such as:

\begin{Verbatim}
    num = num + 1   // add 1 to num
\end{Verbatim}

Judicious use of comments can add greatly to a program's readability, but they
are second best to self-documentation.  Their weakness is that it is all too
easy for a programmer to modify the program but forget to make any
corresponding modification to the comments, so the comments no longer quite
go with the program.  At worst, the comments can become so out-of-date as
to be positively misleading.

\subsection*{Exercise B}

Say for each of the following whether it is a valid identifier
in Groovy and, if not, why not:
\begin{Verbatim}
BBC, Groovy, y2k, Y2K, old, new, 3GL, a.out, 
first-choice, 2nd_choice, third_choice
\end{Verbatim}

\subsection{A bit more on String variables}

We have done already many things with string variables: we have
created them, printed them on the screen, even concatenated many of
them together to get a longer value. But there are many other useful
things we can so with strings. 

For example, if you want to know how long a string is, you can find
out with the \verb!length()! 
method. If  \verb!str! is the string, \verb!str.length()! is the
length. Do not forget the dot or the brackets: methods always start
with a dot and end with brackets\footnote{Now you can see that
  console() and readLine() are also methods.} (sometimes empty,
sometimes not). You could say, for example:

\begin{Verbatim}
    print "Please enter some text: "
    String str = System.console().readLine()
    int len = str.length()
    println "The string " + str + " has " + len + " characters."
\end{Verbatim}

And you can obtain a substring of a string with the \verb!substring! function.
For example if the string \verb!s! has the value ``\verb!Silverdale!'', then
\verb!s.substring(0,6)! will give you the first six letters, i.e., the string
``\verb!Silver!''.  The first number in brackets after the \verb!substring!
says where you want the substring to begin, and the second number says how
long you want it to be. \emph{Note that the initial character of the string
is at position~0, not position~1}.  If you leave out the second number,
you get the rest of the string.  For example, \verb!s.substring(6)! would
give you ``\verb!dale!'', i.e., the tail end of the string beginning at
character 6 ('d' is character 6, not 7, because the first one is
character 0). 

You can output substrings with \verb!print! or assign them to other
strings or combine them with the ``+'' operator.  For example:

\VerbatimInput[frame=single,label=Example]{src/s2Example2.groovy}

will output ``soft ices''.

\subsection*{Exercise C}

Say what the output of the following program fragment would be:

\VerbatimInput[frame=single,label=Example]{src/s2Example3.groovy}

Then run the program and see if you were right. 


%%% Local Variables:
%%% mode: latex
%%% TeX-master: "main"
%%% End:
