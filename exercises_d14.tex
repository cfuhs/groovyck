\documentclass{article}
\usepackage[margin=2cm]{geometry}
\usepackage[dvips]{graphicx}
\begin{document}

\section*{Learning goals}
\label{sec:learning-goals}

Before the next day, you should have achieved the following learning
goals: 

\begin{itemize}
\item Understand that storing intermediate values for reuse
  (memoization) can largely speed up some computations. 
\item Implement divide-and-conquer solutions to problems. 
\end{itemize}

\subsection{Memoized Fibonacci}

Write a Java class with a with a static method that calculates 
the n$^{th}$ element of the Fibonacci sequence as seen in the notes.

Do it with and without memoization. Compare
the time that is needed in each case to find F(40) or F(45). 

\section{Anagrams}
\label{sec:anagrams}

An anagram is a word created by recombination of the letters in
another words. Some anagrams make sense (``silent'', ``listen'') and
some do not (``nietsl''). Some people only accepts ``real'' anagrams,
but in this exercise we will accept them all, even if they do not
exist as real words. 

Write a class with a static method that takes a String as a parameter
and returns a list (hint: you can use \verb+List<String>+ and
\verb+ArrayList<String>+) of strings with all the anagrams that can be
made with it. 

Is it easy to do this both iteratively and recursively? Is this
similar to a former exercise?

\section{Binary search}
\label{sec:binary-search}

The most basic example of divide-and-conquer strategies is the binary
search. This is used to look for an element \emph{in a sorted list}. 

We can find an element in a list by traversing through the whole list
and checking whether each element is the one we are looking for. The
number of comparisons that we need by using this algorithm is
proportional to the length of the list. If we know that the list is
sorted, we can do better with a divide-and-conquer strategy, 
like the one defined by repeating these steps: 

\begin{description}
\item[Initial action: ] If the list is empty, it does not contain the
  element and we have finished. 
  If it is not empty, check the middle element, i.e.~the
  element at \verb+list.size()/2+. If it is the element we are looking
  for, we have finished. 
\item[Subproblem: ] If the element we are looking for is before the
  middle element, the next list to search into is the first half of
  the original list; otherwise, it is the second half.
\item[Integration: ] No need for integration in this case. Just repeat
  looking into half-lists until the sublist is only one element long 
  At that point, either the element is the one we are looking for or
  the list does not contain it. 
\end{description}

Implement a binary search algorithm for a list of integer numbers. The
method returns true if the list contains the element and false
otherwise. 

You
can use the classes in the Java Collection Library. Search for numbers
in lists with 10, 100, and 1000 elements; how many comparisons do you
need in each case? (Hint: instead on entering manually 1000 elements
in order, maybe you can implement one of the sorting algorithms in the
following exercises and then use them to order a list of random
numbers. Remember that you can create a random integer between 0 and
N-1 with \verb+Math.abs(N * Math.random())+).

\section{Mergesort}
\label{sec:mergesort}

Mergesort is a popular sorting algorithm that employs a
divide-and-conquer strategy. You can implement a Mergesort for lists
by following the following steps: 

\begin{description}
\item[Subproblem: ] If the list contains 0 or 1 element, it is sorted
  and you can return it. If not, then divide the list into two lists
  of the same length ($\pm$ 1). Then sort the lists (i.e.~apply
  mergesort to each sublist).
\item[Integration: ] When both sublists are returned sorted, check the
  first element of both sublists; remove the one that comes first
  (e.g.~the lowest integer of the two) and add it to the result
  list. Repeat this process until all elements in both sublists have
  been added to the result list. Return the result list. 
\end{description}

Example with five elements: 

\begin{verbatim}
          [3, 7, 2, 9, 1]            
       [3, 7, 2]      [9, 1]         (subproblem)
    [3, 7]    [2]     [9] [1]        (subproblem)
    [3] [7]   [2]     [9] [1]        (subproblem)
    [3, 7]    [2]     [9] [1]        (integration of [3] and [7])
      [2, 3, 7]      [9] [1]         (integration of [3, 7] and [2])
      [2, 3, 7]      [1, 9]          (integration of [9] and [1])
          [1, 2, 3, 7, 9]            (integration of [2, 3, 7] and [1, 9])
\end{verbatim}

\section{Quicksort}
\label{sec:quicksort}

Quicksort is another sorting algorithm that also employs a
divide-and-conquer strategy. It works well in most usual computers
because of the low-level interactions between registers and the main
memory, which make it very popular and widely used. 

You can implement a Mergesort for lists
by following the following steps: 

\begin{description}
\item[Initial action: ] If the list contains 0 or 1 element, it is
  sorted and you can return it. Otherwise, choose one element as
  ``pivot'' (usually the first one). 
\item[Subproblem: ] Divide the
  list into two lists: the first list contains the elements before the
  pivot (e.g.~the integers lower than the pivot) and the second one
  contains the elements after the pivot. Then sort both lists
  (i.e.~apply quicksort to each sublist, choosing a new pivot, etc).
\item[Integration: ] When both sublists are returned sorted,
  simply concatenate them (first list, then pivot, then second list)
  and return the result.
\end{description}

Example with five elements: 

\begin{verbatim}
          [3, 7, 2, 9, 1]            
       [2, 1]   3   [7, 9]           (subproblem, pivot: 3)
    [1] 2 []    3   [7, 9]           (subproblem, pivot: 2)   
    [1] 2 []    3   [] 7 [9]         (subproblem, pivot: 7)
      [1,2]     3   [] 7, [9]        (integration of the sublists of pivot 2)
      [1,2]     3    [7, 9]          (integration of the sublists of pivot 7)
         [1,2, 3, 7, 9]              (integration of the sublists of pivot 3)
\end{verbatim}

\section{Finding the roots of a polynomial (*)}
\label{sec:find-roots-polyn}

Polynomials are a special type of mathematical function with a lot of
applications in mathematics and computer science ---from scientific
calculations to games rendering--- because they have a lot of nice
properties. For instance, they are always continuous and infinitely
derivable for any real number, and can be used to approximate other
functions. A polynomial is a function of the form:

$$ a_0 + a_1 \cdot x + a_2 \cdot x^2 + \ldots + a_n \cdot x^n $$ 

where the $a_i$ are called \emph{coeficients} and the $x$ is a variable. A
polynomial can easily be represented by an array containing the $a_i$
coeficients. For example, the array \verb+[5, 0, 3, -2]+ represents
the polynomial: 

$$ 5 + 0 \cdot x + 3 x^2 - 2 x^3 = 5 + 3 x^2 - 2 x^3 $$

When working with polynomials, it is common to search for their
\emph{roots}, i.e.~the values of $x$ that make them evaluate to
zero. For instance, the polynomial $6 - 5 x + x^2$ has roots 
at $x = 2$ and at $x = 3$: 

$$ 6 - 5 x + x^2 |_{x = 2} = 6 - 5 \cdot 2 + 2^2 = 6 - 10 + 4 = 0 $$
$$ 6 - 5 x + x^2 |_{x = 3} = 6 - 5 \cdot 3 + 3^2 = 6 - 15 + 9 = 0 $$

Create a method that takes an array (minimum size 2) representing a
polynomial and find whether it has a root in the range from -1000.0 to
+1000.0. Use binary search to find the root knowing that: 

\begin{itemize}
\item A polynomial of grade $n$ (i.e.~array length n+1) has at most
  $n$ real numbers as roots 
  (but they may be out of the [-1000, 1000] range).
\item If a polynomial has a positive value at $x = A$, and a negative
  value at $x = B$ (or viceversa), then it has at least one root in the
  interval [A,B]. In other words, if we know that it is continuous,
  and it turns from positive to negative or viceversa, then we know
  that there is at
  least one point in which it is zero. 
\end{itemize}

Find the root with a precision of six decimals. In other words,
assuming that you have a method \verb+eval(double)+ that evaluates the
polynomial for a certain value of $x$, and a \verb+static final+
variable called \verb+PRECISION+ with value \verb+0.000001+;  if:

\begin{verbatim}
    Math.abs((polynomial(x) - 0) < PRECISION)
\end{verbatim}

then \verb+x+ is a root of the polynomial. Remember that rounding errors make
it unlikely that any number will evaluate to exactly 0.0, so we can
only search for double numbers up to a certan precision

Different applications need different precisions. GPS positioning has
a precision of a few meters, the design of a
plane usually works with precision up to one milimeter,
and the measurements for the latests scientific experiments in
quantum physics require the best precision that is achievable by
state-of-the-art technology. 



% Exercises
%   - Greatest common divisor ( p > q => gcd(p,q) = gcd(q, p % q) )
%   - Koch star?
%   - Memoization: Fibonacci numbers
%   - Increments and unfolding
%   - Dominoes
%   - Finding longest common subsequence
%
%
%   - Root finding for a polynomial (one between -1000 and 1000 if
%       sign(f(min)) != sign(f(max)) , as many as possible for **)


\end{document}