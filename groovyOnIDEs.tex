
\section{A note for IDE users}
\label{sec:note-ide-users}

Some of the readers of this document will desire to use an Integrated
Development Environment (IDE) while they try to solve the exercises
described in the preceding 
pages\footnote{Examples of popular IDEs are Eclipse and IntelliJ. }. 
This section is for them and for
them only. Readers that do not want or do not intend to use an IDE can
safely skip it. 

The notes assume that you, the reader, will use 
only a simple editor and the command line to run your scripts. 
\textbf{This is the simplest scenario
and this is what we recommended.} 
In that context, using the command
\verb+System.console().readLine()+ to read a line of text from the
user will work as described in the notes. However, this is not true if
you use an IDE for reasons that are a bit complex to explain at this
basic level.

Once again, our initial advice is to run all your programs from the
command line. This is usually the simplest route forward for the
novice programmer. IDEs are extremely useful tools in the hands of a
seasoned programmer but tend to be distracting and confusing for
novices, especially in the very early stages, and that is why we
recommend to avoid them for now. However, if you insist on using an
IDE, keep reading. 

If you really want to use an IDE to write and execute a program that
receives input from the user then you will have to use a
\verb+Scanner+. This is a bit more complex than using
\verb+System.console().readLine()+ but not that much.

\begin{enumerate}
\item First, you have to declare and instantiate the scanner at the
beginning of your program:

\begin{verbatim}
     Scanner scanner = new java.util.Scanner(System.in);
\end{verbatim}

\item Then you can use it like this:

\begin{verbatim}
    /* not exactly equivalent to                     */
    /* System.console().readline(), but close enough */
    String str = scanner.next();
\end{verbatim}

\end{enumerate}


\section{A note for users of the Groovy Console}
\label{sec:note-ide-console-groovy-users}

Modern installations of Groovy for Windows come with a program called
something like ``Groovy Console''. We recommend readers of this
document \textbf{not} to use it and to use the normal windows console (or
``Command Prompt'') instead. 

Although the Groovy Console is not an IDE, it lacks a console in
the same way IDEs do, so the command
\verb+System.console().readLine()+ does not work for reading a line of
text from the user. In order to make your programs run on the Groovy
Console, you must use a \verb+Scanner+ as described in
Section~\ref{sec:note-ide-users}. 

But the best option is ---once again--- not to use the Groovy Console. 

%%% Local Variables:
%%% TeX-master: "main-aptitude-test-notes"
%%% End:

