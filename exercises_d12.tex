\documentclass{article}
\usepackage[margin=2cm]{geometry}
\usepackage[dvips]{graphicx}
\begin{document}

\section*{Learning goals}
\label{sec:learning-goals}

Before the next day, you should have achieved the following learning
goals: 

\begin{itemize}
\item Create generic classes
\item Use generic classes
\item Overload methods
\item Prevent code repetition in overloaded methods
\item Upcast objects to more general types
\item Downcast objects to more specific types
\end{itemize}

\section{Don't Repeat Yourself}
\label{sec:dont-repeat-yourself}

Look at the following code. Is there anything you can do to make this
code better? Hint: you may need to convert between types
(e.g.~casting). 

\begin{verbatim}
    public class Comparator {
        public int getMax(int n, int m) {
            if (n > m) {
                return n;
            } else {
                return m;
            }
        }
        public double getMax(double d1, double d2) {
            if (d1 > d2) {
                return d1;
            } else {
                return d2;
            }
        }
        public String getMax(String number1, String number2) {
            int n1 = Integer.parseInt(number1);
            int n2 = Integer.parseInt(number2);
            if (n1 > n2) {
                return number1;
            } else {
                return number2;
            }
        }
    }
\end{verbatim}

% Prevent code repetition in method overloading: specific methods call
% general method

\section{Upcasting, downcasting}
\label{sec:upcast-downc}

For this exercise, you will need to use again some classes and
interfaces you created last day: \verb+Phone+, \verb+OldPhone+,
\verb+MobilePhone+, \verb+SmartPhone+. 


\subsection{Start}

Create a script that builds a new \verb+SmartPhone+ with the line: 

\begin{verbatim}
    Smartphone myPhone = new Smartphone();
\end{verbatim}

and then uses all its methods. 

\subsection{Direct upcasting}
\label{sec:b}

Change the script so that the \verb+SmartPhone+ is created with the
line: 

\begin{verbatim}
    Mobilephone myPhone = new Smartphone();
\end{verbatim}

Compile your code again. Are there any problems? Why do this problems
happen? What are the possible solutions?

\subsection{Indirect upcasting when calling a method}
\label{sec:c}

Pass this object to a method \verb+testPhone(Phone)+ 
that has only one parameter of type
\verb+Phone+. What methods can you call on the object inside the method?

\subsection{Downcasting}
\label{sec:downcasting}

Inside the former method, downcast the object to \verb+Smartphone+ so
that you can use all the public methods of \verb+Smartphone+. 

\subsection{Casting exception}
\label{sec:casting-exception}

Create a \verb+MobilePhone+ object and then pass it to method
\verb+testPhone(Phone)+. What happens?

\section{Generic list}
\label{sec:generic-list}

Modify the doubly-linked list that you have created in past weeks to make it
generic, i.e.~to allow it to have values of its elements of any
type. 
% Use the TDD methodology (interface, test, production code,
% repeat\ldots).

Once you have it ready, create a class Company that keeps two linked
lists, one with the names of the employees and one with their National
Insurance Number. 

\section{Sorted list (*)}
\label{sec:sorted-list}

Extend your class from the former exercise to create a sorted
list. You may need to override the method that adds new elements to
the list. The subclass should be generic. 


% Use the TDD methodology.

\section{Generic stack}
\label{sec:generic-stack}

Create a generic stack (with methods for pushing, poping, and checking
emptiness) that only works with classes that extend \verb+Number+
(e.g.~Integer and Double, but not String). 
% Use the TDD methodology
% (interface, test, production code, repeat\ldots).

\section{Generic maps}
\label{sec:generic-map}

\subsection{Simple map (*)}
\label{sec:array-based-impl2}

Create a generic simple map (with methods for putting a key--value pair,
getting the value for a key, and removing a key). The key and the
value may be any type, and they may be different. Each key can only be
linked to one value. 
% Use the TDD methodology
% (interface, test, production code, repeat\ldots).

For simplicity, assume that your map can hold a maximum of 1000
pairs. This way, you can use the hashing method you developed in past
weeks and base your map on an array. 

\subsection{Hash table (*)}
\label{sec:array-based-impl}

Create a generic map (with methods for putting a key--value pair,
getting the value for a key, and removing a key). The key and the
value may be any type, and they may be different. Under each key, the
hash table can store any number of values associated to that key. 
% Use the TDD methodology
% (interface, test, production code, repeat\ldots).

For simplicity, assume that your map can hold a maximum of 1000
pairs. This way, you can use the hashing method you developed in past
weeks and base your map on an array. 



\end{document}