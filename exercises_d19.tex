\documentclass{article}
\usepackage[margin=2cm]{geometry}
\usepackage[dvips]{graphicx}
\begin{document}

\section*{Learning goals}
\label{sec:learning-goals}

Before the next day, you should have achieved the following learning
goals: 

\begin{itemize}
\item Understand what a client--server architecture is.
\item Launch a server and leave it waiting and listening on one
  machine. 
\item Launch clients that connect to a server on the same machine
  (localhost). 
\item Launch clients that connect to a server on another machine. 
\end{itemize}

The exercises for today are examples of simple network services that
you can implement in a client-server way. For each of the exercises,
you will need to define an interface, implement such interface through
a server object, and then implement a client that uses the services
implemented by the server. 

You will also need to write a security
policy file based on the example in the notes.

\section{echo}
\label{sec:echo}

This simple service receives a String and returns the same
String. (Hint: This is the example we have seen in the notes.) 

This service has traditionally listened at port 7.

\section{date}
\label{sec:date}

This service answers requests for the time and date with the current time and 
date. You can access this at the server by using classes Calendar and Data (in
package java.util.*).

A similar service has traditionally listened at port 37.

\section{Calculator}
\label{sec:calculator}

This service exports simple methods for addition, subtraction, multiplication, 
and division. (In the real world, this kind of service usually provides services
like complicated financial calculations). 

\section{Remote file access}
\label{sec:remote-file-access}

This service provides the state of a file on the server machine. The client must
provice a route and file name, which may or may not exist. If it exists, the 
server will return the content of the file as a String. Note: For simplicity, focus 
only on text files. 

\section{Finger (*)}
\label{sec:finger}

This service says who is connected on the machine. You will simulate
this by having two types of clients: \emph{login} clients and
\emph{finger} clients. You will need two servers, or one server that
can serve both services.

Login clients one will "connect" users to the server by sending pairs 
\emph{(user, password)}. These pairs may or may not be correct. The
user will be considered connected if and only if they are correct. 

Finger clients will query about which users are connected, receiving
both a set of logins and information about when those users did
connect to the server. 

A similar service has traditionally listened at port 79.

\end{document}
