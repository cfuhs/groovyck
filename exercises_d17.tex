\documentclass{article}
\usepackage[margin=2cm]{geometry}
\usepackage[dvips]{graphicx}
\begin{document}

\section*{Learning goals}
\label{sec:learning-goals}

Before the next day, you should have achieved the following learning
goals: 

\begin{itemize}
\item Launching different thread in a program
\item Having a general understanding of how a non-deterministic program
  works. 
\item Synchronising access to shared resources
\end{itemize}

\section{Text loops}
\label{sec:text-loops}

Look at the following code (comments ommitted for brevity). What will
the output be for each of the ``thread'' and the ``no threads'' modes? 

\begin{verbatim}
    public class TextLoop implements Runnable {
        public static final int COUNT = 10;
    
        private final String name;
    
        public TextLoop(String name) {
            this.name = name;
        }
    
        @Override
        public void run() {
            for (int i = 0; i < COUNT; i++) {
                System.out.println("Loop:" + name + ", iteration:" + i + ".");
            }
        }
    
        public static void main(String args[]) {
        if (args.length < 1 || (!args[0].equals("0") && !args[0].equals("1"))) {
            System.out.println("USAGE: java TextLoop <mode>");
            System.out.println("     mode 0: without threads");
            System.out.println("     mode 1: with threads");
        } else if (args[0].equals("0")) {
            for (int i = 0; i < 10; i++) {
                Runnable r = new TextLoop("Thread " + i);
                r.run();
            }
        } else {
            for (int i = 0; i < 10; i++) {
                Runnable r = new TextLoop("Thread " + i);
                Thread t = new Thread(r);
                t.start();
            }
        }
    }
\end{verbatim}

Compile and execute this code several times. Do you get the result you
expected? 

\section{Counting}
\label{sec:counter}

Have a look at the following code. What will be the value of the
counter at the end of its execution (comments ommitted for brevity)? 

\begin{verbatim}
    public class Increaser implements Runnable {
       private Counter c;
    
       public Increaser(Counter counter) {
            this.c = counter;
       }
    
       public static void main(String args[]) {
          Counter counter = new Counter();
          for (int i = 0; i < 100; i++) {
             Increaser increaserTask = new Increaser(counter);
             Thread t = new Thread(increaserTask);
             t.start();
          }
       }
    
       public void run() {
          System.out.println("Starting at " + c.getCount());
          for (int i = 0; i < 1000; i++) {
             c.increase();
          }
          System.out.println("Stopping at " + c.getCount());
       }
    }
    
    public class Counter {
       private int n = 0;
       public void increase() {
          n++;
       }
       public int getCount() {
          return n;
       }
    }
\end{verbatim}

Compile and execute this code several times. Do you get the result you
expected? Do you get always the same result? What would you change to make
sure the last value of the counter is what it should be?

\section{Bank account}
\label{sec:bank-account}

Look at the following code of a simplified bank account (comments
ommitted for brevity). Assume that
there are many threads accessing this object and think what is the
minimum number of \verb+synchronized+ statements that are needed to
ensure a correct behaviour. 

\begin{verbatim}
    public class BankAccount {
        private int balance = 0;
        public int getBalance() {
            return balance;
        }
        public void deposit(int money) {
            balance = balance + money;
        }
        public int retrieve(int money) {
            int result = 0;
            if (balance > money) {
                 result = money;
            } else {
                 result = balance;
            }
            balance = balance - result;
            return result;
        }
    }
\end{verbatim}

Check your answer with a colleage or with one of the members of the
faculty. 

\section{Responsive UI}
\label{sec:responsive-ui}

Write a program that asks from the user the length in milliseconds of
ten tasks. As the user enters the length, the tasks start running in
the background while the user enters the length of the other
tasks. When the tasks end, the program must register it and say it
unless it is waiting for the user to enter data. See this sample
output: 

\begin{verbatim}
    Enter the duration (in ms) of task 0: 10
    Enter the duration (in ms) of task 1: 3000
    Finished tasks: 0
    Enter the duration (in ms) of task 2: 2000
    Enter the duration (in ms) of task 3: 1000
    Enter the duration (in ms) of task 4: 10
    Enter the duration (in ms) of task 5: 1000
    Finished tasks: 2, 1, 3, 4
    Enter the duration (in ms) of task 6: 
    ...
\end{verbatim}

\section{Parallel computation}
\label{sec:parallel-computation}

Look at the attached program \verb+ComputationLauncher+ for an example
of a heavy computation being performed in sequence or in parallel
using more than one processor at the same time. In an old machine with
two processors, the output looks like: 

\begin{verbatim}
    Result: 7.999582837247596E14
    Time without threads: 11110ms
    Result: 7.999582837247596E14
    Time with threads: 6326ms
\end{verbatim}

Make sure you understand how the program works. How would you modify
the program if you machine had four processors? You can see how many
processors (or cores) your machine has by reading the value of: 

\begin{verbatim}
    Runtime.getRuntime().availableProcessors();
\end{verbatim}

\section{Immutability (*)}
\label{sec:immutability}

Look at the attached program \verb+ImmutableExample+. Read it
carefully. Do you see any flaws? If yes, what whould you change to
make the program work without problems?

What whould you change to make the IDCard class immutable?

\section{Self-ordering list (*)}
\label{sec:self-ordering-list}

One of the advantages of using threads in some applications is that
they allow programmers to do costly operations in the background when
nobody is looking. For example, a self-sorting list can reorder itself
in between calls to minimise the time needed to add elements (e.g.~in
applications where elements are added in big batches, but consulted
only rarely).

Hint: For this exercise, it may be easier to create your own dynamic
list instead of relying on those from the Java Collections
Library. Remember that the latter are not thread-safe, i.e.~they are
not properly synchronised, so they require external synchronisation. 

\subsection{a)}

Create a list of Integer that keeps itself sorted by sorting itself
when others are not looking. The list must have at least methods
\verb+add(Integer)+ (to add new Integers) and \verb+get(int)+ (to get
the i$^{th}$ integer in the (sorted) list). 

\begin{itemize}
\item The \verb+add()+ method must add the new element at the end
  of the list, mark the list as ``not sorted'' and return immediately.
\item Another thread must be always observing the list and checking whether
  it is sorted; if it is, it must mark it as ``sorted''; if it is
  not, if must sort it. The thread must order the list in small steps
  to make sure it does not cause additional delay when new elements
  are added to the list while it is sorting itself (i.e.~do not
  include a long loop in your synchronised code that may block the
  addition of new elements for a long time).
\item The method \verb+get(int)+ must always return the i$^{th}$
  integer in the (sorted) list; if this method is called while the
  list is ``not sorted'', the list must be fully sorted before it can
  return; no additional elements can be added until the call of
  \verb+get(int)+ returns.
\end{itemize}

\subsection{b)}
\label{sec:b}

Create another self-ordering list in which the \verb+add()+ method
launches a new thread whose only purpose is to add the new integer at
the right position. (Hint: if the new element has to be placed at the
beginning of the list, there is no need to launch a thread only for
that). 

% Reordering a list with several threads at the same time

\section{Dining philosophers (*)}
\label{sec:dining-philosophers}

A group of weak-sighted philosophers gathered around a circular table
to have dinner together. They were served on one big round plate in
the middle of the table, and they were given a fork each. In order to
prevent everybody to take food at the same time, risking their clumsy
hands hacking at their neighbours' with the forks, 
the philosophers discussed and agreed to adhere to the following
protocol: each philosopher could only get food if it had grabbed both
forks on their left and their right. 

Write a program that implements a dinner with $n$ philosophers and $n$
forks. Make sure that each fork can be grabbed by only one philosopher
at a time. Verify that a naive synchronisation strategy can lead very
quickly to a deadlock (when/why does this happen?); 
fix the program so that this cannot happen. 

(Hint: during development, it may be helpful to implement a monitoring
class (waiter?)  that checks what the philosophers are doing, e.g.~how
many forks they have grabbed).



\end{document}