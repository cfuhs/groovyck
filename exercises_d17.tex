\documentclass{article}
\usepackage[margin=2cm]{geometry}
\usepackage[dvips]{graphicx}
\begin{document}

\section*{Learning goals}
\label{sec:learning-goals}

Before the next day, you should have achieved the following learning
goals: 

\begin{itemize}
\item Launch different thread in a program
\item Synchronize access to shared resources
\end{itemize}

\section{Counting}
\label{sec:counter}

Have a look at the following code. What will be the value of the
counter at the end of its execution?

\begin{verbatim}
    public class Increaser implements Runnable {
       private Counter c;
    
       public Increaser(Counter counter) {
            this.c = counter;
       }
    
       public static void main(String args[]) {
          Counter counter = new Counter();
          for (int i = 0; i < 100; i++) {
             (new Thread(new Increaser(counter))).start();
          }
       }
    
       public void run() {
          System.out.println("Starting at " + c.getCount());
          for (int i = 0; i < 1000; i++) {
             c.increase();
          }
          System.out.println("Stopping at " + c.getCount());
       }
    }
    
    class Counter {
       private int n = 0;
       public void increase() {
          n++;
       }
       public int getCount() {
          return n;
       }
    }
\end{verbatim}

Compile and execute this code several times. Do you get the result you
expected? Do you get always the same result? What can you do to make
the last value of the counter what it should be?

% Dining philosophers 

% Exercises from Sokratis

% Adding things to a list 

% Having a list with an internal thread that keeps it sorted

% Reordering a list with several threads at the same time

\end{document}